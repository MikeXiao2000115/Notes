\documentclass[UTF8]{ctexart}

\usepackage{geometry}

\usepackage{amsmath}
\usepackage{amssymb}
\usepackage{esint}
\usepackage{yhmath}
\usepackage{bm}

\usepackage{fancyhdr}
\usepackage{graphicx}

\special{papersize={18.1cm,25.7cm}}
\geometry{left=1.5cm,right=0.5cm,top=2cm,bottom=1cm}
\pagestyle{empty}

\newcommand{\D}{{\text{d}\;\!}}
\newcommand{\cross}{\times}
\newcommand{\dif}[1]{{\mathrm{d}\;\!#1}}
\newcommand{\dev}[1]{{\frac{\text{d}}{\dif{#1}}\;\!}}
\newcommand{\ve}[1]{{\bm{#1}}}
\newcommand{\ven}[2]{{\left\langle#1,#2\right\rangle}}
\newcommand{\veN}[3]{{\left\langle#1,#2,#3\right\rangle}}
\newcommand{\ang}[2]{{(\widehat{\ve{#1},\ve{#2}})}}
\newcommand{\abs}[1]{{\left|{#1}\right|}}
\newcommand{\when}[2]{{\left.{#1}\right|_{#2}}}
\newcommand{\dist}[2]{{\left\|\ve{#1}-\ve{#2}\right\|}}
\newcommand{\norm}[1]{{\left\|#1\right\|}}
\newcommand{\emplin}{\vspace{1em}}

\begin{document}

\section*{常数项无穷级数 Constant term infinite series}
\subsection*{一般项 General terms}
一般的,如果给定一个数列 (sequence)
\[u_1,u_2,\cdots,u_n,\cdots\]
那么由这个数列构成的表达式
\[u_1+u_2+\cdots+u_n+\cdots\]
叫做(常数项)无穷级数,简称级数,记为$\sum_{i=1}^\infty u_i$,即
\[\sum_{i=1}^\infty u_i=u_1+u_2+\cdots+u_n+\cdots\]
其中第$n$项$u_n$叫做级数的一般项

\subsection*{部分和}

做(常数项)级数的前$n$项的和
\[s_n=\sum_{i=1}^n u_i\]
$s_n$称为级数的部分和

\subsection*{收敛与发散 Convergenc \& divergence}
如果级数$\sum_{i=1}^\infty u_i$的部分和数列$\{ s_n\}$有极限$s$,即
\[ \lim_{n\to\infty}s_n=s \]
那么称无穷级数$\sum_{i=1}^\infty u_i$收敛,这是极限$s$叫做这级数的和,并写成
\[ s= u_1+u_2+\cdots+u_n+\cdots\]
如果$\{ s_n\}$没有极限,那么称无穷级数$\sum_{i=1}^\infty u_i$发散

\emplin

当级数收敛时,其部分和$s_n$是级数的和$s$的近似值,它们之间的差值叫做级数的余项;用近似值$s_n$代替和$s$所产生的误差是这个余项的绝对值,即误差是$\abs{r_n}\quad(r_n=s-s_n)$

\subsection*{特殊的几个级数}
\begin{itemize}
  \item 等比级数
  \[ \sum_{i=0}^\infty aq^i=a+aq+\cdots+aq^{n-1}+\cdots\]
  \[s_n=\frac{a(1-q^n)}{1-q}=\frac{a}{1-q}-\frac{aq^n}{1-q} \]
  当$\abs{q}<1$时收敛于$\displaystyle \frac{a}{1-q}$,其余情况发散
  \item 等差级数
  \[ \sum_{i=1}^\infty i=1+2+\cdots+n+\cdots\]
  \[ s_n=\frac{n(n+1)}{2} \]
  此级数发散
  \item 调和级数
  \[ \sum_{i=1}^\infty \frac{1}{i}=1+\frac{1}{2}+\frac{1}{3}+\cdots+\frac{1}{n}+\cdots\]
  调和级数时发散的
  \item 收敛于1
  \[ \sum_{i=1}^\infty \frac{1}{i(i+1)}=\frac{1}{1\cdot2}+\frac{1}{2\cdot3}+\cdots+\frac{1}{n(n+1)}+cdot\]
  \[s_n=1-\frac{1}{n+1}\]
  此级数收敛于1
\end{itemize}

\subsection*{收敛级数的基本性质}
\begin{itemize}
  \item 如果级数$\displaystyle\sum_{i=1}^\infty u_i$收敛于和$s$,那么级数$\displaystyle\sum_{i=1}^infty ku_i$也收敛,其和为$ks$
  \item 如果级数$\displaystyle\sum_{i=1}^\infty u_i$与$\displaystyle\sum_{i=1}^\infty v_i$分别收敛于和$s$和$\sigma$,那么级数
  $\displaystyle\sum_{i=1}^\infty (u_i\pm v_i)$也收敛,其和为$s\pm\sigma$

  即,两个收敛级数可以逐项相加于逐项相减
  \item 在级数中去掉、加上或改变有限项,不会改变级数的收敛性
  \item 如果级数$\displaystyle\sum_{i=1}^\infty u_i$收敛,那么对于这个级数的项任一加括号后组成的级数
  \[(u_1+\cdots+u_{n_1})+(u_{n_1+1}+\cdots+u_{n_2})+\cdots+(u_{n_{k-1}+1}+\cdots+u_{n_k})+\cdots\]
  仍收敛,且其和不变
  \item 如果级数$\displaystyle\sum_{i=1}^\infty u_i$收敛,那么它的一般项$u_n$趋于零(必要条件),即
  \[\lim_{n\to\infty}u_n=0\]
\end{itemize}

\emplin

\section*{级数的审敛法}

\begin{itemize}
  \item 柯西审敛原理 (Cauchy Criterion)

  级数$\displaystyle\sum_{i=1}^\infty u_i$收敛的充分必要条件是:对于任一给定正数$\epsilon$,总存在正整数$N$使得当$n>N$时,对于任意的正整数$p$,都有
  \[\abs{u_{n+1}+u_{n+2}+\cdots+i_{n+p}}<\epsilon\]
  成立

  \item 正项级数$\displaystyle\sum_{i=1}^\infty u_i$收敛的充分必要条件时:它的部分和数列$\{ s_n \}$有界
  \item 比较审敛法

  设$\displaystyle\sum_{i=1}^\infty u_i$和$\displaystyle\sum_{i=1}^\infty v_i$都是正项级数,且$u_i\le v_i$,若级数$\displaystyle\sum_{i=1}^\infty v_i$
  收敛则$\displaystyle\sum_{i=1}^\infty u_i$收敛;若级数$\displaystyle\sum_{i=1}^\infty u_i$发散则$\displaystyle\sum_{i=1}^\infty v_i$发散

  \item 设$\displaystyle\sum_{i=1}^\infty u_i$和$\displaystyle\sum_{i=1}^\infty v_i$都是正项级数:
  若级数$\displaystyle\sum_{i=1}^\infty v_i$收敛,且存在正整数$N$,使得当$n\ge N$时有$u_n\le kv_n\quad(k>0)$,则$\displaystyle\sum_{i=1}^\infty u_i$收敛;
  若级数$\displaystyle\sum_{i=1}^\infty v_i$发散,且存在正整数$N$,使得当$n\ge N$时有$u_n\ge kv_n\quad(k>0)$,则$\displaystyle\sum_{i=1}^\infty u_i$发散

  \item 比较审敛法的极限形式

  设$\displaystyle\sum_{i=1}^\infty u_i$和$\displaystyle\sum_{i=1}^\infty v_i$都是正项级数:
  \begin{itemize}
    \item 如果$\displaystyle\lim_{n\to\infty}\frac{u_n}{v_n}=l\quad(0\le l<+\infty)$,且级数$\displaystyle\sum_{i=1}^\infty v_i$收敛,
    那么级数$\displaystyle\sum_{i=1}^\infty u_i$收敛
    \item 如果$\displaystyle\lim_{n\to\infty}\frac{u_n}{v_n}=l>0$或$\displaystyle\lim_{n\to\infty}\frac{u_n}{v_n}=+\infty$,
    且级数$\displaystyle\sum_{i=1}^\infty v_i$发散,那么级数$\displaystyle\sum_{i=1}^\infty u_i$发散
  \end{itemize}

  \item 比值审敛法,达朗贝尔 (d'Alembert)判别法

  设$\displaystyle\sum_{i=1}^\infty u_i$为正项级数,如果
  \[\lim_{n\to\infty}\frac{u_{n+1}}{u_n}=\rho\]
  那么当$\rho<1$时级数收敛,$\rho>1$时级数发散,$\rho=1$时可能收敛也可能发散

  \item 根值审敛法,柯西判别法

  设$\displaystyle\sum_{i=1}^\infty u_i$为正项级数,如果
  \[\lim_{n\to\infty}\sqrt[n]{u_n}=\rho\]
  那么当$\rho<1$时级数收敛,$\rho>1$时级数发散,$\rho=1$时级数可能收敛也可能发散

  \item 极限审理法

  设$\displaystyle\sum_{i=1}^\infty u_i$为正项级数,
  \begin{itemize}
    \item 如果$\lim_{n\to\infty}nu_n=l>0$,那么级数$\displaystyle\sum_{i=1}^\infty u_i$发散
    \item 如果$p>1$,而$\lim_{n\to\infty}n^pu_n=l\quad(0\le l<+\infty)$,那么级数$\displaystyle\sum_{i=1}^\infty u_i$收敛
  \end{itemize}

  \item 莱布尼茨定理

  如果交错级数$\displaystyle\sum_{i=1}^\infty (-1)^{n-1}u_i$满足条件:
  \begin{itemize}
    \item $u_n\ge u_{n+1}$
    \item $\displaystyle\lim_{n\to\infty}u_n=0$
  \end{itemize}
  那么级数收敛,且其和$s\le u_1$,其余项$r_n$的绝对值$\abs{r_n}\le u_{n+1}$

  \item 绝对收敛
  若级数$\displaystyle\sum_{i=1}^\infty u_i$的各项的绝对值组成的级数$\displaystyle\sum_{i=1}^\infty abs{u_i}$收敛,则称其绝对收敛;
  若$\displaystyle\sum_{i=1}^\infty u_i$收敛,而$\displaystyle\sum_{i=1}^\infty abs{u_i}$发散,则称其条件收敛;
  如果级数$\displaystyle\sum_{i=1}^\infty u_i$绝对收敛,那么级数$\displaystyle\sum_{i=1}^\infty u_i$必定收敛

  \item 绝对收敛级数经改变项的位置后构成的级数也收敛,且与原级数有相同的和(绝对收敛级数具有可交换性)

  \item 绝对收敛级数的乘法

  设级数$\displaystyle\sum_{i=1}^\infty u_i$与$\displaystyle\sum_{i=1}^\infty v_i$都绝对收敛,其和分别为$s$和$\sigma$,则它们的柯西乘积
  \[u_1v_1+(u_1v_2+u_2v_1)+\cdots+(u_1v_n+u_2v_{n-1}+\cdots+u_nv_1)+\cdots\]
  也是绝对收敛的,且其和为$s\sigma$
\end{itemize}

\section*{幂级数}
\subsection*{函数项级数 series of functions}

如果给定一个定义在区间$I$上的函数列
\[u_1(x),u_2(x),\cdots,u_n(x),\cdots\]
那么由这函数列构成的表达式
\[u_1(x)+u_2(x)+\cdots+u_n(x)+\cdots\]
称为定义在区间$I$上的(函数项)无穷级数,简称(函数项)级数

对于每一确定的值$x_0\in I$,函数项级数成为以常数项级数,这个级数可能收敛,也可能发散;对于是该级数收敛的点$x_0$称为函数项级数的收敛点,如果发散则称为发散点;
收敛点的全体称为它的收敛域,发散点的全体称为它的发散域

对于收敛域内的任意$x$,函数项级数成为一收敛的常数项级数,因而存在一确定的和;该和在收敛域上是$x$的一个函数$s(x)$,称为和函数,即
\[s(x)=u_1(x)+u_2(x)+\cdots+u_n(x)+\cdots\]

将函数项级数的前$n$项的部分和记作$s_n(x)$,则在收敛域上有
\[\lim_{n\to\infty}s_n(x)=s(x)\]
记$r_n(x)=s(x)-s_n(x)$,$r_n(x)$叫做函数项级数的余项,且有
\[\lim_{n\to\infty}r_n(x)=0\]

\subsection*{幂级数}
幂级数的形式是
\[\sum_{i=0}^\infty a_ix^i=a_0+a_1x+a_2x^2+\cdots+a_nx^n+\cdots\]
其中常数$a_0,a_1,a_2\cdots,a_n,\cdots$叫做幂级数的系数

\subsection*{幂级数的收敛性}
\begin{itemize}
  \item 阿贝尔 (Abel)定理

  如果级数$\displaystyle\sum_{i=0}^\infty a_ix^i$当$x=x_0\quad(x_0\ne0)$时收敛,那么适合不等式$\abs{x}<\abs{x_0}$的一切$x$使得这幂函数绝对收敛;
  反之,如果级数$\displaystyle\sum_{i=0}^\infty a_ix^i$当$x=x_0$时发散,那么适合不等式$\abs{x}>\abs{x_0}$的一切$x$使得这幂函数发散

  \item 如果幂级数$\displaystyle\sum_{i=0}^\infty a_ix^i$不是仅在$x=0$一点收敛,也不是在整个数轴上都收敛,那么必有一个确定的正数$R$存在,使得

  当$\abs{x}<R$时,幂级数绝对收敛;

  当$\abs{x}>R$时,幂级数发散;

  当$\abs{x}=R$时,收敛性不确定

  该常数$R$称为幂级数的收敛半径,开区间$(-R,R)$称为收敛区间,再由$x=\pm R$时的收敛性确定它的具体收敛域

  \item 如果
  \[ \lim_{n\to\infty}\abs{\frac{a_{n+1}}{a_n}}=\rho \]
  其中$a_n$、$a_{n+1}$是幂级数$\displaystyle\sum_{i=0}^\infty a_ix^i$的相邻的系数那么这幂级数的收敛半径
  \[R=\begin{cases}
  \frac{1}{\rho},&\rho\ne0\\
  +\infty,&\rho=0\\
  0,&\rho=+\infty
  \end{cases}\]
\end{itemize}

\subsection*{幂级数的运算}
设有两幂级数
\[\sum_{i=0}^\infty a_ix^i\]
与
\[\sum_{i=0}^\infty b_ix^i\]
分别再区间$(-R,R)$及$(-R',R')$内收敛,则有

\begin{itemize}
  \item 加法
  \[\sum_{i=0}^\infty a_ix^i+\sum_{i=0}^\infty b_ix^i=\sum_{i=0}^\infty (a_i+b_i)x^i\]
  其收敛域为$(-R,R)\cap(-R',R')$
  \item 减法
  \[\sum_{i=0}^\infty a_ix^i-\sum_{i=0}^\infty b_ix^i=\sum_{i=0}^\infty (a_i-b_i)x^i\]
  其收敛域为$(-R,R)\cap(-R',R')$
  \item 乘法
  \[\left(\sum_{i=0}^\infty a_ix^i\right)\left(\sum_{i=0}^\infty b_ix^i\right)=\sum_{i=0}^\infty\left(\sum_{j=0}^i a_j+b_{i-j}\right)x^i\]
  其收敛域为$(-R,R)\cap(-R',R')$
  \item 除法
  \[\frac{\sum_{i=0}^\infty a_ix^i}{\sum_{i=0}^\infty b_ix^i}=\sum_{i=0}^\infty c_ix^i\]
  其中$b_0\ne0$,则有
  \[a_n=\sum_{i=0}^nb_ic_{n-i}\]
  其收敛域可能远小于$(-R,R)\cap(-R',R')$
\end{itemize}

\subsection*{性质}
\begin{itemize}
  \item 幂级数$\displaystyle\sum_{i=0}^\infty a_ix^i$的和函数$s(x)$在其收敛域上连续
  \item 幂级数$\displaystyle\sum_{i=0}^\infty a_ix^i$的和函数$s(x)$在其收敛域$I$上可积,并有逐项积分公式
  \[\int_0^xs(t)\dif{t}=\int_0^x\left[\sum_{i=0}^\infty a_it^i\right]\dif{t}=\sum_{i=0}^\infty \frac{a_i}{i+1}x^{i+1}\quad(x\in I)\]
  逐项积分后所得到的幂级数和原级数有相同的收敛半径
  \item 幂级数$\displaystyle\sum_{i=0}^\infty a_ix^i$的和函数$s(x)$在其收敛域$I$上可导,且有逐项求导公式
  \[s'(x)=\left(\sum_{i=0}^\infty a_ix^i\right)'=\sum_{i=0}^\infty ia_ix^{i-1}\quad(x\in I)\]
  逐项求导后所得到的幂级数和原级数有相同的收敛半径
  \item 幂级数$\displaystyle\sum_{i=0}^\infty a_ix^i$的和函数$s(x)$在其收敛域$I$上有任意阶导数
\end{itemize}

\subsection*{函数的幂级数展开}
设函数$f(x)$在点$x_0$的某一邻域$U(x_0)$内具有各阶导数,则$f(x)$在该邻域内能展开称泰勒级数的充分必要条件是在该邻域内$f(x)$的泰勒公式中的余项$R_n(x)$
当$n\to\infty$时的极限为零,即
\[ \lim_{n\to\infty}R_n(x)=0,\quad x\in U(x_0) \]

将函数$f(x)$在原点的泰勒级数称为麦克劳林级数

\subsection*{常用展开}
\begin{itemize}
  \item $\displaystyle e^x=1+x+\frac{x^2}{2!}+\cdots+\frac{x^n}{n!}+\cdots=\sum_{i=0}^\infty \frac{x^i}{i!}\quad(-\infty<x<+\infty)$
  \item $\displaystyle a^x=e^{x\ln a}=\sum_{i=0}^\infty\frac{(\ln a)^i}{i!}x^i \quad(-\infty<x<+\infty)$
  \item $\displaystyle \ln(1+x)=\sum_{i=0}^\infty\frac{(-1)^i}{i+1}x^{i+1} \quad(-1<x\le+1)$
  \item $\displaystyle \cos x=1-\frac{x^2}{2!}+\frac{x^4}{4!}+\cdots+(-1)^n\frac{x^{2n}}{(2n)!}+\cdots \sum_{i=0}^\infty\frac{(-1)^i}{(2n)!}x^{2n}\quad(-\infty<x<+\infty)$
  \item $\displaystyle \sin x=x-\frac{x^3}{3!}+\frac{x^5}{5!}+\cdots+(-1)^n\frac{x^{2n+1}}{(2n+1)!}+\cdots \sum_{i=0}^\infty\frac{(-1)^i}{(2n+1)!}x^{2n+1}\quad(-\infty<x<+\infty)$
  \item $\displaystyle \frac{1}{1+x}=\sum_{i=0}^\infty(-1)^ix^i \quad(-1<x<+1)$
  \item $\displaystyle \frac{1}{1+x^2}=\sum_{i=0}^\infty(-1)^ix^{2i} \quad(-1<x<+1)$
  \item $\displaystyle \arctan x = \sum_{i=0}^\infty\frac{(-1)^i}{2i+1}x^{2i+1}\quad(-1\le x\le+1)$
  \item $\displaystyle (1+x)^m = 1+mx+\frac{m(m-1)}{2!}x^2+\cdots+\frac{m(m-1)\cdots(m-n+1)}{n!}x^n+\cdots=\sum_{i=0}^\infty C_m^ix^i\quad(-1< x<+1)$

  (二项式展开)
\end{itemize}

\subsection*{用幂级数解微分方程}
如果$y''+P(x)y'+Q(x)y=0$中的$P(x)$与$Q(x)$可在$(-R,R)$内展开为$x$的幂级数,那么在$(-R,R)$内必有形如
\[y=\sum_{i=0}^\infty a_ix^i\]
的解

\section*{一致收敛性}
\subsection*{定义}
设有函数项级数$\displaystyle\sum_{n=1}^\infty u_n(x)$,如果对于任意给定的正数$\epsilon$,都存在着一个只依赖于$\epsilon$的正整数$N$,
使得当$n>N$时,对于区间$I$上的一切$x$,都有不等式
\[ \abs{r_n(x)}=\abs{s(x)-s_n(x)}<\epsilon \]
成立,那么称函数项级数$\displaystyle\sum_{n=1}^\infty u_n(x)$在区间$I$上一致收敛于和$s(x)$,也称函数序列$\{ s_n(x) \}$在区间$I$上一致收敛于$s(x)$

\subsection*{魏尔斯特拉斯(Weierstrass)判别法}
如果函数项级数$\displaystyle\sum_{n=1}^\infty u_n(x)$在区间$I$上满足条件:
\begin{itemize}
  \item $\abs{u_n(x)}\le a_n$
  \item 正项级数$\displaystyle\sum_{n=1}^\infty a_n(x)$收敛
\end{itemize}
那么函数项级数$\displaystyle\sum_{n=1}^\infty u_n(x)$在区间$I$上一致收敛

\subsection*{性质}
\begin{itemize}
  \item 如果级数$\displaystyle\sum_{n=1}^\infty u_n(x)$的各项$u_n(x)$在区间$[a,b]$上都连续,且$\displaystyle\sum_{n=1}^\infty u_n(x)$在区间
  $[a,b]$上一致收敛于$s(x)$,那么$s(x)$在$[a,b]$上也连续
  \item 如果级数$\displaystyle\sum_{n=1}^\infty u_n(x)$的各项$u_n(x)$在区间$[a,b]$上都连续,且$\displaystyle\sum_{n=1}^\infty u_n(x)$在区间
  $[a,b]$上一致收敛于$s(x)$,那么$s(x)$在$[a,b]$上可逐项积分,即
  \[\int_{x_0}^xs(t)\dif{t}=\sum_{i=1}^\infty\int_{x_0}^xu_i(t)\dif{t}\]
  其中$a\le x_0<x\le b$,并且上式右侧的级数在$[a,b]$上也一致收敛
  \item 如果级数$\displaystyle\sum_{n=1}^\infty u_n(x)$的各项$u_n(x)$在区间$[a,b]$上收敛于$s(x)$,它的各项$u_n(x)$在区间$[a,b]$上都具有连续导数
  $u'_n(x)$,并且级数$\sum_{n=1}^\infty u'_n(x)$在$[a,b]$上一致收敛,那么级数$\sum_{n=1}^\infty u_n(x)$在$[a,b]$上也一致收敛,且可逐项求导,即
  \[s'(x)=\sum_{n=1}^\infty u'_n(x)\]
  \item 如果幂级数$\displaystyle\sum_{i=0}^\infty a_ix^i$的收敛半径为$R>0$,那么此级数在$(-R,R)$内的任一闭区间$[a,b]$上一致收敛
  \item 如果幂级数$\displaystyle\sum_{i=0}^\infty a_ix^i$的收敛半径为$R>0$,那么其和函数$s(x)$在$(-R,R)$内可导,且有逐项求导公式
  \[s'(x)=\left(\sum_{n=0}^\infty a_nx^n\right)'=\sum_{n=1}^\infty na_nx^{n-1}\]
  逐项求导后所得到的幂级数与原级数有相同的收敛半径
\end{itemize}

\end{document}

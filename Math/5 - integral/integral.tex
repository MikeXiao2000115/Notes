\documentclass[UTF8]{ctexart}

\usepackage{geometry}

\usepackage{amsmath}
\usepackage{amssymb}

\usepackage{fancyhdr}
\usepackage{graphicx}

\special{papersize={18.1cm,25.7cm}}
\geometry{left=1.5cm,right=0.5cm,top=2cm,bottom=1cm}
\pagestyle{empty}

\newcommand{\D}{\text{d}\;\!}
\newcommand{\dif}[1]{\text{d}\;\!#1}
\newcommand{\dev}[1]{\frac{\text{d}}{\dif{#1}}\;\!}

\begin{document}

\section*{定积分 integral}

\bigskip

设函数$f(x)$在$[a,b]$上有界,在$[a,b]$中任意插入若干个分点$a=x_0<x_1<x_2<\cdots<x_n=b$把区间$[a,b]$分成$n$个小区间$[x_0,x_1],[x_1,x_2],\cdots,[x_{n-1},x_n]$,
各个小区间的长度依次为$\Delta x_1 = x_1-x_0,\Delta x_2 = x_2-x_1,\cdots,\Delta x_n = x_n-x_{n-1}$在每个小区间$[x_{i-1},x_i]$上任取一点$\xi_i\:(x_{i-1}\le\xi_i\le x_i)$,
作函数值$f(\xi_i)$与小区间长度$\Delta x_i$的乘积$f(\xi_i)\Delta x_i$,并作出和
\[ S=\sum^n_{i=1}f(\xi_i)\Delta x_i \]
记$\lambda=\max\{\Delta x_1,\Delta x_2,\cdots,\Delta x_n\}$,如果当$\lambda\to0$时,这和的极限追踪存在,且与闭区间$[a,b]$的分法及点$\xi_i$的取法无关,
那么称这个极限$I$为函数$f(x)$在区间$[a,b]$上的定积分,
\[\int^b_af(x)\dif{x}=I=\lim_{\lambda\to0}\sum^n_{i=1}f(\xi_i)\Delta x_i\]
其中$f(x)$叫做被积函数的,$f(x)\dif{x}$叫做被积表达式,$x$叫做积分变量,$a$叫做积分下限,$b$叫做积分上限,$[a,b]$叫做积分区间


\bigskip

\textbf{如果$f(x)$在区间$[a,b]$上连续,则$f(x)$在$[a,b]$上可积}
\textbf{如果$f(x)$在区间$[a,b]$上有界,且只有有限个间断点,则$f(x)$在$[a,b]$上可积}

\bigskip
\bigskip

\section*{定积分的性质}

\bigskip

设$\alpha$与$\beta$均为常数,则
\[ \int^b_a[\alpha f(x)+\beta g(x)]\dif{x}=\alpha\int^b_af(x)\dif{x}+\beta\int^b_ag(x)\dif{x} \]

设$a<c<b$,则
\[ \int^b_af(x)\dif{x}=\int^c_af(x)\dif{x}+\int^b_cf(x)\dif{x} \]

如果在区间$[a,b]$上$f(x)\equiv1$那么
\[ \int^b_af(x)\dif{x}=\int^b_a1\dif{x}=\int^b_a\dif{x}=b-a\]

如果在区间$[a,b]$上$f(x)\ge0$,那么
\[ \int^b_af(x)\dif{x}\ge0\quad(a\le b) \]

如果在区间$[a,b]$上$f(x)\ge g(x)$,那么
\[ \int^b_af(x)\dif{x}\ge\int^b_ag(x)\dif{x}\quad(a\le b) \]

\[ \left| \int^b_af(x)\dif{x} \right|\le\int^b_a|f(x)|\dif{x}\quad(a\le b) \]

设$M$及$m$分别时函数$f(x)$在区间$[a,b]$上的最大值及最小值,则
\[ m(b-a)\le\int^b_af(x)\dif{x}\le M(b-a) \quad(a\le b)\]

如果函数$f(x)$在积分区间$[a,b]$上连续,那么在$[a,b]$上至少存在一点$\xi$,使
\[ \int^b_af(x)\dif{x}=f(\xi)(b-a)\quad(a\le\xi\le b) \]
成立(积分中值公式)

如果函数$f(x)$在区间$[a,b]$上连续,那么积分上限的函数
\[ \Phi(x)=\int^x_af(t)\dif{t} \]
在$[a,b]$上可导,并且它的导数
\[ \Phi'(x)=\dev{x}\int^x_af(t)\dif{t}=f(x)\quad(a\le x\le b) \]

如果函数$f(x)$在区间$[a,b]$上连续,那么函数
\[ \Phi(x)=\int^x_af(t)\dif{t} \]
就是$f(x)$在$[a,b]$上的一个原函数

\textbf{牛顿-莱布尼茨公式 Newton-Leibniz formula}
如果函数$F(x)$使连续函数$f(x)$在区间$[a,b]$上的一个原函数,那么
\[ \int_a^bf(x)\dif{x}=F(b)-F(a) \]

\bigskip

假设函数$f(x)$在区间$[a,b]$上连续,函数$x=\phi(t)$满足条件
\begin{itemize}
  \item $\phi(\alpha)=a,\phi(\beta)=b$
  \item $\phi(t)$在$[\alpha,\beta]$(或$[\beta,\alpha]$)上具有连续导数,且其值域$R_\phi=[a,b]$
\end{itemize}
则有
\[ \int_a^bf(x)\dif{x}=\int_\alpha^\beta f[\phi(t)]\phi'(t)\dif{t} \]

\bigskip

\[ \int_a^bu\dif{v}=[uv]_a^b-\int_a^bv\dif{uz} \]

\bigskip
\bigskip

\section*{反常积分 Improper integral}

\bigskip

\begin{itemize}
  \item 设函数$f(x)$在区间$[a,+\infty)$上连续,如果极限$\displaystyle\lim_{t\to+\infty}\int_a^tf(x)\dif{x}$存在,
  那么称反常积分$\displaystyle\int_a^{+\infty}f(x)\dif{x}$收敛,并称此极限为反常积分的值;否则称其发散
  \item 设函数$f(x)$在区间$(-\infty,b]$上连续,如果极限$\displaystyle\lim_{t\to-\infty}\int_t^bf(x)\dif{x}$存在,
  那么称反常积分$\displaystyle\int_{\infty}^bf(x)\dif{x}$收敛,并称此极限为反常积分的值;否则称其发散
  \item 设函数$f(x)$在区间$(-\infty,\infty]$上连续,如果反常积分$\displaystyle\int_0^{+\infty}f(x)\dif{x}$ 与$\displaystyle\int_{\infty}^0f(x)\dif{x}$
  均收敛,那么称反常积分$\displaystyle\int_{-\infty}^{+\infty}f(x)\dif{x}$收敛,其值为前两者的和;否则称其发散
\end{itemize}
以上反常积分统称为无穷限的反常积分

\bigskip

\begin{itemize}
  \item 设函数$f(x)$在区间$(a,b]$上连续,点$a$为$f(x)$的瑕点,如果极限$\displaystyle\lim_{t\to a^+}\int_t^bf(x)\dif{x}$存在,
  那么称反常积分$\int_a^bf(x)\dif{x}$收敛,并称此极限为反常积分的值;否则称其发散
  \item 设函数$f(x)$在区间$[a,b)$上连续,点$b$为$f(x)$的瑕点,如果极限$\displaystyle\lim_{t\to b^-}\int_a^tf(x)\dif{x}$存在,
  那么称反常积分$\int_a^bf(x)\dif{x}$收敛,并称此极限为反常积分的值;否则称其发散
  \item 设函数$f(x)$在区间$[a,c)$及$(c,b]$上连续,点$c$为$f(x)$的瑕点,如果反常积分$\int_a^cf(x)\dif{x}$与$\int_c^bf(x)\dif{x}$均收敛,
  那么称反常积分$\int_a^bf(x)\dif{x}$收敛,其值为前两者之和;否则称其发散
\end{itemize}

\bigskip
\bigskip

\section*{反常积分审敛法\& $\Gamma$函数}

\bigskip

设函数$f(x)$在区间$[a,+\infty)$上连续,且$f(x)\ge0$,若函数
\[F(x)=\int_a^xf(t)\dif{t}\]
在$[a,+\infty)$上有上界,则反常积分$\int_a^{+\infty}f(x)\dif{x}$收敛

设函数$f(x)$,$g(x)$在区间$[a,+\infty)$上连续;
如果$0\le f(x)\le g(x)\quad(a\le x<+\infty)$,
并且$\int_a^{+\infty}g(x)\dif{x}$收敛,那么$\int_a^{+\infty}f(x)\dif{x}$也收敛;
$0\le g(x)\le f(x)\quad(a\le x<+\infty)$,
并且$\int_a^{+\infty}g(x)\dif{x}$发散,那么$\int_a^{+\infty}f(x)\dif{x}$也发散

设函数$f(x)$在区间$[a,+\infty)\quad(a>0)$上连续,且$f(x)\ge0$;
如果存在常数$M>0$及$p>1$,使得$f(x)\le\frac{M}{x^p}\quad(a\le x<+\infty)$,
那么反常积分$\int_a^{+\infty}f(x)\dif{x}$收敛;
如果存在常数$N>0$,使得$f(x)\ge\frac{N}{x}\quad(a\le x<+\infty)$,
那么反常积分$\int_a^{+\infty}f(x)\dif{x}$发散

设函数$f(x)$在区间$[a,+\infty)$上连续,且$f(x)\ge0$;
如果存在常数$p>1$,使得$\lim_{x\to+\infty}x^pf(x)=c<+\infty$,
那么反常积分$\int_a^{+\infty}f(x)\dif{x}$收敛;
如果存在常数$N>0$,使得$\lim_{x\to+\infty}xf(x)=d>0$,
那么反常积分$\int_a^{+\infty}f(x)\dif{x}$发散

设函数$f(x)$在区间$[a,+\infty)$上连续,如果反常积分
\[ \int_a^{+\infty}|f(x)|\dif{x} \]
收敛(即,绝对收敛),那么反差积分
\[ \int_a^{+\infty}f(x)\dif{x} \]
也收敛

\bigskip

设函数$f(x)$在区间$(a,b]$上连续,且$f(x)\ge0$,$x=a$为$f(x)$的瑕点,
如果存在常数$M>0$及$q<1$,使得
\[ f(x)\le\frac{M}{(x-a)^q} \quad(a<x\le b)\]
那么反常积分$\int_a^bf(x)\dif{x}$收敛;
如果存在常数$N>0$,使得
\[ f(x)\ge\frac{N}{x-a}\quad(a<x\le ) \]
那么反常积分$\int_a^bf(x)\dif{x}$发散

设函数$f(x)$在区间$(a,b]$上连续,且$f(x)\ge0$,$x=a$为$f(x)$的瑕点,
如果存在常数$0<q<1$,使得
\[ \lim_{x\to a^+}(x-a)^qf(x)\]
存在,那么反常积分$\int_a^bf(x)\dif{x}$收敛;
如果存在常数$N>0$,使得
\[ \lim_{x\to a^+}(x-a)f(x)>0\]
那么反常积分$\int_a^bf(x)\dif{x}$发散

\bigskip

\textbf{$\Gamma$函数}
\[\Gamma(s)=\int_0^{+\infty}e^{-x}x^{s-1}\quad(s>0)\]

\bigskip

递推公式
\[\Gamma(s+1)=s\Gamma(s)\quad(s>0)\]
\[\Gamma(n+1)=n!\quad n\in\mathbb{N}^+\]

当$s\to0^+$时,$\Gamma(s)\to+\infty$

\[\Gamma(s)\Gamma(1-s)=\frac{\pi}{\sin \pi s}\quad(0<s<1)\]
\[\Gamma(\frac{1}{2})=\sqrt{\pi}\]
\[\int_0^{+\infty}e^{-u^2}\dif{u}=\frac{\sqrt{\pi}}{2}\]
\end{document}

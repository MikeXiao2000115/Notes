\documentclass[UTF8]{ctexart}

\usepackage{geometry}

\usepackage{amsmath}
\usepackage{amssymb}
\usepackage{esint}
\usepackage{yhmath}
\usepackage{bm}

\usepackage{fancyhdr}
\usepackage{graphicx}

\special{papersize={18.1cm,25.7cm}}
\geometry{left=1.5cm,right=0.5cm,top=2cm,bottom=1cm}
\pagestyle{empty}

\newcommand{\D}{{\text{d}\;\!}}
\newcommand{\cross}{\times}
\newcommand{\dif}[1]{{\mathrm{d}\;\!#1}}
\newcommand{\dev}[1]{{\frac{\text{d}}{\dif{#1}}\;\!}}
\newcommand{\ve}[1]{{\bm{#1}}}
\newcommand{\ven}[2]{{\left\langle#1,#2\right\rangle}}
\newcommand{\veN}[3]{{\left\langle#1,#2,#3\right\rangle}}
\newcommand{\ang}[2]{{(\widehat{\ve{#1},\ve{#2}})}}
\newcommand{\abs}[1]{{\left|{#1}\right|}}
\newcommand{\when}[2]{{\left.{#1}\right|_{#2}}}
\newcommand{\dist}[2]{{\left\|\ve{#1}-\ve{#2}\right\|}}
\newcommand{\norm}[1]{{\left\|#1\right\|}}
\newcommand{\emplin}{\vspace{1em}}

\begin{document}

\section*{曲线积分 curve intergral}
\subsection*{定义}
设$L$为$xOy$面内的一条光滑曲线弧,函数$f(x,y)$在$L$上有界,在$L$上任意插入一点列$M_1,M_2,\cdots,M_{n-1}$
把$L$分成$n$个小段;设第$i$个小段的长度为$\Delta s_i$,又$(\xi_i,\eta_i)$为第$i$个小段上任意取定的一点,
作积$f(\xi_i,\eta_i)\Delta s_i$,并作和$\sum^n_{i=1}f(\xi_i,\eta_i)\Delta s_i$,如果当各小弧段的长度的最大值
$\lambda\to0$时,这和的极限总存在,且与曲线弧$L$的分法无关,那么称此极限为函数$f(x,y)$在曲线弧$L$上对弧长的曲线积分
或第一类曲线积分,记作$\int_Lf(x,y)\dif{s}$,即
\[\int_Lf(x,y)\dif{s}=\lim_{\lambda\to0}\sum^n_{i=1}f(\xi_i,\eta_i)\Delta s_i\]
其中$f(x,y)$称为被积函数,$L$叫做积分弧段

\emplin

类似的情况可以扩展至三维
\[\int_Lf(x,y,z)\dif{s}=\lim_{\lambda\to0}\sum^n_{i=1}f(\xi_i,\eta_i,\zeta_i)\Delta s_i\]

甚至是任意维
\[\int_Lf(P)\dif{s}=\lim_{\lambda\to0}\sum^n_{i=1}f(P_i)\Delta s_i\]

\emplin

\subsection*{性质}
\begin{itemize}
  \item 若曲线$L$为一闭合曲线,则将对$f(P)$在$L$上的曲线积分记为$\displaystyle \oint_Lf(P)\dif{s}$
  \item 若$\alpha$与$\beta$均为常数,则
  \[\int_L[\alpha f(P)+\beta g(P)]\dif{s}=\alpha\int_Lf(P)\dif{s}+\beta\int_Lg(P)\dif{s}\]
  \item 若曲线$L$可以被分为有限多条曲线$L_1,L_2,\cdots,L_n$,则
  \[ \int_Lf(P)\dif{s}=\sum^n_{i=1}\int_{L_i}f(P)\dif{s} \]
  \item 设在$L$上$f(P)\le g(P)$,则
  \[\int_Lf(P)\dif{s}\le\int_Lg(P)\dif{s}\]
  特别有
  \[\abs{\int_Lf(P)\dif{s}}\le\int_L\abs{f(P)}\dif{s}\]
\end{itemize}

\emplin

\subsection*{曲线积分的计算}
若曲线$L$可由参数方程$\ve{r}(t)\;t\in[\alpha,\beta]$表述(其中$\alpha\le\beta$),则对于函数$f(P)$在曲线$L$上的曲线积分可改写为
\[\int_Lf(P)\dif{s}=\int_\alpha^\beta f(\ve{r}(t))\norm{\ve{r}'(t)}\dif{t}\]

\section*{对坐标的曲线积分}
\subsection*{定义}
设$L$为$n$维空间内点$A$到点$B$的一条有向光滑曲线弧,函数$F_i(P)$在$L$上有界,在$L$上沿$L$的方向任意插入一点列
$M_1({x_1}_1,{x_2}_1,\cdots,{x_n}_1),M_2({x_1}_2,{x_2}_2,\cdots,{x_n}_2),\cdots,M_{m-1}({x_1}_{m-1},{x_2}_{m-1},\cdots,{x_n}_{m-1})$
把$L$分为$m$个有向小弧段
\[ \wideparen{M_{j-1}M_j}\quad(j=1,2,\cdots,n;M_0=A,M_n=B)\]
设$\Delta {x_i}_j={x_i}_j-{x_i}_{j-1}$,点$P_j$为$\wideparen{M_{j-1}M_j}$上任意取定的点,做乘积
$F_i(P_j)\Delta {x_i}_j$,并做和$\sum_{j=1}^mF_i(P_j)\Delta {x_i}_j$,如果当各小弧段长度的最大值
$\lambda\to0$是,这和的极限总存在,且与曲线弧$L$的分法及点$P_j$的取法无关,那么称此极限为函数$F_i(P)$
在有向曲线弧$L$上对坐标$x_i$的曲线积分或第二类曲线积分,记作$\int_LF_i(P)\dif{x_i}$,即
\[\int_LF_i(P)\dif{x_i}=\lim_{\lambda\to0}F_i(P_j)\Delta {x_i}_j\]
其中$F_i(p)$叫做被积函数,$L$叫做积分弧段

\emplin

若$\ve{F}(P)=\left\langle F_1(P),F_2(P),\cdots,F_n(P) \right\rangle$,则有
\[\sum^n_{i=1}\left(\int_LF_i(P)\dif{x_i}\right)=\int_L\left(\sum^n_{i=1}F_i(P)\dif{x_i}\right)=\int_L\ve{F}\cdot\dif{\ve{r}}\]
其中$\dif{\ve{r}}$称为有向曲线元

\emplin

\subsection*{性质}
\begin{itemize}
  \item 若$\alpha$与$\beta$均为常数,则
  \[\int_L[\alpha\ve{F}(P)+\beta\ve{G}(P)]\cdot\dif{\ve{r}}=\alpha\int_L\ve{F}(P)\cdot\dif{\ve{r}}+\beta\int_L\ve{G}(P)\cdot\dif{\ve{r}}\]
  \item 若有向曲线$L$可以被分为有限多条曲线$L_1,L_2,\cdots,L_n$,则
  \[ \int_L\ve{F}(P)\cdot\dif{\ve{r}}=\sum^n_{i=1}\int_{L_i}\ve{F}(P)\cdot\dif{\ve{r}} \]
  \item 设$L$是有向光滑曲线弧,$L^-$是$L$的反向曲线弧,则
  \[\int_{L^-}\ve{F}\cdot\dif{\ve{r}}=-\int_L\ve{F}\cdot\dif{\ve{r}}\]
\end{itemize}

\emplin

\subsection*{计算}
若曲线$L$可由参数方程$\ve{r}(t)\;t\in[\alpha,\beta]$表述(其中$\ve{r}(\alpha)$位于起点$A$,$\ve{r}(\beta)$位于起点$B$),
则
\[\int_L\ve{F}(P)\cdot\dif{\ve{r}}=\int_\alpha^\beta \ve{F}(\ve{r}(t))\cdot\ve{r}'(t)\dif{t}\]

\subsection*{两类曲线积分之间的联系}
\[\int_L\ve{F}\cdot\dif{\ve{r}}=\int_LF_T\dif{s}\]
其中$F_T$为向量$\ve{F}$在曲线的切向量$\ve{T}$上的投影

\emplin

\section*{格林公式及其应用}
\subsection*{格林公式 Green's theorem}

设闭区域$D$由分段光滑的曲线$\partial D$围成,若函数$P(x,y)$与$Q(x,y)$在$D$上具有一阶连续偏导数,则有
\[\iint_D\left(\frac{\partial Q}{\partial x}-\frac{\partial P}{\partial y}\right)\dif{x}\dif{y}
  =\oint_{\partial D}P\dif{x}+Q\dif{y}\]
其中$\partial D$为$D$的正向边界(即,$\partial D$的左边均在$D$内,右边均在$D$外)

\subsection*{路径无关}
若对于函数$\ve{F}(P)$的曲线积分在$G$内与其积分路径无关,仅与其起点与终点相关,
则称该积分在$G$内与路径无关,否则称其与路径有关

\emplin

设区域$G$是一连通区域,若函数$P(x,y)$与$Q(x,y)$在$G$内具有一阶连续偏导,则曲线积分
$\int_LP\dif{x}+Q\dif{y}$在$G$内与路径无关的充分必要条件为
\[\frac{\partial P}{\partial y}=\frac{\partial Q}{\partial x}\]
在$G$内恒成立

\emplin
若对于函数$\ve{F}(P)$的曲线积分在$G$内与路径无关,则沿$G$内任意闭合曲线的曲线积分$\oint_LP\dif{x}+Q\dif{y}$均为$0$

\emplin

\subsection*{二元函数的全微分求积}
设区域$G$是一个单连通域,若函数$P(x,y)$与$Q(x,y)$在$G$内具有一阶连续偏导数,
则$P(x,y)\dif{x}+Q(x,y)\dif{y}$在$G$内为某一函数$u(x,y)$的全微分的充分必要条件是
\[\frac{\partial P}{\partial y}=\frac{\partial Q}{\partial x}\]
在$G$内恒成立

\emplin

设区域$G$是一连通区域,若函数$P(x,y)$与$Q(x,y)$在$G$内具有一阶连续偏导,则曲线积分
$\int_LP\dif{x}+Q\dif{y}$在$G$内与路径无关的充分必要条件为:在$G$内存在函数$u(x,y)$,
使$\dif{u}=P\dif{x}+Q\dif{y}$

\subsection*{曲线积分的基本定理}
若曲线积分$\int_L\ve{F}\cdot\dif{\ve{r}}$在区域$G$内与积分路径无关,则称向量场$\ve{F}$为保守场(conservative field)

\emplin

若向量场$\ve{F}(P)$在$G$内为一保守场,则一定存在一数量函数$f(P)$,使得$\ve{F}=\nabla f$,
而曲线积分$\int_L\ve{F}\cdot\dif{r}$在$G$内与路径无关,且
\[\int_L\ve{F}\cdot\dif{r}=f(B)-f(A)\]
其中$L$是位于$G$内起点为$A$,终点为$B$的任意分段光滑曲线

\emplin

\textbf{向量场$\ve{F}$为保守场的充分必要条件是$\nabla\cross\ve{F}=0$,即$\ve{F}$的旋度为$0$}

\section*{对面积的曲面积分}
\subsection*{定义}
设曲面$\Sigma$是光滑的,函数$f(x,y,z)$在$\Sigma$上有界,把$\Sigma$任意分成$n$小块$\Delta S_i$
($\Delta S_i$同时表示第$i$小块曲面的面积),设$(\xi_i,\eta_i,\zeta_i)$是$\Delta S_i$上任意取定的一点,
做乘积$f(\xi_i,\eta_i,\zeta_i)\Delta S_i$,并作和$\sum^n_{i=1}f(\xi_i,\eta_i,\zeta_i)\Delta S_i$,
如果当各小块曲面的直径的最大值$\lambda\to0$时,这和的极限总存在,且与曲面$\Sigma$的分法及点$(\xi_i,\eta_i,\zeta_i)$
的取法无关,那么称此极限为函数$f(x,y,z)$在曲面$\Sigma$上对面积的曲面积分或第一类曲面积分,记作$\iint_\Sigma f(x,y,z)\dif{S}$,即
\[\iint_\Sigma f(x,y,z)\dif{S}=\lim_{\lambda\to0}\sum^n_{i=1}f(\xi_i,\eta_i,\zeta_i)\Delta S_i\]
其中$f(x,y,z)$叫做被积函数,$\Sigma$叫做积分曲面

\subsection*{性质}
\begin{itemize}
  \item 若$\alpha$与$\beta$均为常数,则
  \[\iint_\Sigma[\alpha f(P)+\beta g(P)]\dif{S}=\alpha\iint_\Sigma f(P)\dif{S}+\beta\iint_\Sigma g(P)\dif{S}\]
  \item 若曲面$\Sigma$可以被分为有限多个曲面$\Sigma_1,\Sigma_2,\cdots,\Sigma_n$,则
  \[ \iint_\Sigma f(P)\dif{S}=\sum^n_{i=1}\iint_{\Sigma_i}f(P)\dif{S} \]
  \item 设在$\Sigma$上$f(P)\le g(P)$,则
  \[\iint_\Sigma f(P)\dif{S}\le\iint_\Sigma g(P)\dif{S}\]
  特别有
  \[\abs{\iint_\Sigma f(P)\dif{S}}\le\iint_\Sigma \abs{f(P)}\dif{S}\]
\end{itemize}

\subsection*{计算}
若曲面$\Sigma$可由参数方程$G(u,v)=\left( x(u,v),y(u,v),z(u,v) \right)$表示,
则函数$f(x,y,z)$对于曲面$\Sigma$的曲面积分可表示为
\[\iint_\Sigma f(x,y,z)\dif{S}=\iint_D f(x(u,v),y(u,v),z(u,v))\norm{\frac{\partial \ve{G}}{\partial u}\cross\frac{\partial \ve{G}}{\partial v}}\dif{u}\dif{v}\]


\section*{对坐标的曲面积分}
\subsection*{定义}
设$\Sigma$为光滑的有向的,向量函数$\ve{F}(x,y,z)$在$z$轴方向上的分量函数${F}_z(x,y,z)$在$\Sigma$上有界,把$\Sigma$任意分成$n$小块$\Delta S_i$
($\Delta S_i$同时表示第$i$小块曲面的面积),$\Delta S_i$在$xOy$面上的投影为$(\Delta S_i)_{xy}$,
设$(\xi_i,\eta_i,\zeta_i)$是$\Delta S_i$上任意取定的一点,
做乘积${F}_z(\xi_i,\eta_i,\zeta_i)(\Delta S_i)_{xy}$,
并作和$\sum^n_{i=1}{F}_z(\xi_i,\eta_i,\zeta_i)(\Delta S_i)_{xy}$,
如果当各小块曲面的直径的最大值$\lambda\to0$时,这和的极限总存在,且与曲面$\Sigma$的分法及点$(\xi_i,\eta_i,\zeta_i)$
的取法无关,那么称此极限为函数${F}_z(x,y,z)$在有向曲面$\Sigma$上对坐标$x$、$y$的曲面积分或第二类曲面积分,记作$\iint_\Sigma {F}_z(x,y,z)\dif{x}\dif{y}$,即
\[\iint_\Sigma {F}_z(x,y,z)\dif{x}\dif{y}=\lim_{\lambda\to0}\sum^n_{i=1}{F}_z(\xi_i,\eta_i,\zeta_i)(\Delta S_i)_{xy}\]
其中${F}_z(x,y,z)$叫做被积函数,$\Sigma$叫做积分曲面

\emplin

类似的可定义对坐标$y$、$z$的曲面积分以及对坐标$x$、$z$的曲面积分
\[\iint_\Sigma {F}_x(x,y,z)\dif{y}\dif{z}=\lim_{\lambda\to0}\sum^n_{i=1}{F}_x(\xi_i,\eta_i,\zeta_i)(\Delta S_i)_{yz}\]
\[\iint_\Sigma {F}_y(x,y,z)\dif{x}\dif{z}=\lim_{\lambda\to0}\sum^n_{i=1}{F}_y(\xi_i,\eta_i,\zeta_i)(\Delta S_i)_{xz}\]

\emplin

应用中常出现的形式为求向量场$\ve{F}(x,y,z)$在曲面有向$\Sigma$上的通量$\Phi$,为方便记为
\begin{align*}
\Phi(\Sigma)&=\iint_\Sigma {F}_x(x,y,z)\dif{y}\dif{z}+\iint_\Sigma {F}_y(x,y,z)\dif{x}\dif{z}+\iint_\Sigma {F}_z(x,y,z)\dif{x}\dif{y}\\
&=\iint_\Sigma{F}_x(x,y,z)\dif{y}\dif{z}+{F}_y(x,y,z)\dif{x}\dif{z}+{F}_z(x,y,z)\dif{x}\dif{y}\\
&=\iint_\Sigma \ve{F}(x,y,z)\cdot\dif{\ve{S}}
\end{align*}
其中$\dif{\ve{S}}$为有向面积微元

\subsection*{性质}
\begin{itemize}
  \item 若$\alpha$与$\beta$均为常数,则
  \[\iint_\Sigma[\alpha\ve{F}(P)+\beta\ve{G}(P)]\cdot\dif{\ve{S}}=\alpha\iint_\Sigma\ve{F}(P)\cdot\dif{\ve{S}}+\beta\iint_\Sigma\ve{G}(P)\cdot\dif{\ve{S}}\]
  \item 若有向曲面$\Sigma$可以被分为有限多个曲面$\Sigma_1,\Sigma_2,\cdots,\Sigma_n$,则
  \[ \iint_\Sigma\ve{F}(P)\cdot\dif{\ve{S}}=\sum^n_{i=1}\iint_{\Sigma_i} \ve{F}(P)\cdot\dif{\ve{S}} \]
  \item 设$\Sigma$是有向光滑曲面,$\Sigma^-$是$\Sigma$的反向光滑曲面(曲面重叠,但法向量相反),则
  \[\iint_{\Sigma^-} \ve{F}(P)\cdot\dif{\ve{S}}=-\iint_{\Sigma} \ve{F}(P)\cdot\dif{\ve{S}}\]
\end{itemize}

\subsection*{计算}
若曲面$\Sigma$可由参数方程$G(u,v)=\left( x(u,v),y(u,v),z(u,v) \right)$表示,
则函数$f(x,y,z)$对于曲面$\Sigma$的曲面积分可表示为
\[\iint_\Sigma \ve{F}(x,y,z)\cdot\dif{S}=\iint_D \ve{F}(x(u,v),y(u,v),z(u,v)\cdot\left({\frac{\partial \ve{G}}{\partial u}\cross\frac{\partial \ve{G}}{\partial v}}\right)\dif{u}\dif{v}\]
其中曲面$\Sigma$的正向与$\frac{\partial \ve{G}}{\partial u}\cross\frac{\partial \ve{G}}{\partial v}$的方向相同

\subsection*{两类曲线积分之间的联系}
\[\iint_\Sigma\ve{F}\cdot\dif{\ve{S}}=\iint_\Sigma F_N\dif{S}\]
其中$F_N$为向量$\ve{F}$在曲面法向量$\ve{N}$上的投影

\section*{高斯通量公式 Gauss' flux theorem}
设空间闭合区域$\Omega$是由分段光滑的闭曲面$\partial\Omega$所围成,若向量函数$\ve{F}(x,y,z)$
的各个分量$F_x$、$F_y$与$F_z$再$\Omega$上具有一阶连续偏导数,则
\[\Phi(\partial\Omega)=
\varoiint_{\partial\Omega}\ve{F}\cdot\dif{\ve{S}}=
\iiint_\Omega \left(\frac{\partial F_x}{\partial x}+\frac{\partial F_y}{\partial y}+\frac{\partial F_z}{\partial z}\right)\dif{v}=
\iiint_\Omega \nabla\cdot\ve{F}\dif{v}
\]
这里$\partial\Omega$是$\Omega$的整个边界曲面的外侧

\emplin

若对于空间区域$G$内任意比去买你所围成的区域全属于$G$,则称$G$是空间二维单连通区域;
如果$G$内任一闭曲线中可以组成一张完全属于$G$的曲面,则称$G$是空间一维单连通区域

\emplin

设$G$是空间二维单连通区域,若$P(x,y,z)$、$Q(x,y,z)$与$R(x,y,z)$在$G$内具有一阶连续偏导数,
则曲面积分
\[\iint_\Sigma P\dif{y}\dif{z}+Q\dif{x}\dif{z}+R\dif{x}\dif{y}\]
在$G$内与所取平面$\Sigma$无关,而只取决于$\Sigma$的边界曲线$\partial\Sigma$的充分必要条件是
\[\nabla\cdot\ve{F}=\frac{\partial P}{\partial x}+\frac{\partial Q}{\partial y}+\frac{\partial R}{\partial z}=0\]
在$G$内恒成立

\subsection*{格林第一公式与拉普拉斯算子}
设函数$u(x,y,z)$与$v(x,y,z)$在闭区域$\Omega$上具有一阶及二阶连续偏导数,则
\[\varoiint_{\partial\Omega}u\frac{\partial v}{\partial \ve{n}}\dif{S}=
\iiint_\Omega u\Delta v\dif{x}\dif{y}\dif{z}+
\iint_\Omega\left(
\frac{\partial u}{\partial x}\frac{\partial v}{\partial x}+
\frac{\partial u}{\partial y}\frac{\partial v}{\partial y}+
\frac{\partial u}{\partial z}\frac{\partial v}{\partial z}
\right)\]
其中$\partial\Omega$为闭区域$\Omega$的整个边界曲面,$\frac{\partial v}{\partial \ve{n}}$
为函数$u(x,y,z)$沿$\partial\Omega$的外法线方向的方向导数,
符号$\Delta=\frac{\partial^2}{{\partial x}^2}+\frac{\partial^2}{{\partial y}^2}+\frac{\partial^2}{{\partial z}^2}$
称为拉普拉斯(Laplace)算子,该公式称为格林第一公式

\subsection*{通量与散度 flux \& divergence}
设有向量场
\[\ve{F}=F_x\ve{i}+F_y\ve{j}+F_z\ve{k}\]
其中函数$F_x$、$F_y$与$F_z$均具有一阶连续偏导数,$\Sigma$是场内的一片有向曲面,$\ve{n}$是$\Sigma$
在点$(x,y,z)$处的单位法向量,则积分
\[\Phi(\Sigma)=\iint_\Sigma\ve{F}\cdot\ve{n}\dif{S}\]
称为向量场$\ve{F}$通过曲面$\Sigma$向着指定侧的通量

\emplin

对于向量场$\ve{F}$,定义其散度$\text{div}\ve{F}$为
\[\text{div}\ve{F}=\frac{\partial F_x}{\partial x}+\frac{\partial F_y}{\partial y}+\frac{\partial F_z}{\partial z}\]

若将$\ve{F}$视为描述一稳定不可压缩流体的流动速度场,
则$\text{div}\ve{F}(M)$可看作为该流体在点$M$的源头强度;
对于$\text{div}\ve{F}(M)>0$的点,流体从该点发散,即正源;
对于$\text{div}\ve{F}(M)<0$的点,流体从该点汇聚,即负源;
对于$\text{div}\ve{F}(M)=0$的点,流体从该点只改变流向,即无源

若用nabla算子
$\displaystyle\nabla=\left\langle\frac{\partial}{\partial x},\frac{\partial}{\partial y},\frac{\partial}{\partial z}\right\rangle$
来表示,则有
\[\text{div}\ve{F}=\nabla\cdot\ve{F}\]

若向量场$\ve{F}$的散度$\nabla\cdot\ve{F}$处处为$0$,则称$\ve{F}$为无源场,否则称其为有源场

\section*{斯托克公式 环流量与旋度}
\subsection*{斯托克公式 Stokes' theorem}
斯托克公式是格林公式的推广,即格林公式是斯托克公式在二维上的一个缺省

\emplin

设$\Gamma$为分段光滑的空间有向闭曲线,$\Sigma$是以$\Gamma$为边界的分片光滑的有向曲面,$\Gamma$的正向与$\Sigma$的侧复合右手规则,
若函数$F_x(x,y,z)$、$F_y(x,y,z)$与$F_z(x,y,z)$在曲面$\Sigma$(连同边界$\Gamma$)上具有一阶连续偏导数,则有
\[\iint_\Sigma
\left( \frac{\partial F_z}{\partial y} - \frac{\partial F_y}{\partial z} \right)\dif{y}\dif{z}+
\left( \frac{\partial F_x}{\partial z} - \frac{\partial F_z}{\partial x} \right)\dif{x}\dif{z}+
\left( \frac{\partial F_y}{\partial x} - \frac{\partial F_x}{\partial y} \right)\dif{x}\dif{y}
=
\oint_\Gamma F_x\dif{x}+F_y\dif{y}+F_z\dif{z}
 \]
为方便记忆,将其用行列式表示,则有
\[\iint_\Sigma \begin{vmatrix}
\dif{y}\dif{z}&\dif{x}\dif{z}&\dif{x}\dif{y}\\
\frac{\partial}{\partial x}&\frac{\partial}{\partial y}&\frac{\partial}{\partial z}\\
F_x&F_y&F_z
\end{vmatrix}
=
\oint_\Gamma F_x\dif{x}+F_y\dif{y}+F_z\dif{z}
\]
若引入环流量与旋度,则可表示为
\[\oint_\Gamma \ve{F}\cdot\dif{\ve{r}}=\iint_\Sigma \nabla\cross\ve{F}\cdot\dif{\ve{S}}\]

\subsection*{空间曲线积分的路径无关}
设空间区域$G$是一维单连通域,若函数$F_x(x,y,z)$、$F_y(x,y,z)$与$F_z(x,y,z)$在$G$内具有一阶连续偏导数,则空间积分$\int_\Gamma F_x\dif{x}+F_y\dif{y}+F_z\dif{z}$
在$G$内域路径无关的充分必要条件是
\[
\frac{\partial F_z}{\partial y} = \frac{\partial F_y}{\partial z}\quad
\frac{\partial F_x}{\partial z} = \frac{\partial F_z}{\partial x}\quad
\frac{\partial F_y}{\partial x} = \frac{\partial F_x}{\partial y}\quad
\quad\left(\text{ or }\nabla\cross\ve{F}=\ve{0}\right)
\]
在$G$内恒成立

\emplin

设区域$G$是空间一维单连通区域,若函数$F_x(x,y,z)$、$F_y(x,y,z)$与$F_z(x,y,z)$在$G$内具有一阶连续偏导数,则表达式$F_x\dif{x}+F_y\dif{y}+F_z\dif{z}$
在$G$内成为某一函数$u(x,y,z)$的全微分的充分必要条件是等式$\nabla\cross\ve{F}=\ve{0}$在$G$内恒成立;当该条件满足时,这函数为
\[ u(x,y,z)=\int_{(x_0,y_0,z_0)}^{(x,y,z)} F_x\dif{x}+F_y\dif{y}+F_z\dif{z}+u_0 \]
其中积分路径可任一取定,$u_0$为$(x_0,y_0,z_0)$处的值;
因而也可去
\[u(x,y,z)=\int_{x_0}^x F_x(x,y_0,z_0)\dif{x}+ \int_{y_0}^yF_y(x,y,z_0)\dif{y}+ \int_{z_0}^zF_z(x,y,z)\dif{z}+u_0\]
$(x_0,y_0,z_0)$为$G$内某一点

\subsection*{环流量与旋度 circulation\&rotation}
设有向量场
\[\ve{F}=F_x\ve{i}+F_y\ve{j}+F_z\ve{k}\]
其中函数$F_x$、$F_y$与$F_z$均具有一阶连续偏导数,$\Gamma$是场内的一条分段光滑的有向闭曲线,$\ve{r}$是$\Gamma$
在点$(x,y,z)$处的单位切向量,则积分
\[\oint_\Gamma\ve{F}\cdot\ve{r}\dif{s}\]
称为向量场$\ve{F}$沿有向曲线$\Gamma$的环流量,也可表示为
\[ \oint_\Gamma\ve{F}\cdot\dif{\ve{r}} \]

\emplin

对于向量场$\ve{F}$,定义其旋度$\text{rot}\ve{F}$为
\[\text{rot}\ve{F}=
\left( \frac{\partial F_z}{\partial y} - \frac{\partial F_y}{\partial z} \right)\ve{i}+
\left( \frac{\partial F_x}{\partial z} - \frac{\partial F_z}{\partial x} \right)\ve{j}+
\left( \frac{\partial F_y}{\partial x} - \frac{\partial F_x}{\partial y} \right)\ve{k}
\]
利用nabla算子来表示则有
\[\text{rot}\ve{F}=
\nabla\cross\ve{F}=
\begin{vmatrix}
\ve{i}&\ve{j}&\ve{k}\\
\frac{\partial}{\partial x}&\frac{\partial}{\partial y}&\frac{\partial}{\partial z}\\
F_x&F_y&F_z
\end{vmatrix}
\]
其表示向量场$\ve{F}$在一点上的旋转量,其旋转方向为$\nabla\cross\ve{F}$按右手规则定义
























\end{document}

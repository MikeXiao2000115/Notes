\documentclass[UTF8]{ctexart}

\usepackage{geometry}

\usepackage{amsmath}
\usepackage{amssymb}

\usepackage{bm}

\usepackage{fancyhdr}
\usepackage{graphicx}

\special{papersize={18.1cm,25.7cm}}
\geometry{left=1.5cm,right=0.5cm,top=2cm,bottom=1cm}
\pagestyle{empty}

\newcommand{\D}{{\text{d}\;\!}}
\newcommand{\cross}{\times}
\newcommand{\dif}[1]{{\text{d}\;\!#1}}
\newcommand{\dev}[1]{{\frac{\text{d}}{\dif{#1}}\;\!}}
\newcommand{\ve}[1]{{\bm{#1}}}
\newcommand{\ven}[2]{{\left\langle#1,#2\right\rangle}}
\newcommand{\veN}[3]{{\left\langle#1,#2,#3\right\rangle}}
\newcommand{\ang}[2]{{(\widehat{\ve{#1},\ve{#2}})}}
\newcommand{\abs}[1]{{\left|{#1}\right|}}
\newcommand{\when}[2]{{\left.{#1}\right|_{#2}}}
\newcommand{\dist}[2]{{\left\|\ve{#1}-\ve{#2}\right\|}}
\newcommand{\norm}[1]{{\left\|#1\right\|}}
\newcommand{\emplin}{\vspace{1em}}

\begin{document}

\section*{曲线积分 curve intergral}
设$L$为$xOy$面内的一条光滑曲线弧,函数$f(x,y)$在$L$上有界,在$L$上任意插入一点列$M_1,M_2,\cdots,M_{n-1}$
把$L$分成$n$个小段;设第$i$个小段的长度为$\Delta s_i$,又$(\xi_i,\eta_i)$为第$i$个小段上任意取定的一点,
作积$f(\xi_i,\eta_i)\Delta s_i$,并作和$\sum^n_{i=1}f(\xi_i,\eta_i)\Delta s_i$,如果当各小弧段的长度的最大值
$\lambda\to0$时,这和的极限总存在,且与曲线弧$L$的分法无关,那么称此极限为函数$f(x,y)$在曲线弧$L$上对弧长的曲线积分
或第一类曲线积分,记作$\int_Lf(x,y)\dif{s}$,即
\[\int_Lf(x,y)\dif{s}=\lim_{\lambda\to0}\sum^n_{i=1}f(\xi_i,\eta_i)\Delta s_i\]
其中$f(x,y)$称为被积函数,$L$叫做积分弧段

\emplin

类似的情况可以扩展至三维
\[\int_Lf(x,y,z)\dif{s}=\lim_{\lambda\to0}\sum^n_{i=1}f(\xi_i,\eta_i,\zeta_i)\Delta s_i\]

\emplin

\subsection*{性质}
\begin{itemize}
  \item 若曲线$L$可以被分为有限多条曲线$L_1,L_2,\cdots,L_n$,则
  \[ \int_Lf(P)\dif{s}=\sum^n_{i=1}\int_{L_i}f(P)\dif{s} \]
  \item 若曲线$L$为一闭合曲线,则将其记为$\displaystyle \oint_Lf(P)\dif{s}$
  \item 
\end{itemize}

\end{document}

\documentclass[UTF8]{ctexart}

\usepackage{geometry}

\usepackage{amsmath}
\usepackage{amssymb}

\usepackage{bm}

\usepackage{fancyhdr}
\usepackage{graphicx}

\special{papersize={18.1cm,25.7cm}}
\geometry{left=1.5cm,right=0.5cm,top=2cm,bottom=1cm}
\pagestyle{empty}

\newcommand{\D}{{\text{d}\;\!}}
\newcommand{\cross}{\times}
\newcommand{\dif}[1]{{\text{d}\;\!#1}}
\newcommand{\dev}[1]{{\frac{\text{d}}{\dif{#1}}\;\!}}
\newcommand{\ve}[1]{{\bm{#1}}}
\newcommand{\ven}[2]{{\left\langle#1,#2\right\rangle}}
\newcommand{\veN}[3]{{\left\langle#1,#2,#3\right\rangle}}
\newcommand{\ang}[2]{{(\widehat{\ve{#1},\ve{#2}})}}
\newcommand{\abs}[1]{{\left|{#1}\right|}}
\newcommand{\when}[2]{{\left.{#1}\right|_{#2}}}
\newcommand{\dist}[2]{{\left\|\ve{#1}-\ve{#2}\right\|}}

\begin{document}

\section*{多元函数 function of several variables}

\subsection*{$n$维空间}
$n$维实数坐标空间可表示为
\[\mathbb{R}^n=\mathbb{R}\times\mathbb{R}\times\cdots\times\mathbb{R}=\{(x_1,x_2,\cdots,x_n)|x_1,x_2,\cdots,x_n\in\mathbb{R}\}\]

设$\ve{x}=(x_1,x_2,\cdots,x_n)$,$\ve{y}=(y_1,y_2,\cdots,y_n)$为$\mathbb{R}^n$中任意两元素,
$\lambda\in\mathbb{R}$,规定
\[\ve{x}+\ve{y}=(x_1+y_1,x_2+y_2,\cdots,x_n+y_n)\]
\[\lambda\ve{x}=(\lambda x_1,\lambda x_2,\cdots,\lambda x_n)\]
它们的距离为
\[\rho(\ve{x},\ve{y})=\sqrt{\sum^n_{i=1}(x_i-y_i)^2}\]
记$\ve{x}$到零元的距离为$\|\ve{x}\|$,即
\[\|\ve{x}\|=\rho(\ve{x},\ve{0})=\sqrt{\sum^n_{i=1}x_i^2}\]
故
\[\rho(\ve{x},\ve{y}=\|\ve{x}-\ve{y}\|)\]

$\mathbb{R}^n$中变元的极限:
设$\ve{x}=(x_1,x_2,\cdots,x_n),\ve{a}=(a_1,a_2,\cdots,a_n)\in\mathbb{R}^n$,如果
\[\dist{x}{a}\to0\]
那么称变元$\ve{x}$在$\mathbb{R}^n$中趋于固定元$\ve{a}$,记作$\ve{x}\to\ve{a}$

即
\[\ve{x}\to\ve{a} \Leftrightarrow x_1\to a_1,x_2\to a_2,\cdots,x_n\to a_n\]

\subsection*{邻域}

在$\mathbb{R}^n$上的邻域 (neighbourhood)的概念,设$P_0$是$\mathbb{R}^n$上的一点呢,$\delta$为某一正数与点$P_0$距离小于$\delta$的点$P$的集合称为点
$P_0$的$\delta$邻域,记作$U(P_0,\delta)$,即
\[ U(P_0,\delta)=\{P\,|\,\dist{P}{P_0}<\delta\} \]

点$P_0$的去心$\delta$邻域 (deleted neighbourhood),记作$\mathring{U}(P_0,\delta)$,即
\[ U(P_0,\delta)=\{P\,|\,0<\dist{P}{P_0}<\delta\} \]

如果不需要特别强调邻域的半径$\delta$,则用$U(P_0)$表示$P_0$的某个邻域,点$P_0$的去心邻域记为$\mathring{U}(P_0)$


\subsection*{点与点集之间的关系}

对于任意一点$P\in\mathbb{R}^n$与任一点集$E\subset\mathbb{R}^n$必为一下三种关系中的一个
\begin{itemize}
  \item {\bf 内点 (interior point)} 如果$\exists U(P)\subset E$,则称$P$为$E$的内点
  \item {\bf 外点 (exterior point)} 如果$\exists U(P)\cap E = \varnothing$,则称$P$为$E$的外点
  \item {\bf 边界点 (boundary point)} 如果$\forall U(P), \exists P_{in},P_{ex}\in U(P)\text{ let }P_{in}\in E,P_{ex}\notin E$,则称$P$为$E$的边界点
\end{itemize}

$E$的边界点的全体称为$E$的边界 (boundary),记作$\partial E$

聚点 (limit point):如果对于任意给定的$\delta>0$,点$P$的去心邻域$\mathring{U}(P,\delta)$内总有$E$中的点,那么称点$P$是$E$的聚点($E\cup\partial E$)

\begin{itemize}
  \item {\bf 开集 (open set)} 如果$E$的点都是$E$的内点,则称$E$为开集
  \item {\bf 闭集 (closed set)} 如果$\partial E\subset E$,则称$E$为闭集
  \item {\bf 连通集 (connected set)} 如果$E$内任意两点都可以用折线联结起来,且该折线上的点都属于$E$,则称$E$为连通集
  \item {\bf (开)区域 (domain)} 连通的开集,称为区域
  \item {\bf 闭区域 (bounded domain)} 开区域与其边界的合集,称为闭区域
  \item {\bf 有界集 (bounded set)} 如果$\exists r, \text{ let }E\subset U(O,r)$($O$为原点),那么称$E$为有界集
  \item {\bf 无界集 (unbounded set)} 如果$E$不是有界集,它就是无界集
\end{itemize}

\subsection*{多元函数}
设$D$是$\mathbb{R}^n$的一个非空子集,称映射$f:D\to\mathbb{R}$为定义在$D$上的$n$元函数,通常记为
\[z=f(P),P\in D\]
其中点集$D$称为该函数的定义域,$P$称为自变量,$z$称为因变量

多元函数的值域
\[f(D)={z\,|\,z=f(P),P\in D}\]

\subsection*{多元函数的极限}
设$n$元函数$f(P)$的定义域为$D$,点$P_0$为$D$的聚点,如果存在常数$A$,对于任意给定的正数$\epsilon$,总存在正数$\delta$,
使得当点$P(x,y)\in D\cap\mathring{U}(P_0,\delta)$时,都有
\[ |f(P)-A|<\epsilon \]
成立,那么就称常数$A$为函数$f(P)$当$P\to P_0$时的极限,记作
\[ \lim_{P\to P_0}f(P)=A\text{ or }f(P)\to A(P\to P_0) \]

\subsection*{多元函数的连续性}
设多元函数$f(P)$的定义域为$D$,$P_0$为$D$的聚点,且$P_0\in D$,如果
\[\lim_{P\to P_0}f(P)=f(P_0)\]
那么称函数$f(x)$在点$P_0$连续

设函数$f(P)$的定义域为$D$,$P_0$是$D$的聚点,如果函数$f(P)$在点$P_0$不连续,那么称$P_0$为函数$f(P)$的断点

{\bf 性质}
\begin{itemize}
  \item 在有界闭区间$D$上的多元连续函数,必定在$D$上有界,且能取得它的最大值与最小值
  \item 在有界闭区间$D$上的多元连续函数必取得介于最大值与最小值之间的任意值
  \item 在有界闭区间$D$上的多元连续函数必定在$D$上一致连续($\forall \epsilon>0,\exists \delta>0,\text{ let }|f(P_1)-f(P_2)|<\epsilon\text{ when }\dist{P_1}{P_2}<\delta$)
\end{itemize}
\bigskip
\bigskip

\section*{偏导数}

\bigskip

设函数$z=f(P)$在点$P_0$的某一邻域内有定义,对于变元$x_i$有增量$h$,对应函数的增量
\[f(P_0+(0,0,\cdots,h,\cdots,0))-f(P_0)\]
如果
\[ \lim_{h\to0}\frac{f(P_0+(0,0,\cdots,h,\cdots,0))-f(P_0)}{h} \]
存在,那么称此极限为函数$z=f(P)$在点$P_0$处对于$x_i$的偏导数 (partial derivative),记作
\[\when{\frac{\partial z}{\partial x_i}}{P=P_0},\when{\frac{\partial f}{\partial x_i}}{P=P_0}\text{ or }f_{x_i}(P_0)\]

如果函数$z=f(P)$在区间$D$内任意点均可导,那么这些点位置与其对应的偏导数可以构成一个新的函数,
称为偏导函数,记为
\[\frac{\partial z}{\partial x_i},\frac{\partial f}{\partial x_i}\text{ or }f_{x_i}(P)\]

\subsection*{高阶偏导数}
设函数$z=f(P)$在区域$D$内具有偏导函数,如果对该偏导函数继续求偏导,得二阶偏导函数,如果求导对象不同则称为混合偏导,如
\[ \frac{\partial}{\partial x_j}\left(\frac{\partial z}{\partial x_i}\right)=\frac{\partial^2z}{\partial x_i \partial x_j}=f_{x_ix_j}f(P) \]

如果函数$z=f(P)$的两个二阶混合偏导数$\frac{\partial^2z}{\partial x_i \partial x_j}$与$\frac{\partial^2z}{\partial x_j \partial x_i}$在区域$D$内连续,
那么在该区间内这两个二阶混合偏导函数必定相等

\subsection*{拉普拉斯 (Laplace)方程}
\[\frac{\partial^2z}{\partial x^2}+\frac{\partial^2z}{\partial y^2}=0\quad(z=\ln\sqrt{x^2+y^2})\]
\[\frac{\partial^2u}{\partial x^2}+\frac{\partial^2u}{\partial y^2}+\frac{\partial^2u}{\partial z^2}=0\quad\left(z=\frac{1}{\sqrt{x^2+y^2+z^2}}\right)\]
\bigskip
\bigskip

\section*{全微分 total derivative}

\bigskip

设函数$z=f(x)$在点$P$的某邻域内有定义,如果函数在点$P$的某邻域内有定义,如果函数在点$P$的全增量
\[\Delta z=f(P+(\Delta x_1,\Delta x_2,\cdots,\Delta x_n))-f(P)\]
维实数坐标空间可表示为
\[\Delta z=\sum^n_{i=1}A_i\Delta x_i+o(\rho)\]
其中$A_i$不依赖与$\Delta x_i$而仅与$P$有关,$\rho=\|(\Delta x_1,\Delta x_2,\cdots,\Delta x_n)\|$,那么称
函数$z=f(P)$在点$P$可微分,而$\displaystyle \sum^n_{i=1}A_i\Delta x_i$称为函数$f(P)$在点$P$的全微分,记作$\dif{z}$
\[\dif{z}=\sum^n_{i=1}A_i\Delta x_i\]

\bigskip

如果函数在区域$D$内各点处都可微分,那么称这函数在$D$内可微分

\bigskip

如果函数$z=f(P)$在点$P$可微分,那么该函数在点$P$的偏导$\frac{\partial z}{\partial x_i}$必定存在,
且函数$z=f(P)$在点$P$的全微分为
\[\dif{z}=\sum^n_{i=1}\frac{\partial z}{\partial x_i}\Delta x_i\]
如果函数$z=f(P)$的偏导数$\frac{\partial z}{\partial x_i}$在点$P$连续,那么函数在改点可微分
\bigskip
\bigskip

\section*{多元复合函数的求导法则}

\bigskip

\subsection*{一元函数与多元函数复合的情形}
如果函数$u_i=\varphi_i(t)$在点$t$可导,函数$z=f(u_1,u_2,\cdots,u_n)$在对应点$(u_1,u_2,\cdots,u_n)$具有连续的偏导数,
那么复合函数$z=f[\varphi_1(t)],\varphi_2(t),\cdots,\varphi_n(t)]$在点$t$可导,且有
\[\frac{\dif{z}}{\dif{t}}=\sum^n_{i=1}\frac{\partial z}{\partial u_i}\frac{\dif{u_i}}{\dif{t}}\]

\subsection*{多元函数与多元函数复合的情形}
如果函数$v_i=\psi_i(P)$都在点$P=(u_1,u_2,\cdots,u_n)$具有对其各个分量的偏导,函数$z=f(v_1,v_2,\cdots,v_n)$在对应点$(v_1,v_2,\cdots,v_n)$具有连续偏导数,
那么复合函数$z=f(\psi_1(P),\psi_2(P),\cdots,\psi_n(P))$在点$P$的各个偏导数均存在,且
\[\frac{\partial z}{\partial u_i} = \sum^n_{i=1}\frac{\partial z}{\partial v_i}\frac{\partial v_i}{\partial u_i}\]

\subsection*{其他情况}
其他任意情况均可通过构造中间函数,将其转化为多元函数与多元函数复合的情形
\bigskip
\bigskip

\section*{隐函数的求导公式}

\bigskip
\subsection*{二元函数}
设函数$F(x,y)$在点$P(x_0,y_0)$的某一邻域内具有连续的偏导数,且$F(x_0,y_0)=0$,$F_y(x_0,y_0)\ne0$,
则方程$F(x,y)=0$在点$(x_0,y_0)$的某一邻域内恒能唯一确定一个连续且具有连续导数的函数$y=f(x)$,它满足条件$y_0=f(x_0)$,并有
\[ \frac{\dif{y}}{\dif{x}}=-\frac{F_x}{F_y} \]

\subsection*{多元函数}
设函数$F(P)$在点$P = (x_i,x_j,\cdots)$的某一邻域内具有连续的偏导数,且$F(P_0)=0$,$F_{x_j}(P_0)\ne0$,
则方程$F(P)=0$在点$P_0$的某一邻域内恒能确定唯一的连续且具有连续偏导的函数$x_j=f(x_i,\cdots)$,
它满足条件${x_j}_0=f({x_i}_0,\cdots)$,且有
\[\frac{\partial x_j}{\partial x_i} = -\frac{F_{x_i}}{F_{x_j}}\]

\subsection*{方程组}
设一方程组$\displaystyle
\begin{cases}
  F_1(x_1,x_2,\cdots,x_n,y_1,y_2,\cdots,y_k)=0
\\F_2(x_1,x_2,\cdots,x_n,y_1,y_2,\cdots,y_k)=0
\\\cdots\cdots\cdots\cdots\cdots\cdots\cdots\cdots\cdots\cdots\cdots\cdots\\
  F_k(x_1,x_2,\cdots,x_n,y_1,y_2,\cdots,y_k)=0
\end{cases}$,其中$F_i(x_1,x_2,\cdots,x_n,y_1,y_2,\cdots,y_k)$在点$P_0$具有连续的偏导数,
又$F_i(P_0)=0$,且雅可比 (Jacobi)行列式
\[Jac(P)=\frac{\partial(F_1,F_2,\cdots,F_k)}{\partial(y_1,y_2,\cdots,y_k)}=
{
\renewcommand\arraystretch{1.5}
\begin{vmatrix}
  \displaystyle\frac{\partial F_1}{\partial y_1}&\displaystyle\frac{\partial F_1}{\partial y_2} &\displaystyle\cdots&\displaystyle\frac{\partial F_1}{\partial y_k} \\
  \displaystyle\frac{\partial F_2}{\partial y_1}&\displaystyle\frac{\partial F_2}{\partial y_2} &\displaystyle\cdots&\displaystyle\frac{\partial F_2}{\partial y_k} \\
  \displaystyle\vdots&\displaystyle\vdots&\displaystyle\ddots&\displaystyle\vdots \\
  \displaystyle\frac{\partial F_k}{\partial y_1}&\displaystyle\frac{\partial F_k}{\partial y_2} &\displaystyle\cdots&\displaystyle\frac{\partial F_k}{\partial y_k}
\end{vmatrix}
}
\]
在点$P_0$不为$0$ ($Jac(P_0)\ne0$),则方程组在点$P_0$的某一邻域内恒能唯一确定一组连续且具有连续偏导数的函数
$y_i=f_i(x_1,x_2,\cdots,x_n)$,它们满足条件${y_i}_0=f_i({x_1}_0,{x_2}_0,\cdots,{x_n}_0)$,并有
\[\frac{\partial f_i}{\partial x_j}=-\frac{1}{Jac(P_0)}\frac{\partial(F_1,F_2,\cdots,F_k)}{\partial(y_1,y_2,\cdots,y_{i-1},x_j,y_{i+1},\cdots,y_k)}\]
\bigskip
\bigskip

\section*{多元函数微分学的几何应用}

\bigskip
\subsection*{一元向量值函数与导数}
设数集$D\subset\mathbb{R}$,则称映射$f:D\to\mathbb{R}^n$为一元向量值函数,通常记为
\[\ve{r}=\ve{f}(t)\quad t\in D\]
其中数集$D$称为该函数的定义域,$t$称为自变量,$\ve{r}$称为因变量

\bigskip

设向量函数$\ve{f}(t)$在点$t_0$的某一去心邻域内有定义,如果存在一个常向量$\ve{r}_0$对于任意给定正数$\epsilon$,
总存在正数$\delta$,使得当$t\in\mathring{U}(t_0,\delta)$时,对应函数值$\ve{f}(t)$都满足不等式
\[|\ve{f}(t)-\ve{r}_0|<\epsilon\]
那么,常向量$\ve{r}_0$就叫做向量值函数$\ve{f}(t)$当$t\to t_0$时的极限,记作
\[\lim_{t\to t_0}\ve{f}(t)=\ve{r}_0\text{ or }\ve{f}(t)\to\ve{r}_0,t\to t_0\]

\bigskip

如果$\ve{f}(t)=\left\langle f_1(t),f_2(t),\cdots,f_n(t) \right\rangle$
\[\lim_{t\to t_0}\ve{f}(t)=\left\langle \lim_{t\to t_0}f_1(t),\lim_{t\to t_0}f_2(t),\cdots,\lim_{t\to t_0}f_n(t) \right\rangle\]

\bigskip

设向量值函数$\ve{f}(t)$在点$t_0$的某一邻域内有定义,如果
\[ \lim_{\Delta t\to0}\frac{\Delta \ve{r}}{\Delta t}=\lim_{\Delta t\to0}\frac{\ve{f}(t_0+\Delta t)-\ve{f}(t_0)}{\Delta t} \]
存在,那么就称这个极限向量为向量函数$\ve{r}=\ve{f}(t)$在$t_0$处的导数或导向量,
记作$\ve{f'}(t)$或$\when{\frac{\dif{\ve{r}}}{\dif{t}}}{t=t_0}$

设向量函数$\ve{r}=\ve{f}(t)$,$t\in D$,若$D_1\subset D$,$\ve{f}(t)$在$D_1$中的每一点$t$处都存在导向量
$\ve{f'}(t)$,则称$\ve{f}(t)$在$D_1$上可导

\bigskip

设向量值函数$\ve{f}(t)$在点$t_0$可导的充分必要条件是其各个分类在$t_0$都可导,且
\[ \ve{f'}(t)=\left\langle f_1'(t),f_2'(t),\cdots,f_n'(t) \right\rangle \]

\subsection*{一元向量值函数导数的性质}

\begin{itemize}
  \item $\displaystyle\dev{t}\ve{C}=\ve{0}$
  \item $\displaystyle\dev{t}[c\ve{u}(t)]=c\ve{u'}(t)$
  \item $\displaystyle\dev{t}[\ve{u}(t)\pm\ve{v}(t)]=\ve{u'}(t)\pm\ve{v'}(t)$
  \item $\displaystyle\dev{t}[\varphi(t)\ve{u}(t)]=\varphi'(t)\ve{u}(t)+\varphi(t)\ve{u'}(t)$
  \item $\displaystyle\dev{t}[\ve{u}(t)\cdot\ve{v}(t)]=\ve{u'}(t)\cdot\ve{v}(t)+\ve{u}(t)\cdot\ve{v'}(t)$
  \item $\displaystyle\dev{t}[\ve{u}(t)\times\ve{v}(t)]=\ve{u'}(t)\times\ve{v}(t)+\ve{u}(t)\times\ve{v'}(t)$
  \item $\displaystyle\dev{t}\ve{u}[\varphi(t)]=\varphi'(t)\ve{u'}[\varphi(t)]$
\end{itemize}

\subsection*{空间曲线的切线与法平面}
设空间曲线$\Gamma$的参数方程为
\[
\begin{cases}
  x=\varphi(t)\\
  y=\psi(t)\\
  z=\omega(t)
\end{cases}
\]
其中$t\in[\alpha,\beta]$,则向量$\ve{T}=\ve{f'}(t_0)=\left\langle \varphi'(t_0),\psi'(t_0),\omega'(t_0) \right\rangle$
就是曲线$\Gamma$在点$M$处的一个切向量,从而曲线$\Gamma$在点$M(x_0,y_0,z_0)$处的切线方程为
\[\frac{x-x_0}{\varphi'(t_0)}=\frac{y-y_0}{\psi'(t_0)}=\frac{z-z_0}{\omega'(t_0)}\]
通过点$M$且与切线垂直的平面称为曲线$Gamma$在点$M$处的法平面,它是通过点$M$,以$\ve{T}$为法向量的平面
\[\varphi'(t_0)(x-x_0)+\psi'(t_0)(y-y_0)+\omega'(t_0)(z-z_0)=0\]

\begin{itemize}
  \item $\displaystyle\text{unit tangent vector}\quad\ve{T}(t) = \frac{\ve{f'}(t)}{\|\ve{f'}(t)\|}$
  \item $\displaystyle\text{unit normal vector}\quad\ve{N}(t)=\frac{\ve{T'}(t)}{\|\ve{T'}(t)\|}$
  \item $\displaystyle\text{binormal vector}\quad\ve{B}(t)=\ve{T}\times\ve{N}$
\end{itemize}

\subsection*{曲面的切平面与法线}
引入向量
\[\ve{n}=\veN{F_x(P)}{F_y(P)}{F_z(P)}\]
为曲面的法向量,它垂直于曲面$F(x,y,z)=0$上任意一条过点$P$的曲线,
而曲面的切平面为
\[\ve{n}\cdot((x,y,z)-P_0)=0\]
\bigskip
\bigskip

\section*{方向导数与梯度}

\bigskip
\subsection*{方向导数 directional derivative}
如果函数$f$在点$P_0$可微分,那么函数在该点沿任一方向$\ve{l}$的方向导数存在,且有
\[D_\ve{l}f(P_0)=\when{\frac{\partial f}{\partial \ve{l}}}{P_0}=\sum^n_{i=1}f_{x_i}(P_0)\cos\alpha_i\]
其中$\cos\alpha_i$是方向$\ve{l}$的方向余弦

\subsection*{梯度 gradient}
对于任一函数$f(P)$可定出一向量
\[\left\langle f_{x_1}(P_0),f_{x_2}(P_0),\cdots,f_{x_n}(P_0) \right\rangle\]
这个向量称为函数$f(P)$在点$P_0$的梯度,记作$\textbf{grad}f(P_0)$或$\nabla f(P_0)$,即
\[ \textbf{grad}f(P_0)=\nabla f(P_0)=\left\langle f_{x_1}(P_0),f_{x_2}(P_0),\cdots,f_{x_n}(P_0) \right\rangle \]
其中$\displaystyle\nabla = \left\langle \frac{\partial}{\partial x_1},\frac{\partial}{\partial x_2},\cdots,\frac{\partial}{\partial x_n} \right\rangle$
称为向量微分算子或Nabla算子

\bigskip

因而,方向导数也可表示为
\[\when{\frac{\partial f}{\partial \ve{l}}}{P_0}=\nabla f(P_0)\cdot\ve{l}\]

\subsection*{数量场与向量场 scalar field \& vector field}
如果对于空间区域$G$内的任一点$M$,都有一个确定的数量$f(M)$,那么称在这空间区域$G$内确定了一个数量场,
一个数量场可以用一个数量函数$f(M)$表示;如果点$M$对应一个确定的矢量$\ve{F}(M)$,那么称在着空间区域$G$内确定了一个向量场,
一个向量场可以用一个向量值函数$\ve{F}(M)$表示

\subsection*{势函数与势场 Scalar potential}
若矢量场$\ve{F}(M)$是某个数量函数$f(M)$的梯度($\nabla f(M)=\ve{F}(M)$),则称$f(M)$是向量场$\ve{F}(M)$的一个势函数,
并称向量场$\ve{F}(M)$为势场
\bigskip
\bigskip

\section*{多元函数的极值及其求法}

\bigskip
\subsection*{多元函数的极值与最大最小值}
设函数$z=f(P)$的定义域为$D$,$P_0$为$D$的内点,若存在$P_0$的某个邻域$U(P_0)\subset D$,使得对于该区间内异于
$P_0$的任何点$P$,都有
\[ f(P)<f(P_0) \]
则称函数$f(P)$在点$P_0$有极大值$f(P_0)$,点$P_0$称为函数$f(P)$的极大值点;
若存在$P_0$的某个邻域$U(P_0)\subset D$,使得对于该区间内异于
$P_0$的任何点$P$,都有
\[ f(P)>f(P_0) \]
则称函数$f(P)$在点$P_0$有极小值$f(P_0)$,点$P_0$称为函数$f(P)$的极小值点;
极大值与极小值统称为极值,使i函数取得极值的点称为极值点

\subsection*{极值判断}
设函数$z=f(P)$在点$P_0$具有偏导数,且在点$P_0$取得极值,那么$f(P)$在这个点的各个偏导数均为$0$

\bigskip

设函数$z=f(x,y)$在点$(x_0,y_0)$的某邻域内连续且有一阶及二阶连续偏导数,又$f_x(x_0,y_0)=0,f_y(x_0,y_0)=0$,令
\[D={\renewcommand{\arraystretch}{1}\begin{vmatrix}f_{xx}&f_{xy}\\f_{yx}&f_{yy}\end{vmatrix}}\]
则$f(x,y)$在点$(x_0,y_0)$处是否取得极值的条件如下:
\begin{itemize}
  \item $D>0$ 具有极值;$f_{xx}<0$为极大值,$f_{xx}>0$为极小值
  \item $D<0$ 不具有极值 (saddle)
  \item $D=0$ 均有可能,需另行判断
\end{itemize}


\subsection*{条件极值 拉格朗日乘数 Lagrange Multipliers}
对于函数$f(P)$在满足条件$g(P)=0$下的极值点$P_0$需满足
\[\nabla f(P_0)=\lambda\nabla g(P_0)\]
其中$\lambda$为拉格朗日乘数

拉格朗日乘数可以有个(每个对于一个限制条件)
\bigskip
\bigskip

\section*{二元函数的泰勒公式}

\bigskip
设$z=f(x,y)$在点$(x_0,y_0)$的某一邻域内连续且有$(n+1)$阶连续偏导数,$(x_0+h,y_0+k)$为此领域内任一点,
则有
\[
\begin{aligned}
  f(x_0+h,y_0+k)
  =&f(x_0,y_0)+
  \left(h\frac{\partial}{\partial x}+k\frac{\partial}{\partial y}\right)f(x_0,y_0)+
  \frac{1}{2!}{\left(h\frac{\partial}{\partial x}+k\frac{\partial}{\partial y}\right)}^2f(x_0,y_0)+\cdots+\\
  &\frac{1}{n!}{\left(h\frac{\partial}{\partial x}+k\frac{\partial}{\partial y}\right)}^nf(x_0,y_0)+
  \frac{1}{(n+1)!}{\left(h\frac{\partial}{\partial x}+k\frac{\partial}{\partial y}\right)}^{n+1}f(x_0+\theta h,y_0+\theta k)\quad(0<\theta<1)
\end{aligned}
\]
其中记号
\[
{\left(h\frac{\partial}{\partial x}+k\frac{\partial}{\partial y}\right)}^m
\Leftrightarrow
\when{\sum^m_{p=0}\text{C}_m^ph^pk^{m-p}\frac{\partial^m f}{\partial x^p \partial y^{m-p}}}{(x_0,y_0)}
\]

\bigskip

如果函数$f(x,y)$的偏导数$f_x$,$f_y$在某一区域内都恒为零,那么函数$f(x,y)$在该区域内为一常数

\end{document}

\documentclass[UTF8]{ctexart}

\usepackage{geometry}

\usepackage{amsmath}
\usepackage{amssymb}

\usepackage{bm}

\usepackage{fancyhdr}
\usepackage{graphicx}

\special{papersize={18.1cm,25.7cm}}
\geometry{left=1.5cm,right=0.5cm,top=2cm,bottom=1cm}
\pagestyle{empty}

\newcommand{\D}{{\text{d}\;\!}}
\newcommand{\cross}{\times}
\newcommand{\dif}[1]{{\text{d}\;\!#1}}
\newcommand{\dev}[1]{{\frac{\text{d}}{\dif{#1}}\;\!}}
\newcommand{\ve}[1]{{\bm{#1}}}
\newcommand{\ven}[2]{{\left\langle#1,#2\right\rangle}}
\newcommand{\veN}[3]{{\left\langle#1,#2,#3\right\rangle}}
\newcommand{\ang}[2]{{(\widehat{\ve{#1},\ve{#2}})}}
\newcommand{\abs}[1]{{\left|{#1}\right|}}
\newcommand{\when}[2]{{\left.{#1}\right|_{#2}}}
\newcommand{\dist}[2]{{\left\|\ve{#1}-\ve{#2}\right\|}}

\begin{document}

\section*{多元函数 function of several variables}

\subsection*{$n$维空间}
$n$维实数坐标空间可表示为
\[\mathbb{R}^n=\mathbb{R}\times\mathbb{R}\times\cdots\times\mathbb{R}=\{(x_1,x_2,\cdots,x_n)|x_1,x_2,\cdots,x_n\in\mathbb{R}\}\]

设$\ve{x}=(x_1,x_2,\cdots,x_n)$,$\ve{y}=(y_1,y_2,\cdots,y_n)$为$\mathbb{R}^n$中任意两元素,
$\lambda\in\mathbb{R}$,规定
\[\ve{x}+\ve{y}=(x_1+y_1,x_2+y_2,\cdots,x_n+y_n)\]
\[\lambda\ve{x}=(\lambda x_1,\lambda x_2,\cdots,\lambda x_n)\]
它们的距离为
\[\rho(\ve{x},\ve{y})=\sqrt{\sum^n_{i=1}(x_i-y_i)^2}\]
记$\ve{x}$到零元的距离为$\|\ve{x}\|$,即
\[\|\ve{x}\|=\rho(\ve{x},\ve{0})=\sqrt{\sum^n_{i=1}x_i^2}\]
故
\[\rho(\ve{x},\ve{y}=\|\ve{x}-\ve{y}\|)\]

$\mathbb{R}^n$中变元的极限:
设$\ve{x}=(x_1,x_2,\cdots,x_n),\ve{a}=(a_1,a_2,\cdots,a_n)\in\mathbb{R}^n$,如果
\[\dist{x}{a}\to0\]
那么称变元$\ve{x}$在$\mathbb{R}^n$中趋于固定元$\ve{a}$,记作$\ve{x}\to\ve{a}$

即
\[\ve{x}\to\ve{a} \Leftrightarrow x_1\to a_1,x_2\to a_2,\cdots,x_n\to a_n\]

\subsection*{邻域}

在$\mathbb{R}^n$上的邻域 (neighbourhood)的概念,设$P_0$是$\mathbb{R}^n$上的一点呢,$\delta$为某一正数与点$P_0$距离小于$\delta$的点$P$的集合称为点
$P_0$的$\delta$邻域,记作$U(P_0,\delta)$,即
\[ U(P_0,\delta)=\{P\,|\,\dist{P}{P_0}<\delta\} \]

点$P_0$的去心$\delta$邻域 (deleted neighbourhood),记作$\mathring{U}(P_0,\delta)$,即
\[ U(P_0,\delta)=\{P\,|\,0<\dist{P}{P_0}<\delta\} \]

如果不需要特别强调邻域的半径$\delta$,则用$U(P_0)$表示$P_0$的某个邻域,点$P_0$的去心邻域记为$\mathring{U}(P_0)$


\subsection*{点与点集之间的关系}

对于任意一点$P\in\mathbb{R}^n$与任一点集$E\subset\mathbb{R}^n$必为一下三种关系中的一个
\begin{itemize}
  \item {\bf 内点 (interior point)} 如果$\exists U(P)\subset E$,则称$P$为$E$的内点
  \item {\bf 外点 (exterior point)} 如果$\exists U(P)\cap E = \varnothing$,则称$P$为$E$的外点
  \item {\bf 边界点 (boundary point)} 如果$\forall U(P), \exists P_{in},P_{ex}\in U(P)\text{ let }P_{in}\in E,P_{ex}\notin E$,则称$P$为$E$的边界点
\end{itemize}

$E$的边界点的全体称为$E$的边界 (boundary),记作$\partial E$

聚点 (limit point):如果对于任意给定的$\delta>0$,点$P$的去心邻域$\mathring{U}(P,\delta)$内总有$E$中的点,那么称点$P$是$E$的聚点($E\cup\partial E$)

\begin{itemize}
  \item {\bf 开集 (open set)} 如果$E$的点都是$E$的内点,则称$E$为开集
  \item {\bf 闭集 (closed set)} 如果$\partial E\subset E$,则称$E$为闭集
  \item {\bf 连通集 (connected set)} 如果$E$内任意两点都可以用折线联结起来,且该折线上的点都属于$E$,则称$E$为连通集
  \item {\bf (开)区域 (domain)} 连通的开集,称为区域
  \item {\bf 闭区域 (bounded domain)} 开区域与其边界的合集,称为闭区域
  \item {\bf 有界集 (bounded set)} 如果$\exists r, \text{ let }E\subset U(O,r)$($O$为原点),那么称$E$为有界集
  \item {\bf 无界集 (unbounded set)} 如果$E$不是有界集,它就是无界集
\end{itemize}

\subsection*{多元函数}
设$D$是$\mathbb{R}^n$的一个非空子集,称映射$f:D\to\mathbb{R}$为定义在$D$上的$n$元函数,通常记为
\[z=f(P),P\in D\]
其中点集$D$称为该函数的定义域,$P$称为自变量,$z$称为因变量

多元函数的值域
\[f(D)={z\,|\,z=f(P),P\in D}\]

\subsection*{多元函数的极限}
设$n$元函数$f(P)$的定义域为$D$,点$P_0$为$D$的聚点,如果存在常数$A$,对于任意给定的正数$\epsilon$,总存在正数$\delta$,
使得当点$P(x,y)\in D\cap\mathring{U}(P_0,\delta)$时,都有
\[ |f(P)-A|<\epsilon \]
成立,那么就称常数$A$为函数$f(P)$当$P\to P_0$时的极限,记作
\[ \lim_{P\to P_0}f(P)=A\text{ or }f(P)\to A(P\to P_0) \]

\subsection*{多元函数的连续性}
设多元函数$f(P)$的定义域为$D$,$P_0$为$D$的聚点,且$P_0\in D$,如果
\[\lim_{P\to P_0}f(P)=f(P_0)\]
那么称函数$f(x)$在点$P_0$连续

设函数$f(P)$的定义域为$D$,$P_0$是$D$的聚点,如果函数$f(P)$在点$P_0$不连续,那么称$P_0$为函数$f(P)$的断点

{\bf 性质}
\begin{itemize}
  \item 在有界闭区间$D$上的多元连续函数,必定在$D$上有界,且能取得它的最大值与最小值
  \item 在有界闭区间$D$上的多元连续函数必取得介于最大值与最小值之间的任意值
  \item 在有界闭区间$D$上的多元连续函数必定在$D$上一致连续($\forall \epsilon>0,\exists \delta>0,\text{ let }|f(P_1)-f(P_2)|<\epsilon\text{ when }\dist{P_1}{P_2}<\delta$)
\end{itemize}
\bigskip
\bigskip

\section*{偏导数}

\bigskip

设函数$z=f(P)$在点$P_0$的某一邻域内有定义,对于变元$x_i$有增量$h$,对应函数的增量
\[f(P_0+(0,0,\cdots,h,\cdots,0))-f(P_0)\]
如果
\[ \lim_{h\to0}\frac{f(P_0+(0,0,\cdots,h,\cdots,0))-f(P_0)}{h} \]
存在,那么称此极限为函数$z=f(P)$在点$P_0$处对于$x_i$的偏导数 (partial derivative),记作
\[\when{\frac{\partial z}{\partial x_i}}{P=P_0},\when{\frac{\partial f}{\partial x_i}}{P=P_0}\text{ or }f_{x_i}(P_0)\]

如果函数$z=f(P)$在区间$D$内任意点均可导,那么这些点位置与其对应的偏导数可以构成一个新的函数,
称为偏导函数,记为
\[\frac{\partial z}{\partial x_i},\frac{\partial f}{\partial x_i}\text{ or }f_{x_i}(P)\]

\subsection*{高阶偏导数}
设函数$z=f(P)$在区域$D$内具有偏导函数,如果对该偏导函数继续求偏导,得二阶偏导函数,如果求导对象不同则称为混合偏导,如
\[ \frac{\partial}{\partial x_j}\left(\frac{\partial z}{\partial x_i}\right)=\frac{\partial^2z}{\partial x_i \partial x_j}=f_{x_ix_j}f(P) \]

如果函数$z=f(P)$的两个二阶混合偏导数$\frac{\partial^2z}{\partial x_i \partial x_j}$与$\frac{\partial^2z}{\partial x_j \partial x_i}$在区域$D$内连续,
那么在该区间内这两个二阶混合偏导函数必定相等

\subsection*{拉普拉斯 (Laplace)方程}
\[\frac{\partial^2z}{\partial x^2}+\frac{\partial^2z}{\partial y^2}=0\quad(z=\ln\sqrt{x^2+y^2})\]
\[\frac{\partial^2u}{\partial x^2}+\frac{\partial^2u}{\partial y^2}+\frac{\partial^2u}{\partial z^2}=0\quad\left(z=\frac{1}{\sqrt{x^2+y^2+z^2}}\right)\]
\bigskip
\bigskip

\section*{全微分 total derivative}

\bigskip

设函数$z=f(x)$在点$P$的某邻域内有定义,如果函数在点$P$的某邻域内有定义,如果函数在点$P$的全增量
\[\Delta z=f(P+(\Delta x_1,\Delta x_2,\cdots,\Delta x_n))-f(P)\]
维实数坐标空间可表示为
\[\Delta z=\sum^n_{i=1}A_i\Delta x_i+o(\rho)\]
其中$A_i$不依赖与$\Delta x_i$而仅与$P$有关,$\rho=\|(\Delta x_1,\Delta x_2,\cdots,\Delta x_n)\|$,那么称
函数$z=f(P)$在点$P$可微分,而$\displaystyle \sum^n_{i=1}A_i\Delta x_i$称为函数$f(P)$在点$P$的全微分,记作$\dif{z}$
\[\dif{z}=\sum^n_{i=1}A_i\Delta x_i\]

\bigskip

如果函数在区域$D$内各点处都可微分,那么称这函数在$D$内可微分

\bigskip

如果函数$z=f(P)$在点$P$可微分,那么该函数在点$P$的偏导$\frac{\partial z}{\partial x_i}$必定存在,
且函数$z=f(P)$在点$P$的全微分为
\[\dif{z}=\sum^n_{i=1}\frac{\partial z}{\partial x_i}\Delta x_i\]
如果函数$z=f(P)$的偏导数$\frac{\partial z}{\partial x_i}$在点$P$连续,那么函数在改点可微分
\bigskip
\bigskip

\section*{多元复合函数的求导法则}

\bigskip

\subsection*{一元函数与多元函数复合的情形}
如果函数$u_i=\varphi_i(t)$在点$t$可导,函数$z=f(u_1,u_2,\cdots,u_n)$在对应点$(u_1,u_2,\cdots,u_n)$具有连续的偏导数,
那么复合函数$z=f[\varphi_1(t)],\varphi_2(t),\cdots,\varphi_n(t)]$在点$t$可导,且有
\[\frac{\dif{z}}{\dif{t}}=\sum^n_{i=1}\frac{\partial z}{\partial u_i}\frac{\dif{u_i}}{\dif{t}}\]

\subsection{多元函数与多元函数复合的情形}





\end{document}

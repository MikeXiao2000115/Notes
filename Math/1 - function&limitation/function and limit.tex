\documentclass[UTF8]{ctexart}

\usepackage{geometry}

\usepackage{amsmath}
\usepackage{amssymb}

\usepackage{fancyhdr}
\usepackage{graphicx}

\special{papersize={18.1cm,25.7cm}}
\geometry{left=1.5cm,right=0.5cm,top=2cm,bottom=1cm}
\pagestyle{empty}

\begin{document}



\section*{映射与函数 Mapping \& Functon}

\bigskip

设$X$, $Y$ 为两非空集合,如存在一个法则$f$(mapping),使对$X$中任意元素$x$,按法则$f$,
在$Y$中都有唯一确定元素$y$与之对应,那么成$f$为从$X$到$Y$的映射,记作:
\[
  f:\,X\to Y
\]
其中$y$称为$x$(在映射$f$下)的像 (image),并记作$f(x)$,即:
\[
  y=f(x)
\]
而元素$x$称为元素$y$(在映射$f$下)的原像 (preimage / inverse image);
集合$X$称为映射$f$的定义域 (domain),记作$D_f$;
$X$中所有元素的想所组成的集合称为映射$f$的值域(range),记作$R_f$或$f(X)$

\bigskip
\bigskip

\begin{center}
  \begin{tabular}{crl}
  单射 & injection & $x_1,x_2\in X\,and\,f(x_1)=f(x_2)\quad\Rightarrow\quad x_1=x_2$\\
  满射 & surjection & $R_f=Y; \,\forall y\in Y,\,\exists x\in X, let\, f(x)=y$\\
  双射 & bijection & both surjection and bijection ($X\leftrightarrow Y$)
  \end{tabular}
\end{center}

\bigskip

设$f$为$X\to Y$的单射,定义一个新的映射$f^{-1}$:$R_f\to X$,对每一个$y\in R_f$,
规定$f^{-1}(y)=x$,使$f(x)=y$;$f^{-1}$称为$f$的逆映射 (inverse mapping)
 ($D_{f^{-1}}=R_f$ and $R_{f^{-1}}=X$)

\bigskip

设$g$:$X\to Y_1$,$f$:$Y_2\to Z$,$Y_1\subset Y_2$,
则符合映射 (mapping composition)$f\circ g$确定了一个从$X$到$Z$的映射,
$f\circ g=f[g(x)]$

\bigskip

设数集 $D\subset \mathbb{R}$,则称映射$f$:$D\to\mathbb{R}$为定义在$D$上的函数 (function),
通常记为$y=f(x)$,$x\in D$,其中$x$为自变量 (independent variable),
$y$为因变量 (dependent variable); $D_f=D$

\bigskip

平面上的点集 {$P(x,y)| y=f(x),x\in D$} 为函数 $y=f(x)$,$x\in D$的图像 (graph)

\bigskip

函数的有界性

\begin{center}
  \begin{tabular}{rrr}
    $f(x)\le K_1 for any x \in D$ & upper bounded & 有上界\\
    $f(x)\ge K_2 for any x \in D$ & lower bounded & 有下界\\
    $|f(x)|\le M for any x \in D$ & bounded & 有界\\
    $M$ is not exsit & unbounded & 无界
  \end{tabular}
\end{center}

\bigskip

函数的单调性 Monotonic

\begin{center}
  \begin{tabular}{cccc}
    $x_1,x_2 \in D$ & $x_1<x_2\quad\Rightarrow\quad f(x_1)<f(x_2)$ & monotonically increasing & 单调递增\\
                    & $x_1>x_2\quad\Rightarrow\quad f(x_1)>f(x_2)$ & monotonically decreasing & 单调递减
  \end{tabular}
\end{center}

\bigskip

函数的奇偶性

\begin{center}
  \begin{tabular}{rrcl}
    $ f(x)=f(-x)$ & even function & 偶函数 & symmetic with y-axis\\
    $-f(x)=f(-x)$ &  odd function & 奇函数 & rotational sysmmetic with origin
  \end{tabular}
\end{center}

\bigskip

函数的周期性 Periodicity

$T\in\mathbb{R}^+$ 使 $f(x+kT)=f(x),\quad k\in\mathbb{N}$,则 $f(x)$ 为周期函数,
$T$为周期 (period)

\bigskip

设函数$f$:$D\to f(D)$是单射,则它存在逆映射$f^{-1}$:$f(D)\to D$,称此映射$f^{-1}$
为函数$f$的反函数 (inverse function)

相对于反函数$y=f^{-1}(x)$来说,原来的函数$y=f(x)$称为原函数
(原函数与反函数的图像沿$y=x$对称)

\bigskip

设函数$y=f(u)$的定义域为$D_f$,函数$u=g(x)$的定义域为$D_g$,且其值域$R_g\subset D_f$,
则由$y=f[g(x)]$ ($x\in D_g$) 确定的函数称为由函数$u=g(x)$和函数$y=f(u)$构成的复合函数 (function composition),
它的定义域为$D_g$,变量$u$称为中间变量 (intermediate variables)

\bigskip

\textbf{基本初等函数 (basic elementary function)}

\begin{center}
  \begin{tabular}{rrcll}
    冥函数 & Power function & $y=x^\mu$ & $\mu\in\mathbb{R}$\\
    指数函数 & Exponential function & $y=\alpha^x$ & $\alpha > 0$\\
    对数函数 & Logarithm & $y=\log_\alpha x$ & $\alpha>0$ and $\alpha\ne1$\\
    三角函数 & Trigonometric function & $y=\sin{x}$, $y=\cos{x}$, and $y=\tan{x}$\\
    反三角函数 & Inverse trigonometric function & $y=\arcsin{x}$, $y=\arccos{x}$, and $y=\arctan{x}$
  \end{tabular}
\end{center}
由基本初等函数经过有限次四则运算和有限次复合后得到的函数时初等函数
\bigskip

双曲函数 (Hyperbolic function)

\begin{center}
  \begin{tabular}{cl}
    双曲正弦 & $\sinh{x}=\frac{e^x-e^{-x}}{2}$\\
    双曲余弦 & $\cosh{x}=\frac{e^x+e^{-x}}{2}$\\
    双曲正切 & $\tanh{x}=\frac{\sinh{x}}{\cosh{x}}=\frac{e^x-e^{-x}}{e^x+e^{-x}}$
  \end{tabular}
\end{center}

\bigskip

双曲函数公式

\begin{center}
  \begin{tabular}{l}
    $\sinh{x\pm y}=\sinh{x}\cosh{y}\pm\cosh{x}\sinh{y}$\\
    $\cosh{x\pm y}=\cosh{x}\cosh{y}\mp\sinh{x}\sinh{y}$\\
    $\cosh^2{x} - \sinh^2 = 1$\\
    $\sinh{2x}=2\sinh{x}\cosh{x}$\\
    $\cosh{2x}=\cosh^2{x}+\sinh^2{x}$
  \end{tabular}
\end{center}

\bigskip

反双曲函数

\begin{center}
  \begin{tabular}{l}
    $y=\sinh^{-1}{x}=\ln(x+\sqrt{x^2+1})$\\
    $y=\cosh^{-1}{x}=\ln(x+\sqrt{x^2-1})$\\
    $y=\tanh^{-1}{x}=\frac{1}{2}\ln\frac{1+x}{1-x}$
  \end{tabular}
\end{center}


\bigskip
\bigskip
\section*{数列的极限 The limit of sequence}

\bigskip

如果按照某一法则,对每个$n\in\mathbb{N}^+$,对应着一个确定的实数$x_n$,
这些实数$x_n$按照下标$n$从小到大排列得到一个序列
\[
x_1,x_2,x_3,\cdots,x_n,\cdots
\]
就叫做数列(sequence),简记为数列$\{x_n\}$

数列中的每一个数字叫做数列的项 (term),第$n$项$x_n$叫做数列的一般项(或通项) (general term)

\bigskip

设$\{x_n\}$为一数列,如果存在常数$a$,对于任意给定的正数$epsilon$,总存在正整数$N$,使当$n>N$时,
不等式
\[
  |x_n-a|<\epsilon
\]
成立,那么就称常数$a$是数列$\{x_n\}$的极限,或者称数列$\{x_n\}$收敛于$a$,记为
\[
  \lim_{n\to\infty}x_n=a
\]
或
\[
  x_n \to a \quad (n\to\infty)
\]

\[
  \lim_{n\to\infty}x_n=a\iff\forall\epsilon>0,\; \exists N\in \mathbb{N},\; let\; when\; n>N,|x_n-a|<\epsilon
\]

\bigskip

\textbf{收敛数列的性质}
\begin{itemize}
\item 如果数列$\{x_n\}$收敛,那么它的极限唯一
\item 如果数列$\{x_n\}$收敛,那么数列$\{x_n\}$一定有界
\item 如果$\lim_{n\to\infty}x_n=a$且$a>0$(或$a<0$),那么存在正整数$N$,当$n>N$时都有$x_n>0$(或$x_n<0$)
\item 如果数列$\{x_n\}$从某项起有$x_n>0$(或$x_n<0$),且$\lim_{n\to\infty}x_n=a$,那么$a>0$(或$a<0$)
\item 如果数列$\{x_n\}$收敛于$a$,那么它的任一子数列也收敛,且极限也是$a$
\end{itemize}
\bigskip
\bigskip
\section*{函数的极限 The limit of function}

\bigskip

设函数$f(x)$在点$x_0$的某个去心邻域$\mathring{U}(x_0)$ (deleted neighborhood)内有定义,如果存在常数$A$,对于任意给定的正数$\epsilon$,
总存在正数$\delta$,使当$x$满足不等式$0<|x-x_0|<\delta$时,对应的函数值$f(x)$都满足不等式
\[
  |f(x)-A|<\epsilon
\]
那么常熟A就叫做函数$f(x)$当$x\to x_0$时的极限,记作
\[
  \lim_{x\to x_0}=0 \quad or \quad f(x)\to A \quad (x\to x_0)
\]

\[
  \lim_{x\to x_0}=A\iff\forall\epsilon>0,\;\exists\delta>0;\;let\;when\;0<|x-x_0|<\delta,|f(x)-A|<\epsilon
\]

左极限
\[
  \lim_{x\to x_0^-}=A\quad f(x_0^-)=A\iff\forall\epsilon>0,\;\exists\delta>0;\;let\;when\;x_0-\delta<x|<x_0,|f(x)-A|<\epsilon
\]

右极限
\[
  \lim_{x\to x_0^+}=A\quad f(x_0^+)=A\iff\forall\epsilon>0,\;\exists\delta>0;\;let\;when\;x_0<x|<x_0+\delta,|f(x)-A|<\epsilon
\]
左右极限统称 单侧极限

\bigskip

设函数$f(x)$当$|x|$大于某一正数时有定义,如果存在常数$A$,对于任意给定的正数$\epsilon$,
总存在着正数$X$使得当$x$满足不等式$|x|>X$时,对应的函数值$f(x)$都满足不等式
\[
  |f(x)-A|<\epsilon
\]
那么常数A就叫做函数$f(x)$当$x\to\infty$时的极限,记作
\[
  \lim_{x\to\infty}f(x)=A\quad or \quad f(x)\to A\quad(x\to\infty)
\]

\[
  \lim_{x\to\infty}=A\iff\forall\epsilon>0,\;\exists X>0;\;let\;when\;|x|>X,|f(x)-A|<\epsilon
\]
从几何上来说,$y=\lim_{x\to\infty}f(x)$是函数$f(x)$的一条水平渐近线

\[
  \lim_{x\to+\infty}=A\iff\forall\epsilon>0,\;\exists X>0;\;let\;when\;x>X,|f(x)-A|<\epsilon
\]
\[
  \lim_{x\to-\infty}=A\iff\forall\epsilon>0,\;\exists X>0;\;let\;when\;x<X,|f(x)-A|<\epsilon
\]

\bigskip

\textbf{函数极限的性质}
\begin{itemize}
  \item 如果$\lim_{x\to x_0}f(x)$存在,那么极限唯一
  \item 如果$\lim_{x\to x_0}f(x)=A$那么存在常数$M>0$和$\delta>0$,使得当$0<|x-x_0|<\delta$时,有$|f(x)|\le M$
  (函数极限的局部有界性)
  \item 如果$\lim_{x\to x_0}f(x)=A$,且$A>0$(或$A<0$),那么存在常数$\delta>0$,
  使得当$0<|x-x_0|<\delta$时,有$f(x)>0$(或$f(x)<0$)
  \item 如果$\lim_{x\to x_0}f(x)=A\quad(A\ne0)$,那么就存在着$x_0$的某一去心领域$\mathring{U}(x_0)$,
  当$x\in\mathring{U}(x_0)$时,就有$|f(x)>\frac{|A|}{2}|$
  \item 如果在$x_0$的某个去心领域内$f(x)\ge0$(或$f(x)\le0$),而且$\lim_{x\to x_0}f(x)=A$,那么$A\ge0$(或$A\le0$)
  \item 如果极限$\lim_{x\to x_0}f(x)$存在,${x_n}$为函数$f(x)$的定义域内任一收敛于$x_0$的数列,
  且满足:$x_n\ne x_0\quad (x\in\mathbb{N}^+)$,那么对应的函数值数列${f(x_n)}$必收敛,
  且$\lim_{n\to\infty}f(x_n)=\lim_{x\to x_0}f(x)$
\end{itemize}
\bigskip
\bigskip
\section*{无穷小与无穷大 Infinitesimal and Infinity}

\bigskip

如果函数$f(x)$当$x\to x_0$(或$x\to\infty$)时的极限为零,那么称函数$f(x)$为当$x\to x_0$(或$x\to\infty$)
时的无限小

\bigskip

在自变量的同一变化过程$x\to x_0$(或$x\to\infty$)中,函数$f(x)$具有极限$A$的充分必要条件是$f(x)=A+\alpha$,
其中$\alpha$是无限小

\bigskip

设函数$f(x)$在$x_0$的某一去心领域内有定义(或$|x|$大于某一正数时有定义),如果对于任意给定的正数$M$,总存在正数$\delta$(或正数$X$),
使得只要$x$满足不等式$0<|x-x_0|<\delta$(或$|x|>X$),对应的函数值$f(x)$总满足不等式
\[
  |f(x)|>M
\]
那么称函数$f(x)$是当$x\to x_0$(或$x\to\infty$)时的无穷大

按极限的定义来说,当$x\to x_0$(或$x\to\infty$)时的无穷大的函数$f(x)$的极限是不存在的,但为了便于叙述这一性态,
也称其为“函数的极限是无穷大”,并记为
\[
  \lim_{x\to x_0}f(x)=\infty\quad or\quad \lim_{x\to\infty}f(x)=\infty
\]
如果把定义中的$|f(x)|>M$换为$f(x)>M$(或$f(x)<M$)就记为
\[
    f(x)\to+\infty\quad or\quad f(x)\to-\infty
\]

\bigskip

在自变量的同一变化过程中,如果$f(x)$为无穷大,那么$\frac{1}{f(x)}$为无穷小;
反之,如果$f(x)$为无穷小,那么$\frac{1}{f(x)}$为无穷大
\bigskip
\bigskip
\section*{极限的运算法则 Limit algorithms}

\bigskip

\begin{itemize}
  \item 两个无穷小的和是无穷小$\to$(有限个无穷小的和也是无穷小)
  \item 有界函数和无穷小的乘积也是无穷小
  \begin{itemize}
    \item 常数与无穷小的乘积是无穷小
    \item 有限个无穷小的乘积是无穷小
  \end{itemize}
  \item 如果$\lim f(x)=A$,$\lim g(x)=B$,那么
  \begin{itemize}
    \item $\lim[f(x)\pm g(x)]=\lim f(x)\pm \lim g(x)=A\pm B$
    \item $\lim[f(x)\cdot g(x)]=\lim f(x)\cdot \lim g(x)=A\cdot B$
    \item if $B\ne0$, then
    \[
      \lim \frac{f(x)}{g(x)}=\frac{\lim f(x)}{\lim g(x)}=frac{A}{B}
    \]
  \end{itemize}
  推论(inference)
  \begin{itemize}
    \item 如果$\lim f(x)$存在,而$c$为常数(constant),那么
    \[ \lim[cf(x)]=c\lim(f(x)) \]
    \item 如果$\lim f(x)$存在,而$n\in\mathbb{N}^+$,那么
    \[ \lim[f(x)]^n = [\lim f(x)]^n\]
  \end{itemize}
  \item 如果有数列${x_n}$和${y_n}$,如果
  \[ \lim_{n\to\infty}x_n=A,\quad\lim_{n\to\infty}y_n=B \]
  那么
  \begin{itemize}
    \item $\lim_{n\to\infty}(x_n\pm y_n)=A\pm B$
    \item $\lim_{n\to\infty}(x_n\cdot y_n)=A\cdot B$
    \item if $y_n\ne0\quad(n\in\mathbb{N}^+)$ and $B\ne0$, then$\lim_{n\to\infty}\frac{x_n}{y_n}=\frac{A}{B}$
  \end{itemize}
  \item 如果$\varphi(x)\ge\psi(x)$,而$\lim\varphi(x)=A$,$\lim\psi(x)=B$,那么$A\ge B$
  \item 设函数$y=f[g(x)]$是由函数$u=g(x)$与函数$y=f(u)$复合而成,$y=f[g(x)]$在点$x_0$的某去心领域内有定义,
  若$\lim_{x\to x_0}g(x)=u_0$,$\lim_{u\to u_0}f(u)=A$,且存在$\delta_0>0$,当$x\in\mathring{U}(x_0,\delta_0)$时,
  有$g(x)\ne u_0$,则\[\lim_{x\to x_0}f[g(x)]=\lim_{u\to u_0}f(u)=A\]
\end{itemize}
\bigskip
\bigskip
\section*{极限存在的准则 Condition of Convergence}

\bigskip

\begin{itemize}
  \item[] 夹逼准则 (Squeeze Theorem)
  \begin{itemize}
    \item 如果数列${x_n}$,${y_n}$及${z_n}$满足:
    \begin{itemize}
      \item 从某项起,即$\exists n_0\in\mathbb{N}^+$,当$n>n_0$时,有
      \[ y_n\le x_n \le z_n \]
      \item $\lim_{n\to\infty}y_n=a,\quad\lim_{n\to\infty}x_n=a$
    \end{itemize}
    那么数列${x_n}$的极限存在,且$\lim_{n\to\infty}x_n=a$
    \item 如果
    \begin{itemize}
      \item 当$x\in\mathring{U}(x_0,r)$(或$|x|>M$)时,
      \[ g(x)\le f(x)\le h(x) \]
      \item $g(x)\to A,\quad h(x)\to A$
    \end{itemize}
    那么$\lim f(x)$存在,且等于$A$
  \end{itemize}
  \item[] 单调有界
  \begin{itemize}
    \item 如果数列${x_n}$满足条件
    \[ x_1\le x_2 \le x_3\le \cdots \le x_n\le x_{n+1}\le\cdots \]
    那么就称数列${x_n}$是单调增加的;如果数列${x_n}$满足条件
    \[ x_1\ge x_2 \ge x_3\ge \cdots \ge x_n\ge x_{n+1}\ge\cdots \]
    那么就称数列${x_n}$是单调减少的;单调增加数列和单调减少数列统称为单调数列

    单调有界数列必有极限
    \item 设函数$f(x)$在点$x_0$的某个左邻域内单调且有界,则$f(x)$在$x_0$的左极限$f(x_0^-)$必定存在
    \item 柯西 (Cauchy)极限存在准则 数列${x_n}$收敛的充分必要条件是:对于任意给定的正数$\epsilon$,
    存在正整数$N$,使得当$m>N,\quad n>N$时,有
    \[ |x_n-x_m|<\epsilon \]
  \end{itemize}
\end{itemize}
\bigskip
\bigskip
\section*{无穷小的比较 Competition of Infinitesimal}

\bigskip

\begin{tabular}{l}
  如果$\lim\frac{\beta}{\alpha}=0$,那么就说$\beta$是比$\alpha$高阶的无穷小 (infinitesimal of higher order),记作$\beta=o(\alpha)$\\
  如果$\lim\frac{\beta}{\alpha}=\infty$,那么就说$\beta$是比$\alpha$低阶的无穷小 (infinitesimal of lower order)\\
  如果$\lim\frac{\beta}{\alpha}=c\ne0$,那么就说$\beta$与$\alpha$是同阶无穷小 (Infinitesimal of the same order)\\
  如果$\lim\frac{\beta}{\alpha^k}=c\ne0,\quad k>0$,那么就说$\beta$是关于$\alpha$的$k$阶无穷小 (Infinitesimal of $k^\text{th}$ order)\\
  如果$\lim\frac{\beta}{\alpha}=1$,那么就说$\beta$与$\alpha$是等阶无穷小 (equivalent infinitesimal),记作$\alpha\sim\beta$
\end{tabular}

\bigskip

$\beta$与$\alpha$是等阶无穷小的充分必要条件为
\[ \beta=\alpha+o(\alpha) \]

\bigskip

设$\alpha\sim\tilde{\alpha}$,$\beta\sim\tilde{\beta}$且$\lim\frac{\tilde{\beta}}{\tilde{\alpha}}$存在,则
\[ \lim\frac{\beta}{\alpha}=\lim\frac{\tilde{\beta}}{\tilde{\alpha}} \]
\bigskip
\bigskip
\section*{函数的连续性与间断点 Continuity and Discontinuity point}

\bigskip

设函数$y=f(x)$在点$x_0$的某一邻域内有定义,如果
\[ \lim_{\Delta x\to0}\Delta y=\lim_{\Delta x\to0}[f(x_0+\Delta x)-f(x_0)]=0 \]
那么就称函数$y=f(x)$在点$x_0$连续 (continuous)

\bigskip

设函数$y=f(x)$在点$x_0$的某一邻域内有定义,如果
\[ \lim_{x\to x_0}f(x)=f(x_0) \]
那么就称函数$f(x)$在点$x_0$连续

\bigskip

左连续:$f(x_0^-)=f(x_0)$

右连续:$f(x_0^+)=f(x_0)$

\bigskip

在区间$I$ (interval)上每一点都连续的函数,叫做在区间$I$上的连续函数

\bigskip

\bigskip

设函数$f(x)$在点$x_0$的某个去心邻域内有定义,如果函数:
\begin{itemize}
  \item 在$x=x_0$没定义
  \item 在$x=x_0$有定义,但$\lim_{x\to x_0}f(x)$不存在
  \item 虽在$x=x_0$有定义,且$\lim_{x\to x_0}f(x)$存在,但$\lim_{x\to x_0}f(x)\ne f(x_0)$
\end{itemize}
那么函数$f(x)$在点$x_0$为不连续,且点$x_0$称为函数$f(x)$的不连续点或间断点 (discontinuity point)

\bigskip

\begin{itemize}
  \item 一类间断点 (左右极限都存在) discontinuity point of the first kind
  \begin{itemize}
    \item 可去间断点 (removable discontinuity):$f(x_0)$没有定义,$f(x_0)\to c$,$c$为一常数
    \item 跳跃间断点 (jump discontinuity):$f(x_0)$没有定义,$f(x_0^+)\ne f(x_0^-)$
  \end{itemize}
  \item 二类间断点:非一类间断点的间断点 discontinuity point of the second kind
  \begin{itemize}
    \item 无穷间断点 (infinite discontinuit):$f(x_0)$没有定义,$f(x_0)\to\infty$
    \item 振荡间断点 (oscillating discontinuit):$f(x_0)$没有定义,$\lim f(x_0)$没有定义 ($f(x)$在$x\to x_0$的过程中不断震荡)
  \end{itemize}
\end{itemize}
\bigskip
\bigskip
\section*{连续函数的运算与初等函数的连续性}

\bigskip

设函数$f(x)$和$g(x)$在点$x_0$连续,则它们的和(差)$f\pm g$、积$f\cdot g$及商$\frac{f}{g}$
(当$g(x_0)\ne0$时)都在点$x_0$连续

\bigskip

如果函数$y=f(x)$在区间$I_x$上单调增加(或单调减少)且连续,那么它的反函数$x=f^{-1}(y)$也在对应的区间$I_y=\{y|y=f(x),x\in I_x\}$
上单调增加(或单调减少)且连续

\bigskip

设函数$y=f[g(x)]$由函数$u=g(x)$与函数$y=f(u)$复合而成,$\mathring{U}(x_0)\subset D_{f\circ g}$;
若$\lim_{x\to x_0}=u_0$,而函数$y=f(u)$在$u=u_0$连续,则
\[ \lim_{x\to x_0}f[g(x)]=\lim_{u\to u_0}f(u)=f(u_0) \]

\bigskip

设函数$y=f[g(x)]$是由函数$u=g(x)$与函数$y=f(u)$复合而成,$U(x_0)\subset D_{f\circ g}$;
若函数$u=g(x)$在$x=x_0$连续,且$g(x_0)=u_0$,而函数$y=f(u)$在$u=u_0$连续,则复合函数$y=f[g(x)]$
在$x=x_0$也连续

\bigskip

\textbf{一切初等函数在其定义域内都是连续的}
\bigskip
\bigskip
\section*{闭区间上连续函数的性质 Properties of continuous functions on closed intervals}

\bigskip

如果函数$f(x)$在开区间$(a,b)$内连续,且在$b$左连续,在$a$右连续,那么函数$f(x)$在闭区间$[a,b]$上连续

\bigskip

对于在区间$I$上有定义的函数$f(x)$,如果由$x_0\in I$,使得对于任意$x\in I$都有
\[ f(x)\le f(x_0)\quad(f(x)\ge f(x_0)) \]
那么称$f(x_0)$是函数$f(x)$在区间$I$上的最大值(最小值)

\bigskip

在闭区间上连续的函数在该区间上有界且一定能取得它的最大值和最小值

\bigskip

零点定理:设函数$f(x)$在闭区间$[a,b]$上连续,且$f(a)$与$f(b)$异号(即$f(a)\cdot f(b)<0$),
则在开区间$(a,b)$内至少有一点$\xi$,使
\[f(\xi)=0\]

\bigskip

介值定理:设函数$f(x)$在闭区间$[a,b]$上连续,且在这区间端点取不同的函数值
\[ f(a)=A \quad and \quad f(b)=B \]
则对于$A$与$B$之间的任意一个数$C$,在开区间$(a,b)$内至少于一点$\xi$,使得
\[ f(\xi)=C\quad\xi\in(a,b) \]

\bigskip

在闭合区间$[a,b]$上连续的函数$f(x)$的值域为闭区间$[m,M]$,其中$m$与$M$依次为$f(x)$
在$[a,b]$上的最小值与最大值

\bigskip

\bigskip

设函数$f(x)$在区间$I$上有定义,如果对于任意给定的正数$\epsilon$,总存在正数$\delta$,
使得对于区间$I$上的任意两点$x_1$、$x_2$,当$|x_1-x_2|<\delta$时,有
\[ |f(x_1)-f(x_2)|<\epsilon \]
那么称函数$f(x)$在区间$I$上一致连续

\bigskip
\[y=frac{1}{x}\quad x\in(0,1]\]
\bigskip

如果函数$f(x)$在闭区间$[a,b]$上连续,那么它在该区间上一致连续
\end{document}

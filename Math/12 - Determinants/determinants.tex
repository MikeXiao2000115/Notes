\documentclass[UTF8]{ctexart}

\usepackage{geometry}

\usepackage{amsmath}
\usepackage{amssymb}
\usepackage{esint}
\usepackage{yhmath}
\usepackage{bm}

\usepackage{fancyhdr}
\usepackage{graphicx}

\special{papersize={18.1cm,25.7cm}}
\geometry{left=1.5cm,right=0.5cm,top=2cm,bottom=1cm}
\pagestyle{empty}

\newcommand{\D}{{\text{d}\;\!}}
\newcommand{\cross}{\times}
\newcommand{\dif}[1]{{\mathrm{d}\;\!#1}}
\newcommand{\dev}[1]{{\frac{\text{d}}{\dif{#1}}\;\!}}
\newcommand{\ve}[1]{{\bm{#1}}}
\newcommand{\ven}[2]{{\left\langle#1,#2\right\rangle}}
\newcommand{\veN}[3]{{\left\langle#1,#2,#3\right\rangle}}
\newcommand{\ang}[2]{{(\widehat{\ve{#1},\ve{#2}})}}
\newcommand{\abs}[1]{{\left|{#1}\right|}}
\newcommand{\when}[2]{{\left.{#1}\right|_{#2}}}
\newcommand{\dist}[2]{{\left\|\ve{#1}-\ve{#2}\right\|}}
\newcommand{\norm}[1]{{\left\|#1\right\|}}
\newcommand{\emplin}{\vspace{1em}}

\begin{document}

\section*{二阶与三阶行列式}
\subsection*{二阶行列式 Second-order determinant}
记号$\displaystyle \begin{vmatrix}a_{11}&a_{12}\\a_{21}&a_{22}\end{vmatrix}$表示代数和$a_{11}a_{22}-a_{12}a_{21}$,称为二阶行列式,即
\[\begin{vmatrix}a_{11}&a_{12}\\a_{21}&a_{22}\end{vmatrix}=a_{11}a_{22}-a_{12}a_{21}\]
其中数$a_{11}$,$a_{22}$,$a_{12}$,$a_{21}$称为行列式的元素,横排叫做行,竖排叫做列;
\textbf{元素$a_{ij}$的第一个下标$i$称为行标,表明该元素位于第$i$行,第二个下标$j$叫做列标,表明该元素位于第$j$列};
行列式是一个代数和

这个规则称为“对角线法则”,把从$a_{11}$到$a_{22}$的连线叫做主对角线,把$a_{12}$到$a_{21}$的连线交副对角线

\subsection*{二元线性方程}
对于二元线性方程
\[\left\{
\begin{aligned}
a_{11}x_1+a_{12}x_2=b_1\\
a_{21}x_1+a_{22}x_2=b_2
\end{aligned}
\right.\]
定义
\[D=\begin{vmatrix}
a_{11}&a_{12}\\
a_{21}&a_{22}
\end{vmatrix}
\]
\[
D_1=\begin{vmatrix}
b_1&a_{12}\\
b_2&a_{22}
\end{vmatrix}
\quad,\quad
D_2=\begin{vmatrix}
a_{11}&b_1\\
a_{21}&b_2
\end{vmatrix}
\]
则二元线性方程组可改写为
\[\left\{
\begin{aligned}
  Dx_1&=D_1\\
  Dx_2&=D_2
\end{aligned}
\right.
\]
于是,当$D\ne0$时,该二元线性方程有唯一解
\[x_1=\frac{D_1}{D}\quad x_2=\frac{D_2}{D}\]


\subsection*{三阶行列式 Third-order determinant}
记号$\displaystyle\begin{vmatrix}
a_{11}&a_{12}&a_{13}\\
a_{21}&a_{22}&a_{23}\\
a_{31}&a_{32}&a_{33}
\end{vmatrix}$表示代数和
\[a_{11}a_{22}a_{33}+a_{12}a_{23}a_{31}+a_{13}a_{21}a_{32}
-a_{13}a_{22}a_{31}-a_{12}a_{21}a_33-a{11}a_{23}a_{32}\]
称为三阶行列式

三阶行列式有6项,每一项均为不同行不同列的三个元素之积,在按一定法则冠以正负号

\subsection*{三元线性方程组}
类似于二元线性方程组,对于三元线性方程组
\[
\left\{
\begin{aligned}
  a_{11}x_1+a_{12}x_2+a_{13}x_3=b_1\\
  a_{21}x_1+a_{22}x_2+a_{23}x_3=b_2\\
  a_{31}x_1+a_{32}x_2+a_{33}x_3=b_3
\end{aligned}
\right.
\]
同时记
\[
D=
\begin{vmatrix}
  a_{11}&a_{12}&a_{13}\\
  a_{21}&a_{22}&a_{23}\\
  a_{31}&a_{32}&a_{33}
\end{vmatrix}
\quad,\quad
D_1=
\begin{vmatrix}
  b_1&a_{12}&a_{13}\\
  b_2&a_{22}&a_{23}\\
  b_3&a_{32}&a_{33}
\end{vmatrix}
\]
\[
D_2=
\begin{vmatrix}
  a_{11}&b_1&a_{13}\\
  a_{21}&b_2&a_{23}\\
  a_{31}&b_3&a_{33}
\end{vmatrix}
\quad,\quad
D_3=
\begin{vmatrix}
  b_1&a_{12}&b_1\\
  b_2&a_{22}&b_2\\
  b_3&a_{32}&b_3
\end{vmatrix}
\]
若系数行列式$D\ne0$,则该方程组有唯一解
\[x_1=\frac{D_1}{D}\quad x_2\frac{D_2}{D}\quad x_3=\frac{D_3}{D}\]


\section*{$n$阶行列式}
\subsection*{排列与逆序}
由自然数$1,2,\cdots,n$组成的不重复的每一种有确定次序的排列,称为$n$级排列(简称为排列)

\emplin

在一个$n$级排列($i_1i_2\cdots i_t\cdots i_s\cdots i_n$)中,若$i_t>i_s$,则称数$i_t$与数$i_s$构成一个逆序;一个$n$级排列中逆序的总数称为该排列的逆序数,
记为$N(i_1i_2\cdots i_n)$

即一个$n$级排列$i_1i_2\cdots i_n$中,设比$i_k$大的且排在$i_k$前的数共有$t_k$各,则$i_k$的逆序的个数为$t_k$,而该排列中所有自然数的逆序的个数之和就是这个排列的逆序数,即
\[N(i_1i_2\cdots i_n)=\sum_{k=1}^nt_k\]

\emplin

\textbf{逆序数为奇数的排列称为奇排列;逆序数为偶数的排列称为偶排列}

\subsection*{$n$阶行列式的定义}
有$n^2$各元素$a_{ij}$组成的记号
\[\begin{vmatrix}
a_{11}&a_{12}&\cdots&a_{1n}\\
a_{21}&a_{22}&\cdots&a_{2n}\\
\vdots&\vdots&\ddots&\vdots\\
a_{n1}&a_{n2}&\cdots&a_{nn}
\end{vmatrix}\]
称为$n$阶行列式,其中横排称为行,竖排称为列;它表示所有\textbf{取自不同行且不同列的$n$个元素的乘积$a_{1j_1}a_{2j_2}\cdots a_{nj_n}$}的代数和,
其各项的符号为:\textbf{当该项的各个元素的行标按自然数排列后($a_{1j_1}a_{2j_2}\cdots a_{nj_n}$),其对应的列标所构成的排列($j_1j_2\cdots j_n$)
是偶排列则符号取正,否则(是奇排列)则符号取负},即
\[\begin{vmatrix}
a_{11}&a_{12}&\cdots&a_{1n}\\
a_{21}&a_{22}&\cdots&a_{2n}\\
\vdots&\vdots&\ddots&\vdots\\
a_{n1}&a_{n2}&\cdots&a_{nn}
\end{vmatrix}=
\sum_{j_1j_2\cdots j_n}(-1)^{N(j_1j_2\cdots j_n)}a_{1j_1}a_{2j_2}\cdots a_{nj_n}
\]
其中$\displaystyle\sum_{j_1j_2\cdots j_n}$表示对所有$n$级排列$j_1j_2\cdots j_n$求和;行列式有时也简记为$\det(a_{ij})$或$\abs{a_{ij}}$,
这里数$a_{ij}$称为行列式的元素,称
\[(-1)^{N(j_1j_2\cdots j_n)}a_{1j_1}a_{2j_2}\cdots a_{nj_n}\]
为行列式的一般项


\end{document}

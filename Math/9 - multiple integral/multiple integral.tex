\documentclass[UTF8]{ctexart}

\usepackage{geometry}

\usepackage{amsmath}
\usepackage{amssymb}

\usepackage{bm}

\usepackage{fancyhdr}
\usepackage{graphicx}

\special{papersize={18.1cm,25.7cm}}
\geometry{left=1.5cm,right=0.5cm,top=2cm,bottom=1cm}
\pagestyle{empty}

\newcommand{\D}{{\text{d}\;\!}}
\newcommand{\cross}{\times}
\newcommand{\dif}[1]{{\text{d}\;\!#1}}
\newcommand{\dev}[1]{{\frac{\text{d}}{\dif{#1}}\;\!}}
\newcommand{\ve}[1]{{\bm{#1}}}
\newcommand{\ven}[2]{{\left\langle#1,#2\right\rangle}}
\newcommand{\veN}[3]{{\left\langle#1,#2,#3\right\rangle}}
\newcommand{\ang}[2]{{(\widehat{\ve{#1},\ve{#2}})}}
\newcommand{\abs}[1]{{\left|{#1}\right|}}
\newcommand{\when}[2]{{\left.{#1}\right|_{#2}}}
\newcommand{\dist}[2]{{\left\|\ve{#1}-\ve{#2}\right\|}}
\newcommand{\norm}[1]{{\left\|#1\right\|}}

\begin{document}

\section*{二重积分}
\subsection*{定义}
设$f(x,y)$是有界闭区域$D$上的有界函数,将闭区间$D$任意分成$n$个小区域
\[\Delta \sigma_1,\Delta \sigma_2,\cdots,\Delta \sigma_n\]
其中$\Delta \sigma_i$表示第$i$个小闭区域,也表示它的面积,在每个$\Delta \sigma_i$上任取一点
$(\xi_i,\eta_i)$,做乘积$f(\xi_i,\eta_i)\Delta \sigma_i$,并作和\\
$\sum^n_{i=1}f(\xi_i,\eta_i)\Delta \sigma_i$;
如果当各小闭区域的直径中的最大值$\lambda\to0$时,这和的极限总存在,且与比和区域$D$的分法及点$(\xi_i,\eta_i)$的取法无关,
那么称此极限为函数$f(x,y)$在闭合区域$D$上的二重积分,记作$\displaystyle\iint_Df(x,y)\dif{\sigma}$,即
\[\iint_Df(x,y)\dif{\sigma}=\lim_{\lambda\to0}f(\xi_i,\eta_i)\Delta \sigma_i\]
其中$f(x,y)$叫做被积函数,$f(x,y)\dif{\sigma}$叫做被积表达式,$\dif{\sigma}$叫做面积元素,
$x$与$y$叫做积分变量,$D$叫做积分区域,$\lim_{\lambda\to0}f(\xi_i,\eta_i)\Delta \sigma_i$叫做积分和

如果依照直角坐标系划分$\Delta \sigma_i$,则有$\dif{\sigma}=\dif{x}\dif{y}$,其中$\dif{x}\dif{y}$叫做直角坐标系中的面积元素

\subsection*{性质}
\begin{itemize}
  \item 设$\alpha$与$\beta$为常数,则
  \[ \iint_D[\alpha f(x,y)+\beta g(x,y)]\dif{\sigma}=\alpha\iint_Df(x,y)\dif{\sigma}+\beta\iint_Dg(x,y)\dif{\sigma} \]

  \item 如果闭区域$D$被有线条曲线划分为有限个部分闭区域,那么在$D$上的二重积分等于各部分区域上的二重积分的和

  \item 如果在$D$上,$f(x,y)\equiv0$,$\sigma$为$D$的面积,那么
  \[ \sigma=\iint_D1\dif{\sigma}=\iint_D\dif{\sigma} \]

  \item 如果在$D$上,$f(x,y)\le g(x,y)$,那么有
  \[\iint_Df(x,y)\dif{\sigma}\le\iint_Dg(x,y)\dif{\sigma}\]
  又有
  \[\abs{\iint_Df(x,y)\dif{\sigma}}\le\iint_D\abs{f(x,y)}\dif{\sigma}\]

  \item 设$M$和$m$分别时$f(x,y)$在闭区域$D$上的最大最小值,$\sigma$是$D$的面积,则有
  \[m\sigma\le\iint_Df(x,y)\dif{\sigma}\le M\sigma\]

  \item 设函数$f(x,y)$在闭区域$D$上连续,$\sigma$是$D$的面积,则在$D$上至少存在一点$(\xi,\eta)$,使得
  \[\iint_Df(x,y)\dif{\sigma}=f(\xi,\eta)\sigma\]
\end{itemize}

\subsection*{换元法}
设$f(x,y)$在$xOy$平面上的闭区域$D$上连续,若变换
\[T:x=x(u,v),y=y(u,v)\]
将$uOv$平面上的闭区域$D'$投影至$xOy$平面上的闭区域$D$,且满足
\begin{itemize}
  \item $x(u,v)$,$y(u,v)$在$D'$上具有一阶连续偏导数
  \item 在$D'$上雅可比式
  \[J(u,v)=\frac{\partial(x,y)}{\partial(u,v)}\ne0\]
  \item 变换$T:D'\to D$是一对一的
\end{itemize}
则有
\[\iint_Df(x,y)\dif{\sigma}=\iint_Df(x,y)\dif{x}\dif{y}=\iint_Df(x,y)\abs{J(u,v)}\dif{u}\dif{v}\]
\bigskip
\bigskip

\section*{三重积分}
\subsection*{定义}
设$f(x,y,z)$是空间有界闭区域$\Omega$上的有界函数,将闭区间$\Omega$任意分成$n$个小区域
\[\Delta v_1,\Delta v_2,\cdots,\Delta v_n\]
其中$\Delta v_i$表示第$i$个小闭区域,也表示它的体积,在每个$\Delta v_i$上任取一点
$(\xi_i,\eta_i,\zeta_i)$,做乘积$f(\xi_i,\eta_i,\zeta_i)\Delta v_i$,并作和$\sum^n_{i=1}f(\xi_i,\eta_i,\zeta_i)\Delta v_i$;
如果当各小闭区域的直径中的最大值$\lambda\to0$时,这和的极限总存在,且与比和区域$\Omega$的分法及点$(\xi_i,\eta_i,\zeta_i)$的取法无关,
那么称此极限为函数$f(x,y,z)$在闭合区域$\Omega$上的三重积分,记作$\displaystyle\iiint_\Omega f(x,y,z)\dif{v}$,即
\[\iiint_\Omega f(x,y,z)\dif{v}=\lim_{\lambda\to0}f(\xi_i,\eta_i,\zeta_i)\Delta v_i\]
其中$f(x,y,z)$叫做被积函数,$f(x,y,z)\dif{v}$叫做被积表达式,$\dif{v}$叫做体积元素,$\Omega$叫做积分区域

如果依照直角坐标系划分$\Delta v_i$,则有$\dif{v}=\dif{x}\dif{y}\dif{z}$,其中$\dif{x}\dif{y}\dif{z}$叫做直角坐标系中的体积元素

\[\dif{v}=\dif{x}\dif{y}\dif{z}=r\dif{\theta}\dif{r}\dif{z}=\rho^2\sin\varphi\dif{\rho}\dif{\varphi}\dif{\theta}\]
\bigskip
\bigskip

\section*{曲面面积}
\subsection*{曲面方程}
设曲面$S$由方程
\[z=f(x,y)\]
给出,$D$为曲面$S$在$xOy$上的投影,而其表面积为
\[A=\iint_D\sqrt{1+{\left(\frac{\partial z}{\partial x}\right)}^2+{\left(\frac{\partial z}{\partial y}\right)}^2}\dif{x}\dif{y}\]
\subsection*{参数方程}
若曲面$S$由参数方程
\[\begin{cases}
x=x(u,v)\\
y=y(u,v)\\
z=z(u,v)
\end{cases}\]
给出,即
\[G(u,v)=(x(u,v),y(u,v),z(u,v))\]
其中$(u,v)\in D$,则
\[\dif{A}=\norm{\frac{\partial G}{\partial u}\cross\frac{\partial G}{\partial v}}\dif{u}\dif{v}=\sqrt{(x_u^2+y_u^2+z_u^2)(x_v^2+y_v^2+z_v^2)-(x_ux_v+y_uy_v+z_uz_v)^2}\dif{u}\dif{v}\]
\bigskip
\bigskip

\section*{含参变量的积分}
如果函数$f(x,y)$在矩形$R=[a,b]\cross[c,d]$上连续,那么由积分$\varphi(x)=\int_c^df(x,y)\dif{y}$确定的函数
$\varphi(x)$在$[a,b]$上也连续

\bigskip

如果函数$f(x,y)$在矩形$R=[a,b]\cross[c,d]$上连续,那么$\int_a^b[\int_c^df(x,y)\dif{y}]\dif{x}=\int_c^d[\int_a^bf(x,y)\dif{x}]\dif{y}$

\bigskip

如果函数$f(x,y)$及其偏导数$f_x(x,y)$在矩形$R=[a,b]\cross[c,d]$上连续,那么由$\varphi(x)=\int_c^df(x,y)\dif{y}$确定的函数
$\varphi(x)$在$[a,b]$上可微,且
\[ \varphi'(x)=\dev{x}\int_c^df(x,y)\dif{y}=\int_c^df_x(x,y)\dif{y} \]

\bigskip

如果函数$f(x,y)$在矩形$R=[a,b]\cross[c,d]$上连续,函数$\alpha(x)$与$\beta(x)$在区间$[a,b]$上连续,且
\[ c\le\alpha(x)\le d,c\le\beta(x)\le d \]
那么由积分$\displaystyle\Phi(x)=\int^{\beta(x)}_{\alpha(x)}f(x,y)\dif{y}$确定的函数$\Phi(x)$在$[a,b]$上也连续

\bigskip

如果函数$f(x,y)$及其偏导数$f_x(x,y)$都在矩形$R=[a,b]\cross[c,d]$上连续,函数$\alpha(x)$与$\beta(x)$在区间$[a,b]$上可微,且
\[ c\le\alpha(x)\le d,c\le\beta(x)\le d \]
那么由积分$\Phi(x)=\int^{\beta(x)}_{\alpha(x)}f(x,y)\dif{y}$确定的函数$\Phi(x)$在$[a,b]$上也可微,且
\[
\begin{aligned}
\Phi'(x)&=\dev{x}\int^{\beta(x)}_{\alpha(x)}f(x,y)\dif{y}\\
&=\int^{\beta(x)}_{\alpha(x)}f_x(x,y)\dif{y}+f[x,\beta(x)]\beta'(x)-f[x,\alpha(x)]\alpha'(x)
\end{aligned}
\]

\end{document}

\documentclass[UTF8]{ctexart}

\usepackage{geometry}

\usepackage{amsmath}
\usepackage{amssymb}

\usepackage{fancyhdr}
\usepackage{graphicx}

\special{papersize={18.1cm,25.7cm}}
\geometry{left=1.5cm,right=0.5cm,top=2cm,bottom=1cm}
\pagestyle{empty}

\newcommand{\D}{\text{d}}

\begin{document}



\section*{微分中值定理}

\bigskip

费马(Fermat)引理:设函数$f(x)$在点$x_0$的某邻域$U(x_0)$内有定义,并且在$x_0$处可导,
如果对于任意的$x\in U(x_0)$,有
\[ f(x)\le f(x_0)\quad or \quad f(x)\ge f(x_0) \]
那么$f'(x_0)=0$

\bigskip

\bigskip

罗尔(Rolle)定理:如果函数$f(x)$满足
\begin{itemize}
  \item 在闭区间$[a,b]$上连续
  \item 在开区间$(a,b)$内可导
  \item 在区间端点处的函数值相等,$f(a)=f(b)$
\end{itemize}
那么在$(a,b)$内至少有一点$\xi\quad(a<\xi<b)$,使得$f'(\xi)=0$

\bigskip

\bigskip

拉格朗日中值定理(Lagrange Median Theorem):如果函数$f(x)$满足
\begin{itemize}
  \item 在闭区间$[a,b]$上连续
  \item 在开区间$(a,b)$内可导
\end{itemize}
那么在$(a,b)$内至少有一点$\xi\quad(a<\xi<b)$,使等式
\[ f(b)-f(a)=f'(\xi)(b-a) \]
成立

\bigskip

\bigskip

如果函数$f(x)$在区间$I$上连续,$I$内可导且导数恒为零,那么$f(x)$在区间$I$上是一个常数

\bigskip

\bigskip

柯西(Cauchy)中值定理:如果函数$f(x)$及$g(x)$满足
\begin{itemize}
  \item 在闭区间$[a,b]$上连续
  \item 在开区间$(a,b)$内可导
  \item 对任一$x\in(a,b)$,$g'(x)\ne0$
\end{itemize}
那么在$(a,b)$内至少有一点$\xi$,使等式
\[ \frac{f(b)-f(a)}{g(b)-g(a)}=\frac{f'(\xi)}{g'(\xi)} \]
成立
\bigskip
\bigskip

\section*{洛必达法则 L'Hôpital's rule}

\bigskip

设
\begin{itemize}
  \item 当$x\to a$时,函数$f(x)$及$g(x)$都趋向于零
  \item 在点$a$的某个去心邻域内,$f'(x)$及$g'(x)$都存在,且$g'(x)\ne0$
  \item $\lim_{x\to a}\frac{f'(x)}{g'(x)}$存在(或为无限大)
\end{itemize}
则
\[ \lim_{x\to a}\frac{f(x)}{g(x)}=\lim_{x\to a}\frac{f'(x)}{g'(x)} \]

\bigskip

设
\begin{itemize}
  \item 当$x\to\infty$时,函数$f(x)$及$g(x)$都趋向于零
  \item 当$|x|>N$时$f'(x)$及$g'(x)$都存在,且$g'(x)\ne0$
  \item $\lim_{x\to a}\frac{f'(x)}{g'(x)}$存在(或为无限大)
\end{itemize}
则
\[ \lim_{x\to\infty}\frac{f(x)}{g(x)}=\lim_{x\to\infty}\frac{f'(x)}{g'(x)} \]

\bigskip
\bigskip

\section*{泰勒公式 Taylor's Formula}

\bigskip

泰勒中值定理1:如果函数$f(x)$在$x_0$处具有$n$阶导数,那么存在$x_0$的一个邻域,对于该邻域内的任一$x$,
有
\[ f(x)=f(x_0)+f'(x_0)(x-x_0)+\cdots+\frac{f^{(n)}(x_0)}{n!}(x-x_0)^n+R_n(x) \]
其中
\[ R_n(x)=o((x-x_0)^n) \]

\bigskip

泰勒中值定理2:如果函数$f(x)$在$x_0$的摸个邻域$U(x_0)$内具有$(n+1)$阶导数,那么对于任意$x\in U(x_0)$,
有
\[ f(x)=f(x_0)+f'(x_0)(x-x_0)+\cdots+\frac{f^{(n)}(x_0)}{n!}(x-x_0)^n+R_n(x) \]
其中
\[ R_n(x)=\frac{f^{(n+1)}(\xi)}{(n+1)!}(x-x_0)^{n+1} \]
这里$\xi$是$x_0$与$x$之间的某个值($R_n$被称为拉格朗日余项 Lagrange form)

\bigskip

在泰勒公式中将$x_0$取为0,则有麦克劳林(Maclaurin)公式

\bigskip
\bigskip

\section*{函数的单调性与曲线的凹凸性}

\bigskip

设函数$y=f(x)$在$[a,b]$上连续,在$(a,b)$内可导
\begin{itemize}
  \item 如果在$(a,b)$内$f'(x)\ge0$,且等号仅在有限点成立,那么函数在$[a,b]$上单调增加
  \item 如果在$(a,b)$内$f'(x)\le0$,且等号仅在有限点成立,那么函数在$[a,b]$上单调减少
\end{itemize}

\bigskip

设$f(x)$在区间$I$上连续,如果对$I$上任意两点$x_1$,$x_2$恒有
\[ f(\frac{x_1+x_2}{x})<\frac{f(x_1)+f(x_2)}{2} \]
那么称$f(x)$在$I$上的图形是凹的 (concave up)

如果恒有
\[ f(\frac{x_1+x_2}{x})>\frac{f(x_1)+f(x_2)}{2} \]
那么称$f(x)$在$I$上的图形是凹的 (concave down)

\bigskip

设$f(x)$在$[a,b]$上连续,在$(a,b)$内具有一阶和二阶导数,那么
\begin{itemize}
  \item 若在$(a,b)$内$f''(x)>0$,则$f(x)$在$[a,b]$上是凹的
  \item 若在$(a,b)$内$f''(x)<0$,则$f(x)$在$[a,b]$上是凸的
\end{itemize}

\bigskip

如果函数在通过点$(x_0,f(x_0))$时,凹凸性发生了改变,那么就称这一点为曲线的拐点 point of inflection (POI)

\bigskip
\bigskip

\section*{函数的单调性与曲线的凹凸性}

\bigskip

设函数$f(x)$在点$x_0$的某个邻域$U(x_0)$内有定义,如果对于去心邻域$\mathring{U}(x_0)$内任意$x$,有
\[ f(x)<f(x_0)\quad or \quad f(x)>f(x_0) \]
那么就称$f(x_0)$时函数$f(x)$的一个极大值(或极小值)

\bigskip

设函数$f(x)$在$x_0$处可导,且在$x_0$处取得极值,则$f'(x_0)=0$

\bigskip

设函数$f(x)$在点$x_0$处连续,且在$x_0$的某去心邻域$\mathring{U}(x_0,\delta)$内可导
\begin{itemize}
  \item 若$x\in(x_0-\delta,x_0)$时,$f'(x)>0$,而$x\in(x_0,x_0+\delta)$时,$f'(x)<0$,则$f(x)$在$x_0$处取得极大值
  \item 若$x\in(x_0-\delta,x_0)$时,$f'(x)<0$,而$x\in(x_0,x_0+\delta)$时,$f'(x)>0$,则$f(x)$在$x_0$处取得极小值
\end{itemize}

\bigskip

设函数$f(x)$在$x_0$处具有二阶导数且$f'(x_0)=0$,$f''(x_0)\ne0$,则
\begin{itemize}
  \item 当$f''(x_0)<0$时,函数$f(x)$在$x_0$处取得极大值
  \item 当$f''(x_0)>0$时,函数$f(x)$在$x_0$处取得极小值
\end{itemize}

\bigskip
\bigskip

\section*{曲率 curvature}

\bigskip

\textbf{弧微分$\D s=\sqrt{1+y'^2}\D x$}

曲率$ \displaystyle\kappa
=\lim_{\Delta s\to0}\big|\frac{\Delta\alpha}{\Delta s}\big|
=\big|\frac{\D\alpha}{\D s}\big|
=\frac{|y''|}{(1+y'^2)^{3/2}}
=\frac{\|\vec{r'}\times\vec{r''}\|}{\|\vec{r'}\|^3}$

曲率半径$\rho$与曲率$\kappa$互为倒数($\rho\kappa=1$)

当点$(x,f(x))$沿曲线$C$移动式,相应的曲率中心$D$的轨迹曲线$G$称为曲线$C$的渐屈线,
而曲线$C$称为曲线$G$的渐伸线,其方程为
$\displaystyle\begin{cases}
\alpha=x-\frac{(1+y'^2)}{y''}y'\\
\beta=y+\frac{1+y'^2}{y''}
\end{cases}$
\bigskip
\bigskip

\section*{牛顿-拉弗森方法(牛顿迭代法) Newton-Raphson method}

\bigskip

寻找一点$x$使得$f(x)=0$,以任意$x_0$为起点进行迭代
\[ x_{n+1}=x_n+\frac{f(x_n)}{f'(x_n)} \]

\end{document}

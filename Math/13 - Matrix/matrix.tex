\documentclass[UTF8]{ctexart}

\usepackage{geometry}

\usepackage{amsmath}
\usepackage{amssymb}
\usepackage{esint}
\usepackage{yhmath}
\usepackage{bm}

\usepackage{fancyhdr}
\usepackage{graphicx}

\special{papersize={18.1cm,25.7cm}}
\geometry{left=1.5cm,right=0.5cm,top=2cm,bottom=1cm}
\pagestyle{empty}

\newcommand{\D}{{\text{d}\;\!}}
\newcommand{\cross}{\times}
\newcommand{\dif}[1]{{\mathrm{d}\;\!#1}}
\newcommand{\dev}[1]{{\frac{\text{d}}{\dif{#1}}\;\!}}
\newcommand{\ve}[1]{{\bm{#1}}}
\newcommand{\mat}[1]{\ve{#1}}
\newcommand{\ven}[2]{{\left\langle#1,#2\right\rangle}}
\newcommand{\veN}[3]{{\left\langle#1,#2,#3\right\rangle}}
\newcommand{\ang}[2]{{(\widehat{\ve{#1},\ve{#2}})}}
\newcommand{\abs}[1]{{\left|{#1}\right|}}
\newcommand{\when}[2]{{\left.{#1}\right|_{#2}}}
\newcommand{\dist}[2]{{\left\|\ve{#1}-\ve{#2}\right\|}}
\newcommand{\norm}[1]{{\left\|#1\right\|}}
\newcommand{\emplin}{\vspace{1em}}

\begin{document}

\section*{矩阵的概念}
由$m\times n$个数$a_{ij}$排成的$m$行$n$列的数表
\[\begin{matrix}
a_{11}&a_{12}&\cdots&a_{1n}\\
a_{21}&a_{22}&\cdots&a_{2n}\\
\vdots&\vdots&\ddots&\vdots\\
a_{m1}&a_{m2}&\cdots&a_{mn}
\end{matrix}\]
称为$m$行$n$列矩阵,简称$m\times n$矩阵,为表示一个整体,总是加一个括弧,并用大写黑体字母表示它,记为
\[\mat{A}=
\begin{bmatrix}
a_{11}&a_{12}&\cdots&a_{1n}\\
a_{21}&a_{22}&\cdots&a_{2n}\\
\vdots&\vdots&\ddots&\vdots\\
a_{m1}&a_{m2}&\cdots&a_{mn}
\end{bmatrix}\]
这$m\times n$个数称为矩阵$\mat{A}$的元素,$a_{ij}$称为矩阵$\mat{A}$的第$i$行第$j$列的元素,一个$m\times n$矩阵$\mat{A}$也可简记为
\[\mat{A}=\mat{A}_{m\times n}=(a_{ij})_{m\times n}\text{ or }\mat{A}=(a_{ij})\]

\emplin

如果矩阵$\mat{A}$与$\mat{B}$的行数及列数均相同,且对应元素相等,则称矩阵$\mat{A}$与矩阵$\mat{B}$相等,记为$\mat{A}=\mat{B}$

\subsection*{几种特殊矩阵}
\begin{itemize}
  \item \textbf{实矩阵}

  元素均为实数的矩阵
  \item \textbf{复矩阵}

  元素为复数的矩阵
  \item \textbf{非负矩阵}

  元素均为非负数的矩阵
  \item \textbf{$n$阶方阵}

  若矩阵$\mat{A}$的行数与列数都等于$n$,则称$\mat{A}$为$n$阶方阵,记为$\mat{A}_n$
  \item \textbf{同型矩阵}

  如果两个矩阵具有相同的行数与相同的列数,则称这两个矩阵为同型矩阵
  \item \textbf{零矩阵}

  所有元素均为零的矩阵,记为$\mat{O}$
  \item \textbf{$n$阶单位矩阵}

  $n$阶方阵$\displaystyle \begin{bmatrix}
  1&0&\cdots&0\\
  0&1&\cdots&0\\
  \vdots&\vdots&\ddots&\vdots\\
  0&0&\cdots&1
  \end{bmatrix}$,记为
  $\mat{E}=\mat{E}_n\text{ or }\mat{I}=\mat{I}_n$
  \item \textbf{行矩阵\&行向量}

  只有一行的矩阵$\displaystyle\begin{bmatrix}a_1&a_2&\cdots&a_n\end{bmatrix}$,
  为避免元素混淆,也记为$\displaystyle\begin{bmatrix}a_1,&a_2,&\cdots,&a_n\end{bmatrix}$
  \item \textbf{列矩阵\&列向量}

  只有一列的矩阵$\displaystyle\begin{bmatrix}b_1\\b_2\\\vdots\\b_m\end{bmatrix}$
  \item \textbf{$n$阶对角矩阵}

  $n$阶方阵$\displaystyle \begin{bmatrix}
  \lambda_1&0&\cdots&0\\
  0&\lambda_2&\cdots&0\\
  \vdots&\vdots&\ddots&\vdots\\
  0&0&\cdots&\lambda_n
  \end{bmatrix}$,也可记为
  $\mat{A}=\text{diag}(\lambda_1,\lambda_2,\cdots,\lambda_n)$
  \item \textbf{$n$阶数量矩阵}

  当一个$n$阶对角矩阵$\mat{A}$的对角元素全部等于某一数$a$是,即$\mat{A}=\text{diag}(a,a,\cdots,a)=a\mat{I}$
\end{itemize}

\section*{矩阵的运算}
\begin{itemize}
  \item \textbf{取负}

  设矩阵$\mat{A}=(a_{ij})$,记$-\mat{A}=(-a_{ij})$,称$-\mat{A}$为矩阵$\mat{A}$的\textbf{负矩阵}
  \item \textbf{加法}

  设有两$m\times n$的同型矩阵$\mat{A}=(a_{ij})$和$\mat{B}=(b_{ij})$,矩阵$\mat{A}$与$\mat{B}$的和记作$\mat{A}+\mat{B}$,规定为
  \[\mat{A}+\mat{B}=(a_{ij}+b_{ij})=\begin{bmatrix}
  a_{11}+b_{11}&a_{12}+b_{12}&\cdots&a_{1n}+b_{1n}\\
  a_{21}+b_{21}&a_{22}+b_{22}&\cdots&a_{2n}+b_{2n}\\
  \vdots&\vdots&\ddots&\vdots\\
  a_{m1}+b_{m1}&a_{m2}+b_{m2}&\cdots&a_{mn}+b_{mn}
  \end{bmatrix}\]

  \item \textbf{减法}

  由于$\mat{A}+(-\mat{A})=\mat{O}$,则可定义减法$\mat{A}-\mat{B}=\mat{A}+(-\mat{B})$

  \item \textbf{数乘运算}

  数$k$与$m\times n$矩阵$\mat{A}$的乘积记作$k\mat{A}$或$\mat{A}k$,定义为
  \[k\mat{A}=\mat{A}k=(ka_{ij})=\begin{bmatrix}
  ka_{11}&ka_{12}&\cdots&ka_{1n}\\
  ka_{21}&ka_{22}&\cdots&ka_{2n}\\
  \vdots&\vdots&\ddots&\vdots\\
  ka_{m1}&ka_{m2}&\cdots&ka_{mn}
  \end{bmatrix}\]

  \item \textbf{线性运算}

  矩阵的加法与数乘两种运算统称为矩阵的线性运算,它们满足规律
  \begin{itemize}
    \item[>] $\displaystyle \mat{A}+\mat{B}=\mat{B}+\mat{A}$
    \item[>] $\displaystyle (\mat{A}+\mat{B})+\mat{C}=\mat{A}+(\mat{B}+\mat{C})$
    \item[>] $\displaystyle \mat{A}+\mat{O}=\mat{A}$
    \item[>] $\displaystyle \mat{A}+(-\mat{A})=\mat{O}$
    \item[>] $\displaystyle 1\mat{A}=\mat{A}$
    \item[>] $\displaystyle k(l\mat{A})=(kl)\mat{A}$
    \item[>] $\displaystyle (k+l)\mat{A}=k\mat{A}+l\mat{A}$
    \item[>] $\displaystyle k(\mat{A}+\mat{B})=k\mat{A}+k\mat{B}$
  \end{itemize}

  \item \textbf{乘法}

  设
  \[\mat{A}=(a_{ij})_{m\times s}=\begin{bmatrix}
  a_{11}&a_{12}&\cdots&a_{1s}\\
  a_{21}&a_{22}&\cdots&a_{2s}\\
  \vdots&\vdots&\ddots&\vdots\\
  a_{m1}&a_{m2}&\cdots&a_{ms}
  \end{bmatrix}\quad
  \mat{B}=(b_{ij})_{s\times n}=\begin{bmatrix}
  b_{11}&b_{12}&\cdots&b_{1n}\\
  b_{21}&b_{22}&\cdots&b_{2n}\\
  \vdots&\vdots&\ddots&\vdots\\
  b_{s1}&b_{s2}&\cdots&b_{sn}
  \end{bmatrix}\]
  矩阵$\mat{A}$与矩阵$\mat{B}$的乘积记作$\mat{A}\mat{B}$,定义为
  \[\mat{A}\mat{B}=(c_{ij})_{m\times n}=\begin{bmatrix}
  c_{11}&c_{12}&\cdots&c_{1n}\\
  c_{21}&c_{22}&\cdots&c_{2n}\\
  \vdots&\vdots&\ddots&\vdots\\
  c_{m1}&c_{m2}&\cdots&c_{mn}
  \end{bmatrix}\quad\left(
  c_{ij}=\sum_{k=1}^sa_{ik}b_{kj}
  \right)\]
  若$\mat{C}=\mat{A}\mat{B}$,则矩阵$\mat{C}$的元素$c_{ij}$即为矩阵$\mat{A}$第$i$行元素与矩阵$\mat{B}$第$j$行元素对应元素乘积之和,即
  \[c_{ij}=
  \begin{bmatrix}
    a_{i1} & a_{i2} & \cdots & a_{is}
  \end{bmatrix}
  \begin{bmatrix}
    b_{1j}\\
    b_{2j}\\
    \vdots\\
    b_{sj}
  \end{bmatrix}
  =a_{i1}b_{1j}+a_{i2}b_{2j}+\cdots+a_{is}b_{sj}
  =\sum_{k=1}^sa_{ik}b_{kj}
  \]

  \emplin

  显然$\mat{A}\mat{B}\ne\mat{B}\mat{A}$(有时两者中只有一个有定义)

  两非零矩阵的乘积可能为零矩阵,故不能从$\mat{A}\mat{B}=\mat{O}$得出$\mat{A}\text{ or }\mat{B}=\mat{O}$

  矩阵乘法一般也不满足消去律,即不能从$\mat{A}\mat{C}=\mat{B}\mat{C}$得出$\mat{A}=\mat{B}$

  \emplin

  矩阵乘法满足运算规则(若有定义)
  \begin{itemize}
    \item[>] $\displaystyle (\mat{A}\mat{B})\mat{C}=\mat{A}(\mat{B}\mat{C})$
    \item[>] $\displaystyle (\mat{A}+\mat{B})\mat{C}=\mat{A}\mat{C}+\mat{B}\mat{C}$
    \item[>] $\displaystyle \mat{C}(\mat{A}+\mat{B})=\mat{C}\mat{A}+\mat{C}\mat{B}$
    \item[>] $\displaystyle k(\mat{A}\mat{B})=(k\mat{A})\mat{B}=\mat{A}(k\mat{B})$
  \end{itemize}

  \item \textbf{可交换}

  如果两矩阵相乘,有$\mat{A}\mat{B}=\mat{B}\mat{A}$,则称矩阵$\mat{A}$与矩阵$\mat{B}$可交换,简称$\mat{A}$与$\mat{B}$可换

  (对于单位矩阵有$\mat{I}_m\mat{A}_{m\times n}=\mat{A}_{m\times n}\mat{I}_n=\mat{A}_{m\times n}$)

  \item \textbf{转置}

  把矩阵$\mat{A}$的行换成同序数的列所得到的新矩阵称为$\mat{A}$的转置矩阵,记作$\mat{A}^T\text{ or }\mat{A}'$

  即若$\displaystyle\mat{A}=\begin{bmatrix}
  a_{11}&a_{12}&\cdots&a_{1n}\\
  a_{21}&a_{22}&\cdots&a_{2n}\\
  \vdots&\vdots&\ddots&\vdots\\
  a_{m1}&a_{m2}&\cdots&a_{mn}
  \end{bmatrix}$,则$\displaystyle\mat{A}^T=\begin{bmatrix}
  a_{11}&a_{21}&\cdots&a_{m1}\\
  a_{12}&a_{22}&\cdots&a_{m2}\\
  \vdots&\vdots&\ddots&\vdots\\
  a_{1n}&a_{2n}&\cdots&a_{mn}
  \end{bmatrix}$

  \emplin

  矩阵的转置满足运算规则(若有定义)
  \begin{itemize}
    \item[>] $\displaystyle (\mat{A}^T)^T=\mat{A}$
    \item[>] $\displaystyle (\mat{A}+\mat{B})^T=\mat{A}^T+\mat{B}^T$
    \item[>] $\displaystyle (k\mat{A})^T=k\mat{A}^t$
    \item[>] $\displaystyle (\mat{A}\mat{B})^T=\mat{B}^T\mat{A}^T$
  \end{itemize}

  \item \textbf{方阵的幂}

  设方阵$\mat{A}=(a_{ij})_{n\times n}$,规定
  \[\mat{A}^0=\mat{I}\text{, }\mat{A}^k=\overbrace{\mat{A}\cdot\mat{A}\cdot\cdots\cdot\mat{A}}^{k\text{ of }\mat{A}}\]

  \emplin

  矩阵的幂满足运算规则
  \begin{itemize}
    \item[>] $\displaystyle \mat{A}^m\mat{A}^n=\mat{A}^{m+n}$
    \item[>] $\displaystyle (\mat{A}^m)^n=\mat{A}^{mn}$
  \end{itemize}

  \item \textbf{方阵的行列式}

  由$n$阶方阵$\mat{A}$的元素所构成的行列式(各元素的位置不变),称为方阵$\mat{A}$的行列式,记作
  \[\abs{\mat{A}}\text{ or }\det\mat{A}\]

  \emplin

  矩阵$\mat{A}$的行列式$\det\mat{A}$满足运算规则(其中$\mat{A}$与$\mat{B}$同为$n$阶方阵)
  \begin{itemize}
    \item[>] $\displaystyle \det\mat{A}^T=\det\mat{A}$
    \item[>] $\displaystyle \det(k\mat{A})=k^n\det\mat{A}$
    \item[>] $\displaystyle \det(\mat{A}\mat{B})=\det\mat{A}\det\mat{B}$
    \item[>] $\displaystyle \det(\mat{A}\mat{B})=\det(\mat{B}\mat{A})$
  \end{itemize}

  \item \textbf{对称矩阵}

  设$\mat{A}$为$n$阶方阵,如果$\mat{A}^T=\mat{A}$,即$a_{ij}=a_{ji}$,则称$\mat{A}$为对称矩阵

  \item \textbf{反对称矩阵}

  设$\mat{A}$为$n$阶方阵,如果$\mat{A}^T=-\mat{A}$,即$a_{ij}=-a_{ji}$,则称$\mat{A}$为反对称矩阵

  \item \textbf{共轭矩阵}

  设$\mat{A}=(a_{ij})$为复矩阵,记$\overline{\mat{A}}=(\overline{a_{ij}})$,其中$\overline{a_{ij}}$为$a_{ij}$的共轭复数,
  称$\overline{\mat{A}}$为$\mat{A}$的共轭矩阵

  \emplin

  \begin{itemize}
    \item[>] $\displaystyle \overline{\mat{A}+\mat{B}}=\overline{\mat{A}}+\overline{\mat{B}}$
    \item[>] $\displaystyle \overline{\lambda\mat{A}}=\overline{\lambda}\overline{\mat{A}}$
    \item[>] $\displaystyle \overline{\mat{A}\mat{B}}=\overline{\mat{A}}\overline{\mat{B}}$
    \item[>] $\displaystyle \overline{(\mat{A}^T)}=(\overline{\mat{A}})^T$
  \end{itemize}
\end{itemize}

\subsection*{逆矩阵}

对于一个$n$阶方阵$\mat{A}$,如果存在一个$n$阶方阵$\mat{B}$,使得$\mat{A}\mat{B}=\mat{B}\mat{A}=\mat{I}$,则称方阵$\mat{A}$为可逆矩阵,
而方阵$\mat{B}$称为$\mat{A}$的逆矩阵

\textbf{若矩阵$\mat{A}$是可逆的,则$\mat{A}$的逆矩阵是唯一的,记为$\mat{A}^{-1}$}

\emplin

如果$n$阶方阵$\mat{A}$的行列式$\det\mat{A}\ne0$,则称$\mat{A}$为非奇异的,否则称$\mat{A}$为奇异的




\section*{线性方程组的矩阵表示}
对于线性方程组
\[\left\{
\begin{aligned}
a_{11}x_1+a_{12}x_2+\cdots+a_{1n}x_n&=b_1\\
a_{21}x_1+a_{22}x_2+\cdots+a_{2n}x_n&=b_2\\
\cdots\cdots\cdots\cdots\cdots\cdots\cdots\cdots\cdots&\cdots\cdots\\
a_{m1}x_1+a_{m2}x_2+\cdots+a_{mn}x_n&=b_m
\end{aligned}
\right.\]
若记
$\displaystyle\mat{A}=\begin{bmatrix}
a_{11}&a_{12}&\cdots&a_{1n}\\
a_{21}&a_{22}&\cdots&a_{2n}\\
\vdots&\vdots&\ddots&\vdots\\
a_{m1}&a_{m2}&\cdots&a_{mn}
\end{bmatrix}$,
$\displaystyle\mat{x}=\begin{bmatrix}
x_1\\
x_2\\
\vdots\\
x_n
\end{bmatrix}$,
$\displaystyle\mat{b}=\begin{bmatrix}
b_1\\
b_2\\
\vdots\\
b_m
\end{bmatrix}$,
则利用矩阵的乘法,线性方程组可表示为
\[\mat{A}\mat{x}=\mat{b}\]
其中$\mat{A}$称为方程组的系数矩阵,方程$\mat{A}\mat{x}=\mat{b}$称为矩阵方程

如果$x_j=c_j$是方程组的解,记列矩阵$\mat{\eta}=\begin{bmatrix}
c_1\\
c_2\\
\vdots\\
c_n
\end{bmatrix}$,则$\mat{A}\mat{\eta}=\mat{b}$这是也称$\mat{\eta}$是矩阵方程的解;反之如果$\mat{\eta}$是矩阵方程的解,
既有矩阵等式$\mat{A}\mat{\eta}=\mat{b}$成立,则$\mat{x}=\mat{\eta}$,即$x_j=c_j$也是线性方程组的解

\section*{线性变换}
变量$x_1,x_2,\cdots,x_n$与变量$y_1,y_2,\cdots,y_m$之间的关系式
\[\left\{
\begin{aligned}
y_1&=a_{11}x_1+a_{12}x_2+\cdots+a_{1n}x_n\\
y_2&=a_{21}x_1+a_{22}x_2+\cdots+a_{2n}x_n\\
\cdots&\cdots\cdots\cdots\cdots\cdots\cdots\cdots\cdots\cdots\cdots\\
y_m&=a_{m1}x_1+a_{m2}x_2+\cdots+a_{mn}x_n
\end{aligned}
\right.\]
称为从变量$x_1,x_2,\cdots,x_n$到变量$y_1,y_2,\cdots,y_m$的线性变换,其中$a_{ij}$为常数;线性变换的系数$a_{ij}$构成的矩阵
$\mat{A}=(a_{ij})_{m\times n}$称为线性变换的系数矩阵

若记
$\displaystyle\mat{A}=\begin{bmatrix}
a_{11}&a_{12}&\cdots&a_{1n}\\
a_{21}&a_{22}&\cdots&a_{2n}\\
\vdots&\vdots&\ddots&\vdots\\
a_{m1}&a_{m2}&\cdots&a_{mn}
\end{bmatrix}$,
$\displaystyle\mat{x}=\begin{bmatrix}
x_1\\
x_2\\
\vdots\\
x_n
\end{bmatrix}$,
$\displaystyle\mat{y}=\begin{bmatrix}
y_1\\
y_2\\
\vdots\\
y_m
\end{bmatrix}$,
则线性变换关系是可表示为矩阵形式
\[\mat{y}=\mat{A}\mat{x}\]

当一线性变换的系数矩阵为单位矩阵$\mat{I}$式,线性变换$\mat{y}=\mat{I}\mat{x}$称为恒等变换,因为$\mat{x}=\mat{I}\mat{x}$

\textbf{线性变换实际上构建了一种从矩阵$\mat{x}$到矩阵$\mat{A}\mat{x}$的矩阵变换关系$\mat{x}\to\mat{A}\mat{x}$}





\end{document}

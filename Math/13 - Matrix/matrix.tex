\documentclass[UTF8]{ctexart}

\usepackage{geometry}

\usepackage{amsmath}
\usepackage{amssymb}
\usepackage{esint}
\usepackage{yhmath}
\usepackage{bm}

\usepackage{fancyhdr}
\usepackage{graphicx}

\special{papersize={18.1cm,25.7cm}}
\geometry{left=1.5cm,right=0.5cm,top=2cm,bottom=1cm}
\pagestyle{empty}

\setcounter{MaxMatrixCols}{20}

\newcommand{\D}{{\text{d}\;\!}}
\newcommand{\cross}{\times}
\newcommand{\dif}[1]{{\mathrm{d}\;\!#1}}
\newcommand{\dev}[1]{{\frac{\text{d}}{\dif{#1}}\;\!}}
\newcommand{\ve}[1]{{\bm{#1}}}
\newcommand{\mat}[1]{\ve{#1}}
\newcommand{\ven}[2]{{\left\langle#1,#2\right\rangle}}
\newcommand{\veN}[3]{{\left\langle#1,#2,#3\right\rangle}}
\newcommand{\ang}[2]{{(\widehat{\ve{#1},\ve{#2}})}}
\newcommand{\abs}[1]{{\left|{#1}\right|}}
\newcommand{\when}[2]{{\left.{#1}\right|_{#2}}}
\newcommand{\dist}[2]{{\left\|\ve{#1}-\ve{#2}\right\|}}
\newcommand{\norm}[1]{{\left\|#1\right\|}}
\newcommand{\emplin}{\vspace{1em}}

\begin{document}

\section*{矩阵的概念}
由$m\times n$个数$a_{ij}$排成的$m$行$n$列的数表
\[\begin{matrix}
a_{11}&a_{12}&\cdots&a_{1n}\\
a_{21}&a_{22}&\cdots&a_{2n}\\
\vdots&\vdots&\ddots&\vdots\\
a_{m1}&a_{m2}&\cdots&a_{mn}
\end{matrix}\]
称为$m$行$n$列矩阵,简称$m\times n$矩阵,为表示一个整体,总是加一个括弧,并用大写黑体字母表示它,记为
\[\mat{A}=
\begin{bmatrix}
a_{11}&a_{12}&\cdots&a_{1n}\\
a_{21}&a_{22}&\cdots&a_{2n}\\
\vdots&\vdots&\ddots&\vdots\\
a_{m1}&a_{m2}&\cdots&a_{mn}
\end{bmatrix}\]
这$m\times n$个数称为矩阵$\mat{A}$的元素,$a_{ij}$称为矩阵$\mat{A}$的第$i$行第$j$列的元素,一个$m\times n$矩阵$\mat{A}$也可简记为
\[\mat{A}=\mat{A}_{m\times n}=(a_{ij})_{m\times n}\text{ or }\mat{A}=(a_{ij})\]

\emplin

如果矩阵$\mat{A}$与$\mat{B}$的行数及列数均相同,且对应元素相等,则称矩阵$\mat{A}$与矩阵$\mat{B}$相等,记为$\mat{A}=\mat{B}$

\subsection*{几种特殊矩阵}
\begin{itemize}
  \item \textbf{实矩阵}

  元素均为实数的矩阵
  \item \textbf{复矩阵}

  元素为复数的矩阵
  \item \textbf{非负矩阵}

  元素均为非负数的矩阵
  \item \textbf{$n$阶方阵}

  若矩阵$\mat{A}$的行数与列数都等于$n$,则称$\mat{A}$为$n$阶方阵,记为$\mat{A}_n$
  \item \textbf{同型矩阵}

  如果两个矩阵具有相同的行数与相同的列数,则称这两个矩阵为同型矩阵
  \item \textbf{零矩阵}

  所有元素均为零的矩阵,记为$\mat{O}$
  \item \textbf{$n$阶单位矩阵}

  $n$阶方阵$\displaystyle \begin{bmatrix}
  1&0&\cdots&0\\
  0&1&\cdots&0\\
  \vdots&\vdots&\ddots&\vdots\\
  0&0&\cdots&1
  \end{bmatrix}$,记为
  $\mat{E}=\mat{E}_n\text{ or }\mat{I}=\mat{I}_n$
  \item \textbf{行矩阵\&行向量}

  只有一行的矩阵$\displaystyle\begin{bmatrix}a_1&a_2&\cdots&a_n\end{bmatrix}$,
  为避免元素混淆,也记为$\displaystyle\begin{bmatrix}a_1,&a_2,&\cdots,&a_n\end{bmatrix}$
  \item \textbf{列矩阵\&列向量}

  只有一列的矩阵$\displaystyle\begin{bmatrix}b_1\\b_2\\\vdots\\b_m\end{bmatrix}$
  \item \textbf{$n$阶对角矩阵}

  $n$阶方阵$\displaystyle \begin{bmatrix}
  \lambda_1&0&\cdots&0\\
  0&\lambda_2&\cdots&0\\
  \vdots&\vdots&\ddots&\vdots\\
  0&0&\cdots&\lambda_n
  \end{bmatrix}$,也可记为
  $\mat{A}=\text{diag}(\lambda_1,\lambda_2,\cdots,\lambda_n)$
  \item \textbf{$n$阶数量矩阵}

  当一个$n$阶对角矩阵$\mat{A}$的对角元素全部等于某一数$a$是,即$\mat{A}=\text{diag}(a,a,\cdots,a)=a\mat{I}$
\end{itemize}

\section*{矩阵的运算}
\begin{itemize}
  \item \textbf{取负}

  设矩阵$\mat{A}=(a_{ij})$,记$-\mat{A}=(-a_{ij})$,称$-\mat{A}$为矩阵$\mat{A}$的\textbf{负矩阵}
  \item \textbf{加法}

  设有两$m\times n$的同型矩阵$\mat{A}=(a_{ij})$和$\mat{B}=(b_{ij})$,矩阵$\mat{A}$与$\mat{B}$的和记作$\mat{A}+\mat{B}$,规定为
  \[\mat{A}+\mat{B}=(a_{ij}+b_{ij})=\begin{bmatrix}
  a_{11}+b_{11}&a_{12}+b_{12}&\cdots&a_{1n}+b_{1n}\\
  a_{21}+b_{21}&a_{22}+b_{22}&\cdots&a_{2n}+b_{2n}\\
  \vdots&\vdots&\ddots&\vdots\\
  a_{m1}+b_{m1}&a_{m2}+b_{m2}&\cdots&a_{mn}+b_{mn}
  \end{bmatrix}\]

  \item \textbf{减法}

  由于$\mat{A}+(-\mat{A})=\mat{O}$,则可定义减法$\mat{A}-\mat{B}=\mat{A}+(-\mat{B})$

  \item \textbf{数乘运算}

  数$k$与$m\times n$矩阵$\mat{A}$的乘积记作$k\mat{A}$或$\mat{A}k$,定义为
  \[k\mat{A}=\mat{A}k=(ka_{ij})=\begin{bmatrix}
  ka_{11}&ka_{12}&\cdots&ka_{1n}\\
  ka_{21}&ka_{22}&\cdots&ka_{2n}\\
  \vdots&\vdots&\ddots&\vdots\\
  ka_{m1}&ka_{m2}&\cdots&ka_{mn}
  \end{bmatrix}\]

  \item \textbf{线性运算}

  矩阵的加法与数乘两种运算统称为矩阵的线性运算,它们满足规律
  \begin{itemize}
    \item[>] $\displaystyle \mat{A}+\mat{B}=\mat{B}+\mat{A}$
    \item[>] $\displaystyle (\mat{A}+\mat{B})+\mat{C}=\mat{A}+(\mat{B}+\mat{C})$
    \item[>] $\displaystyle \mat{A}+\mat{O}=\mat{A}$
    \item[>] $\displaystyle \mat{A}+(-\mat{A})=\mat{O}$
    \item[>] $\displaystyle 1\mat{A}=\mat{A}$
    \item[>] $\displaystyle k(l\mat{A})=(kl)\mat{A}$
    \item[>] $\displaystyle (k+l)\mat{A}=k\mat{A}+l\mat{A}$
    \item[>] $\displaystyle k(\mat{A}+\mat{B})=k\mat{A}+k\mat{B}$
  \end{itemize}

  \item \textbf{乘法}

  设
  \[\mat{A}=(a_{ij})_{m\times s}=\begin{bmatrix}
  a_{11}&a_{12}&\cdots&a_{1s}\\
  a_{21}&a_{22}&\cdots&a_{2s}\\
  \vdots&\vdots&\ddots&\vdots\\
  a_{m1}&a_{m2}&\cdots&a_{ms}
  \end{bmatrix}\quad
  \mat{B}=(b_{ij})_{s\times n}=\begin{bmatrix}
  b_{11}&b_{12}&\cdots&b_{1n}\\
  b_{21}&b_{22}&\cdots&b_{2n}\\
  \vdots&\vdots&\ddots&\vdots\\
  b_{s1}&b_{s2}&\cdots&b_{sn}
  \end{bmatrix}\]
  矩阵$\mat{A}$与矩阵$\mat{B}$的乘积记作$\mat{A}\mat{B}$,定义为
  \[\mat{A}\mat{B}=(c_{ij})_{m\times n}=\begin{bmatrix}
  c_{11}&c_{12}&\cdots&c_{1n}\\
  c_{21}&c_{22}&\cdots&c_{2n}\\
  \vdots&\vdots&\ddots&\vdots\\
  c_{m1}&c_{m2}&\cdots&c_{mn}
  \end{bmatrix}\quad\left(
  c_{ij}=\sum_{k=1}^sa_{ik}b_{kj}
  \right)\]
  若$\mat{C}=\mat{A}\mat{B}$,则矩阵$\mat{C}$的元素$c_{ij}$即为矩阵$\mat{A}$第$i$行元素与矩阵$\mat{B}$第$j$行元素对应元素乘积之和,即
  \[c_{ij}=
  \begin{bmatrix}
    a_{i1} & a_{i2} & \cdots & a_{is}
  \end{bmatrix}
  \begin{bmatrix}
    b_{1j}\\
    b_{2j}\\
    \vdots\\
    b_{sj}
  \end{bmatrix}
  =a_{i1}b_{1j}+a_{i2}b_{2j}+\cdots+a_{is}b_{sj}
  =\sum_{k=1}^sa_{ik}b_{kj}
  \]

  \emplin

  显然$\mat{A}\mat{B}\ne\mat{B}\mat{A}$(有时两者中只有一个有定义)

  两非零矩阵的乘积可能为零矩阵,故不能从$\mat{A}\mat{B}=\mat{O}$得出$\mat{A}\text{ or }\mat{B}=\mat{O}$

  矩阵乘法一般也不满足消去律,即不能从$\mat{A}\mat{C}=\mat{B}\mat{C}$得出$\mat{A}=\mat{B}$

  \emplin

  矩阵乘法满足运算规则(若有定义)
  \begin{itemize}
    \item[>] $\displaystyle (\mat{A}\mat{B})\mat{C}=\mat{A}(\mat{B}\mat{C})$
    \item[>] $\displaystyle (\mat{A}+\mat{B})\mat{C}=\mat{A}\mat{C}+\mat{B}\mat{C}$
    \item[>] $\displaystyle \mat{C}(\mat{A}+\mat{B})=\mat{C}\mat{A}+\mat{C}\mat{B}$
    \item[>] $\displaystyle k(\mat{A}\mat{B})=(k\mat{A})\mat{B}=\mat{A}(k\mat{B})$
  \end{itemize}

  \item \textbf{可交换}

  如果两矩阵相乘,有$\mat{A}\mat{B}=\mat{B}\mat{A}$,则称矩阵$\mat{A}$与矩阵$\mat{B}$可交换,简称$\mat{A}$与$\mat{B}$可换

  (对于单位矩阵有$\mat{I}_m\mat{A}_{m\times n}=\mat{A}_{m\times n}\mat{I}_n=\mat{A}_{m\times n}$)

  \item \textbf{转置}

  把矩阵$\mat{A}$的行换成同序数的列所得到的新矩阵称为$\mat{A}$的转置矩阵,记作$\mat{A}^T\text{ or }\mat{A}'$

  即若$\displaystyle\mat{A}=\begin{bmatrix}
  a_{11}&a_{12}&\cdots&a_{1n}\\
  a_{21}&a_{22}&\cdots&a_{2n}\\
  \vdots&\vdots&\ddots&\vdots\\
  a_{m1}&a_{m2}&\cdots&a_{mn}
  \end{bmatrix}$,则$\displaystyle\mat{A}^T=\begin{bmatrix}
  a_{11}&a_{21}&\cdots&a_{m1}\\
  a_{12}&a_{22}&\cdots&a_{m2}\\
  \vdots&\vdots&\ddots&\vdots\\
  a_{1n}&a_{2n}&\cdots&a_{mn}
  \end{bmatrix}$

  \emplin

  矩阵的转置满足运算规则(若有定义)
  \begin{itemize}
    \item[>] $\displaystyle (\mat{A}^T)^T=\mat{A}$
    \item[>] $\displaystyle (\mat{A}+\mat{B})^T=\mat{A}^T+\mat{B}^T$
    \item[>] $\displaystyle (k\mat{A})^T=k\mat{A}^t$
    \item[>] $\displaystyle (\mat{A}\mat{B})^T=\mat{B}^T\mat{A}^T$
  \end{itemize}

  \item \textbf{方阵的幂}

  设方阵$\mat{A}=(a_{ij})_{n\times n}$,规定
  \[\mat{A}^0=\mat{I}\text{, }\mat{A}^k=\overbrace{\mat{A}\cdot\mat{A}\cdot\cdots\cdot\mat{A}}^{k\text{ of }\mat{A}}\]

  \emplin

  矩阵的幂满足运算规则
  \begin{itemize}
    \item[>] $\displaystyle \mat{A}^m\mat{A}^n=\mat{A}^{m+n}$
    \item[>] $\displaystyle (\mat{A}^m)^n=\mat{A}^{mn}$
  \end{itemize}

  \item \textbf{方阵的行列式}

  由$n$阶方阵$\mat{A}$的元素所构成的行列式(各元素的位置不变),称为方阵$\mat{A}$的行列式,记作
  \[\abs{\mat{A}}\text{ or }\det\mat{A}\]

  \emplin

  矩阵$\mat{A}$的行列式$\det\mat{A}$满足运算规则(其中$\mat{A}$与$\mat{B}$同为$n$阶方阵)
  \begin{itemize}
    \item[>] $\displaystyle \det\mat{A}^T=\det\mat{A}$
    \item[>] $\displaystyle \det(k\mat{A})=k^n\det\mat{A}$
    \item[>] $\displaystyle \det(\mat{A}\mat{B})=\det\mat{A}\det\mat{B}$
    \item[>] $\displaystyle \det(\mat{A}\mat{B})=\det(\mat{B}\mat{A})$
  \end{itemize}

  \item \textbf{对称矩阵}

  设$\mat{A}$为$n$阶方阵,如果$\mat{A}^T=\mat{A}$,即$a_{ij}=a_{ji}$,则称$\mat{A}$为对称矩阵

  \item \textbf{反对称矩阵}

  设$\mat{A}$为$n$阶方阵,如果$\mat{A}^T=-\mat{A}$,即$a_{ij}=-a_{ji}$,则称$\mat{A}$为反对称矩阵

  \item \textbf{共轭矩阵}

  设$\mat{A}=(a_{ij})$为复矩阵,记$\overline{\mat{A}}=(\overline{a_{ij}})$,其中$\overline{a_{ij}}$为$a_{ij}$的共轭复数,
  称$\overline{\mat{A}}$为$\mat{A}$的共轭矩阵

  \emplin

  \begin{itemize}
    \item[>] $\displaystyle \overline{\mat{A}+\mat{B}}=\overline{\mat{A}}+\overline{\mat{B}}$
    \item[>] $\displaystyle \overline{\lambda\mat{A}}=\overline{\lambda}\overline{\mat{A}}$
    \item[>] $\displaystyle \overline{\mat{A}\mat{B}}=\overline{\mat{A}}\overline{\mat{B}}$
    \item[>] $\displaystyle \overline{(\mat{A}^T)}=(\overline{\mat{A}})^T$
  \end{itemize}
\end{itemize}

\section*{逆矩阵}

对于一个$n$阶方阵$\mat{A}$,如果存在一个$n$阶方阵$\mat{B}$,使得$\mat{A}\mat{B}=\mat{B}\mat{A}=\mat{I}$,则称方阵$\mat{A}$为可逆矩阵,
而方阵$\mat{B}$称为$\mat{A}$的逆矩阵

\textbf{若矩阵$\mat{A}$是可逆的,则$\mat{A}$的逆矩阵是唯一的,记为$\mat{A}^{-1}$}

\emplin

如果$n$阶方阵$\mat{A}$的行列式$\det\mat{A}\ne0$,则称$\mat{A}$为非奇异的,否则称$\mat{A}$为奇异的

\subsection*{伴随矩阵与逆矩阵}

行列式$\det\mat{A}$的各个元素的代数余子式$A_{ij}$所构成的矩阵
\[\mat{A}^*=\begin{bmatrix}
A_{11}&A_{21}&\cdots&A_{n1}\\
A_{12}&A_{22}&\cdots&A_{n2}\\
\vdots&\vdots&\ddots&\vdots\\
A_{1n}&A_{2n}&\cdots&A_{nn}
\end{bmatrix}\]
称为矩阵$\mat{A}$的伴随矩阵

\emplin

$n$阶矩阵$\mat{A}$可逆的充分必要条件是其行列式$\det\mat{A}\ne0$,且当$\mat{A}$可逆时,有
\[\mat{A}^{-1}=\frac{1}{\det\mat{A}}\mat{A}^*\]
其中$\mat{A}^*$为$\mat{A}$的伴随矩阵

\emplin

伴随矩阵的一个基本性质
\[\mat{A}\mat{A}^*=\mat{A}^*\mat{A}=(\det\mat{A})\mat{I}\]

\subsection*{逆矩阵的运算性质}

\begin{itemize}
  \item 若矩阵$\mat{A}$可逆,则$\mat{A}^{-1}$也可逆,且$(\mat{A}^{-1})^{-1}=\mat{A}$
  \item 若矩阵$\mat{A}$可逆,数$k\ne0$,则$(k\mat{A})^{-1}=\frac{1}{k}\mat{A}^{-1}$
  \item 两个同阶可逆矩阵$\mat{A}$,$\mat{B}$的乘积也是可逆矩阵,且$(\mat{A}\mat{B})^{-1}=\mat{B}^{-1}\mat{A}^{-1}$
  ($(\mat{A}_1\mat{A}_2\cdots\mat{A}_n)^{-1}=\mat{A}_n^{-1}\cdots\mat{A}_2^{-1}\mat{A}_1^{-1}$)
  \item 若矩阵$\mat{A}$可逆,则$\mat{A}^T$也可逆,且有$(\mat{A}^T)^{-1}=(\mat{A}^{-1})^T$
  \item 若矩阵$\mat{A}$可逆,则$\det(\mat{A}^{-1})=(\det\mat{A})^{-1}$
\end{itemize}

\section*{线性方程组的矩阵表示}
对于线性方程组
\[\left\{
\begin{aligned}
a_{11}x_1+a_{12}x_2+\cdots+a_{1n}x_n&=b_1\\
a_{21}x_1+a_{22}x_2+\cdots+a_{2n}x_n&=b_2\\
\cdots\cdots\cdots\cdots\cdots\cdots\cdots\cdots\cdots&\cdots\cdots\\
a_{m1}x_1+a_{m2}x_2+\cdots+a_{mn}x_n&=b_m
\end{aligned}
\right.\]
若记
$\displaystyle\mat{A}=\begin{bmatrix}
a_{11}&a_{12}&\cdots&a_{1n}\\
a_{21}&a_{22}&\cdots&a_{2n}\\
\vdots&\vdots&\ddots&\vdots\\
a_{m1}&a_{m2}&\cdots&a_{mn}
\end{bmatrix}$,
$\displaystyle\mat{x}=\begin{bmatrix}
x_1\\
x_2\\
\vdots\\
x_n
\end{bmatrix}$,
$\displaystyle\mat{b}=\begin{bmatrix}
b_1\\
b_2\\
\vdots\\
b_m
\end{bmatrix}$,
则利用矩阵的乘法,线性方程组可表示为
\[\mat{A}\mat{x}=\mat{b}\]
其中$\mat{A}$称为方程组的系数矩阵,方程$\mat{A}\mat{x}=\mat{b}$称为矩阵方程

如果$x_j=c_j$是方程组的解,记列矩阵$\mat{\eta}=\begin{bmatrix}
c_1\\
c_2\\
\vdots\\
c_n
\end{bmatrix}$,则$\mat{A}\mat{\eta}=\mat{b}$这是也称$\mat{\eta}$是矩阵方程的解;反之如果$\mat{\eta}$是矩阵方程的解,
既有矩阵等式$\mat{A}\mat{\eta}=\mat{b}$成立,则$\mat{x}=\mat{\eta}$,即$x_j=c_j$也是线性方程组的解

\section*{线性变换}
变量$x_1,x_2,\cdots,x_n$与变量$y_1,y_2,\cdots,y_m$之间的关系式
\[\left\{
\begin{aligned}
y_1&=a_{11}x_1+a_{12}x_2+\cdots+a_{1n}x_n\\
y_2&=a_{21}x_1+a_{22}x_2+\cdots+a_{2n}x_n\\
\cdots&\cdots\cdots\cdots\cdots\cdots\cdots\cdots\cdots\cdots\cdots\\
y_m&=a_{m1}x_1+a_{m2}x_2+\cdots+a_{mn}x_n
\end{aligned}
\right.\]
称为从变量$x_1,x_2,\cdots,x_n$到变量$y_1,y_2,\cdots,y_m$的线性变换,其中$a_{ij}$为常数;线性变换的系数$a_{ij}$构成的矩阵
$\mat{A}=(a_{ij})_{m\times n}$称为线性变换的系数矩阵

若记
$\displaystyle\mat{A}=\begin{bmatrix}
a_{11}&a_{12}&\cdots&a_{1n}\\
a_{21}&a_{22}&\cdots&a_{2n}\\
\vdots&\vdots&\ddots&\vdots\\
a_{m1}&a_{m2}&\cdots&a_{mn}
\end{bmatrix}$,
$\displaystyle\mat{x}=\begin{bmatrix}
x_1\\
x_2\\
\vdots\\
x_n
\end{bmatrix}$,
$\displaystyle\mat{y}=\begin{bmatrix}
y_1\\
y_2\\
\vdots\\
y_m
\end{bmatrix}$,
则线性变换关系是可表示为矩阵形式
\[\mat{y}=\mat{A}\mat{x}\]

当一线性变换的系数矩阵为单位矩阵$\mat{I}$式,线性变换$\mat{y}=\mat{I}\mat{x}$称为恒等变换,因为$\mat{x}=\mat{I}\mat{x}$

\textbf{线性变换实际上构建了一种从矩阵$\mat{x}$到矩阵$\mat{A}\mat{x}$的矩阵变换关系$\mat{x}\to\mat{A}\mat{x}$}

\section*{矩阵方程}
对标准矩阵方程
\[\mat{A}\mat{X}=\mat{B},\quad\mat{X}\mat{A}=\mat{B},\quad\mat{A}\mat{X}\mat{B}=\mat{C}\]
利用矩阵乘法的运算规律和逆矩阵的运算性质,可解出
\[\mat{X}=\mat{A}^{-1}\mat{B},\quad\mat{X}=\mat{B}\mat{A}^{-1},\quad\mat{X}=\mat{A}^{-1}\mat{C}\mat{B}^{-1}\]
而其它形式的矩阵方程,可以转化标准矩阵方程

\section*{矩阵多项式及其运算}
设$\varphi(x)=a_0+a_1x+\cdots+a_mx^m$为$x$的$m$次多项式,$\mat{A}$为$n$阶矩阵,记
\[\varphi(\mat{A})=a_0\mat{I}+a_1\mat{A}+\cdots+a_m\mat{A}^m\]
$\varphi(\mat{A})$称为矩阵$\mat{A}$的$m$次多项式

\emplin

$f(\mat{A})g(\mat{A})=g(\mat{A})f(\mat{A})$总是成立,从而$\mat{A}$的多项式可以像数$x$的多项式一样相乘或分解因式

\emplin

\textbf{如果$\mat{A}=\mat{P}\mat{\Lambda}\mat{P}^{-1}$,则$\mat{A}^k=\mat{P}\mat{\Lambda}^k\mat{P}^{-1}$},从而
\[\varphi(\mat{A})=a_0\mat{I}+a_1\mat{A}+\cdots+a_m\mat{A}^m=\mat{P}\varphi(\mat{\Lambda})\mat{P}^{-1}\]

如果$\mat{\Lambda}=\text{diag}(\lambda_1,\lambda_2,\cdots,\lambda_n)$为对角矩阵,则
\[\mat{\Lambda}^k=\text{diag}(\lambda_1^k,\lambda_2^k,\cdots,\lambda_n^k)\]
从而$\varphi(\mat{\Lambda})=a_0\mat{I}+a_1\mat{\Lambda}+\cdots+a_m\mat{\Lambda}=\text{diag}(\varphi(\lambda_1),\varphi(\lambda_2),\cdots,\varphi(\lambda_n))$

\section*{分块矩阵}
若将大矩阵$\mat{A}$用若干条纵线与横线分成多个小矩阵,每个小矩阵称为$\mat{A}$的\textbf{子块},以子块为元素的\emph{形式上}的矩阵称为\textbf{分块矩阵}

\subsection*{分块矩阵的运算}
\begin{itemize}
  \item \textbf{加法}

   若矩阵$\mat{A}$与$\mat{B}$的行数、列数均相同,且采用相同的分块方法,则$\mat{A}+\mat{B}$的每个分块是$\mat{A}$与$\mat{B}$中对应分块之和
  \item \textbf{数乘}

  设$\mat{A}$是一个分块矩阵,$k$为一实数,则$k\mat{A}$的每个子块是$k$与$\mat{A}$中相应子块的数乘
  \item \textbf{乘法}

  两分块矩阵$\mat{A}$与$\mat{B}$的乘积依然按照普通矩阵的乘积进行运算,即把矩阵$\mat{A}$与$\mat{B}$中的子块当作数量来对待,
  但对于乘积$\mat{A}\mat{B}$,$\mat{A}$的列划分必须与$\mat{B}$的行划分一致
  \item \textbf{转置}

  设$\displaystyle\mat{A}=\begin{bmatrix}\mat{A}_{11}&\cdots&\mat{A}_{1t}\\\vdots&\ddots&\vdots\\\mat{A}_{s1}&\cdots&\mat{A}_{st}\end{bmatrix}$,
  则$\displaystyle\mat{A}^T=\begin{bmatrix}\mat{A}_{11}^T&\cdots&\mat{A}_{s1}^T\\\vdots&\ddots&\vdots\\\mat{A}_{1t}^T&\cdots&\mat{A}_{st}^T\end{bmatrix}$
  \item \textbf{分块对角矩阵}

  若$\mat{A}$为$n$阶矩阵,若$\mat{A}$的分块矩阵实在对角线上有非零子块,其余子式都为零矩阵,且在对角线上的子块都是方阵,即
  \[\mat{A}=\begin{bmatrix}
  \mat{A}_1&&&\mat{O}\\
  &\mat{A}_2\\
  &&\ddots\\
  \mat{O}&&&\mat{A}_s
  \end{bmatrix}\]
  其中$\mat{A}_i$都是方阵,则称$\mat{A}$为分块对角矩阵

  分块对角矩阵具有性质
  \begin{itemize}
    \item 若$\det\mat{A}_i\ne0$,则$\det\mat{A}\ne0$,且$\det\mat{A}=\det\mat{A}_1\mat{A}_2\cdots\mat{A}_s$
    \item \[\mat{A}^{-1}=\begin{bmatrix}
    \mat{A}_1^{-1}&&&\mat{O}\\
    &\mat{A}_2^{-1}\\
    &&\ddots\\
    \mat{O}&&&\mat{A}_s
    \end{bmatrix}\]
    \item 同结构的分块对角矩阵的和、差、积、数乘及逆仍是分块对角矩阵,且运算表现为对应子块的运算
  \end{itemize}

  \item \textbf{分块上(下)三角矩阵}

  形如
  \[\begin{bmatrix}
  \mat{A}_{11}&\mat{A}_{12}&\cdots&\mat{A}_{1s}\\
  \mat{O}&\mat{A}_{22}&\cdots&\mat{A}_{2s}\\
  \vdots&\vdots&\ddots&\vdots\\
  \mat{O}&\mat{O}&\cdots&\mat{A}_{ss}
  \end{bmatrix}\text{ or }\begin{bmatrix}
  \mat{A}_{11}&\mat{O}&\cdots&\mat{O}\\
  \mat{A}_{21}&\mat{A}_{22}&\cdots&\mat{O}\\
  \vdots&\vdots&\ddots&\vdots\\
  \mat{A}_{s1}&\mat{A}_{s2}&\cdots&\mat{A}_{ss}
  \end{bmatrix}\]
  的分块矩阵,分别称为分块上三角矩阵或分块下三角矩阵,其中$\mat{A}_{pp}$是方阵;同结构的分块上(下)三角矩阵的和、差、积、数乘及逆仍是分块上(下)三角形矩阵
\end{itemize}

\section*{矩阵的初等变换}
矩阵的下列三种变换称为矩阵的初等行变换:
\begin{enumerate}
  \item 交换矩阵的两行(交换$i$,$j$两行,记作$r_i\leftrightarrow r_j$)
  \item 以一个非零的数$k$乘矩阵的某一行(第$i$行乘数$k$,记作$kr_i$或$r_i\times k$)
  \item 把矩阵的某一行的$k$倍加到另一行(第$j$行乘数$k$加到第$i$行,记为$r_i+kr_j$)
\end{enumerate}

矩阵的下列三种变换称为矩阵的初等列变换:
\begin{enumerate}
  \item 交换矩阵的两列(交换$i$,$j$两列,记作$c_i\leftrightarrow c_j$)
  \item 以一个非零的数$k$乘矩阵的某一列(第$i$列乘数$k$,记作$kc_i$或$c_i\times k$)
  \item 把矩阵的某一列的$k$倍加到另一列(第$j$列乘数$k$加到第$i$列,记为$c_i+kc_j$)
\end{enumerate}

初等行变换与初等列变换统称\textbf{初等变换};初等别换的逆变换依然为初等变换,且变换类型相同

\emplin
\emplin

若矩阵$\mat{A}$经过有限次的初等变换变成矩阵$\mat{B}$,则称矩阵$\mat{A}$与$\mat{B}$等价,记为
\[\mat{A}\to\mat{B}\text{ or }\mat{A}\sim\mat{B}\]
矩阵间的等价关系具有下列基本性质
\begin{itemize}
  \item \textbf{自反性} $\mat{A}\sim\mat{A}$
  \item \textbf{对称性} 若$\mat{A}\sim\mat{B}$,则$\mat{B}\sim\mat{A}$
  \item \textbf{传递性} 若$\mat{A}\sim\mat{B}$,$\mat{B}\sim\mat{C}$,则有$\mat{A}\sim\mat{C}$
\end{itemize}

\emplin
\emplin

称满足下列条件的矩阵为\textbf{行阶梯形矩阵}
\begin{itemize}
  \item 零行(元素均为零的行)位于矩阵的下方
  \item 各非零行的首个非零元(从左至右的第一个不为零的元素)的列标随行标的增大而严格增大(或说其列标一定不小于行标)
\end{itemize}

\emplin
\emplin

称满足下列条件的阶梯形矩阵为\textbf{行最简形矩阵}
\begin{itemize}
  \item 各非零行的首个非零元都是$1$
  \item 每个首行非零元所在列的其他元素均为$0$
\end{itemize}

\emplin
\emplin

对于任意矩阵$\mat{A}$经过有限次初等线性变换,均可化为\textbf{标准形}矩阵(一行最简形矩阵)
\[\mat{D}=\begin{bmatrix}
1\\
&\ddots\\
&&1\\
&&&0\\
&&&&\ddots\\
&&&&&0
\end{bmatrix}=\begin{bmatrix}
\mat{E}_r&\mat{O}_{r\times(n-r)}\\
\mat{O}_{(m-r)\times r}&\mat{O}_{(m-r)\times(n-r)}
\end{bmatrix}\]

\emplin
\emplin

任一矩阵$\mat{A}$总可以经过有限次初等行变换后化为行阶梯形矩阵,并进而化为行最简形矩阵

如果$\mat{A}$为$n$阶可逆矩阵,则矩阵$\mat{A}$经过有限次初等变换可化为单位矩阵$\mat{I}$,即$\mat{A}\to\mat{I}$

\subsection*{初等矩阵}
对单位矩阵$\mat{I}$施以\emph{一次初等变换}得到的矩阵称为\textbf{初等矩阵},三种初等变换分别对应着三种初等矩阵
\begin{enumerate}
  \item $\mat{I}$的第$i$,$j$行(列)互换得到的矩阵
  \[\mat{I}(i,j)=\begin{bmatrix}
  1\\
  &\ddots\\
  &&1\\
  &&&0&\cdots&\cdots&\cdots&1\\
  &&&\vdots&1&&&\vdots\\
  &&&\vdots&&\ddots&&\vdots\\
  &&&\vdots&&&1&\vdots\\
  &&&1&\cdots&\cdots&\cdots&0\\
  &&&&&&&&1\\
  &&&&&&&&&\ddots\\
  &&&&&&&&&&1
  \end{bmatrix}\]

  \item $\mat{I}$的第$i$行(列)乘以非零数$k$得到的矩阵
  \[\mat{I}(i(k))=\begin{bmatrix}
  1\\
  &\ddots\\
  &&k\\
  &&&\ddots\\
  &&&&1
  \end{bmatrix}\]

  \item $\mat{I}$的第$j$行乘以数$k$加到第$i$行上,或$\mat{I}$的第$i$列乘以数$k$加到第$j$行上得到的矩阵
  \[\mat{I}(i\quad j(k))=\begin{bmatrix}
  1\\
  &\ddots\\
  &&1&\cdots&k\\
  &&&\ddots&\vdots\\
  &&&&1\\
  &&&&&\ddots\\
  &&&&&&1
  \end{bmatrix}\]
\end{enumerate}

初等矩阵具有下列基本性质
\begin{itemize}
  \item $\displaystyle \mat{I}(i,j)^{-1}=\mat{I}(i,j)$
  \item $\displaystyle \mat{I}(i(k))^{-1}=\mat{I}(i(k^{-1}))$
  \item $\displaystyle \mat{I}(i\quad j(k))^{-1}=\mat{I}(i\quad j(-k))$
  \emplin
  \item $\displaystyle \det\mat{I}(i,j)=-1$
  \item $\displaystyle \det\mat{I}(i(k))=k$
  \item $\displaystyle \det\mat{I}(i\quad j(k))=1$
\end{itemize}

\textbf{设$\mat{A}$为一个$m\times n$矩阵,对$\mat{A}$施行一次某种初等行(列)变换,相当一用同种的$m$($n$)阶初等矩阵左(右)乘$\mat{A}$}

\subsection*{利用初等变换求矩阵的逆}

$n$阶矩阵$\mat{A}$可逆的充分必要条件是$\mat{A}$可以表示为若干初等矩阵的乘积

因而求求矩阵$\mat{A}$的逆矩阵$\mat{A}^{-1}$时,可构造$n\times2n$矩阵$\displaystyle\begin{bmatrix}\mat{A}&\mat{I}\end{bmatrix}$,
然后对其施以初等行变换将矩阵$\mat{A}$化为单位矩阵$\mat{I}$,则上述初等行变换同时也将其中的单位矩阵$\mat{I}$化为$\mat{A}^{-1}$,即
\[\begin{bmatrix}\mat{A}&\mat{I}\end{bmatrix}\to\begin{bmatrix}\mat{I}&\mat{A}^{-1}\end{bmatrix}\]

\subsection*{利用初等变换求解矩阵方程}
设矩阵$\mat{A}$可逆,则求解矩阵方程$\mat{A}\mat{X}=\mat{B}$等价于求矩阵$\mat{X}=\mat{A}^{-1}\mat{B}$,为此构造矩阵
$\begin{bmatrix}\mat{A}&\mat{B}\end{bmatrix}$,对其施以初等行变换将矩阵$\mat{A}$化为单位矩阵$\mat{I}$,则上述初等变换矩阵同时将其中的
矩阵$\mat{B}$化为$\mat{A}^{-1}\mat{B}$,即
\[\begin{bmatrix}\mat{A}&\mat{B}\end{bmatrix}\to\begin{bmatrix}\mat{I}&\mat{A}^{-1}\mat{B}\end{bmatrix}\]
这样就给出了用初等变换求解矩阵方程$\mat{A}\mat{X}=\mat{B}$的方法

\emplin

同理,求解矩阵方程$\mat{X}\mat{A}=\mat{B}$等价于计算矩阵$\mat{B}\mat{A}^{-1}$,亦可利用初等列变换求解矩阵$\mat{B}\mat{A}^{-1}$,即
\[\begin{bmatrix}\mat{A}\\\mat{B}\end{bmatrix}\to\begin{bmatrix}\mat{I}\\\mat{B}\mat{A}^{-1}\end{bmatrix}\]

\section*{矩阵的秩 The Rank of Matrix}
矩阵可经初等行变换化为行阶梯形矩阵,且行阶梯形矩阵所行含非零行的行数时唯一确定的(而这个数就是矩阵的秩,介于其唯一性尚未证明,先用行列式定义矩阵的秩)

\emplin

\textbf{在$m\times n$矩阵$\mat{A}$中,任取$k$行$k$列,位于这些行列交叉处的$k^2$个元素,不改变它们在$\mat{A}$中的顺序而得到的$k$阶行列式,称为矩阵$\mat{A}$的$k$阶子式}

\emplin
\emplin

\textbf{设$\mat{A}$为$m\times n$矩阵,如果存在$\mat{A}$的$r$阶子式不为零,而任一$r+1$阶子式皆为零,则称数$r$为矩阵$\mat{A}$的秩,记为$r(\mat{A})$(或$R(\mat{A})$),并规定零矩阵的秩等于零}

\subsection*{性质}
\begin{itemize}
  \item 若矩阵$\mat{A}$中有某个$s$阶子式不为$0$,则$r(\mat{A})\ge s$
  \item 若$\mat{A}$中所有$t$阶子式全为$0$,则$r(\mat{A})<t$
  \item 若$\mat{A}$为$m\times n$矩阵,则$0\le r(\mat{A})\le \min\{ m,n \}$
  \item 当$r(\mat{A})-\min\{ m,n \}$时,称$\mat{A}$为\textbf{满秩矩阵},否则称为\textbf{降秩矩阵}
  \item $r(\mat{A})=r(\mat{A}^T)$
  \item $\max\{ r(\mat{A}),r(\mat{B})\}\le r(\mat{A},\mat{B})\le r(\mat{A})+r(\mat{B})$
  \item $r(\mat{A}+\mat{B})\le r(\mat{A})+r(\mat{B})$
  \item $r(\mat{A}\mat{B})\le \min\{ r(\mat{A},\mat{B}) \}$
  \item 若$\mat{A}_{m\times n}\mat{B}_{n\times l}=\mat{O}$,则$r(\mat{A})+r(\mat{B})\le n$
\end{itemize}

\subsection*{矩阵的秩的求法}
若$\mat{A}\to\mat{B}$($\mat{A}$经过有限次初等变换为$\mat{B}$),则$r(\mat{A})=r(\mat{B})$

\textbf{用初等行变换把矩阵变成行阶梯形矩阵,行阶梯形矩阵中非零行的行数就是该矩阵的秩}

由矩阵的秩及满秩矩阵的定义,显然,若一个$n$阶矩阵$\mat{A}$是满秩的,则$\det\mat{A}\ne0$,因而非奇异;反之亦然




\end{document}

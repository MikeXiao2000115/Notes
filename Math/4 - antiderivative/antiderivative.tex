\documentclass[UTF8]{ctexart}

\usepackage{geometry}

\usepackage{amsmath}
\usepackage{amssymb}

\usepackage{fancyhdr}
\usepackage{graphicx}

\special{papersize={18.1cm,25.7cm}}
\geometry{left=1.5cm,right=0.5cm,top=2cm,bottom=1cm}
\pagestyle{empty}

\newcommand{\D}{\text{d}\;\!}
\newcommand{\dif}[1]{\text{d}\;\!#1}

\begin{document}

\section*{不定积分 anti-derivative}

\bigskip

如果在区间$I$上,可导函数$F(x)$的导数为$f(x)$,即对于任意$x\in I$,都有
\[ F'(x)=f(x)\quad or \quad \D F(x)=f(x)\D x \]
那么函数$F(x)$就称为$f(x)$(或$f(x)\D x$)在区间$I$上的一个原函数 (primitive function)

\bigskip

如果函数$f(x)$在区间$I$上连续,那么在区间$I$上存在可导函数$F(x)$,使对任一$x\in I$都有
\[ F'(x)=f(x) \]

\bigskip

在区间$I$上,函数$f(x)$的带有任一常数项的原函数称为$f(x)$在区间$I$上的不定积分,记作
\[\int f(x)\D x\]
其中记号$\displaystyle\int$称为积分号,$f(x)$称为被积函数,$f(x)\D x$称为被积表达式,$x$称为积分变量

函数$f(x)$的原函数图像称为$f(x)$的积分曲线
\bigskip
\bigskip

\section*{基本积分表}

\bigskip

\begin{center}
  \renewcommand{\arraystretch}{2}
  \begin{tabular}{ll}
    $ \displaystyle\int k \D x=kx+C$&$ \displaystyle\int x^\mu \D x=\frac{x^{\mu+1}}{\mu+1}+C\quad(\mu\ne-1) $\\
    $ \displaystyle\int \frac{1}{x}\D x = \ln{|x|}+C$&$ \displaystyle\int \frac{1}{1+x^2}=\arctan x +C$\\
    $ \displaystyle\int \frac{1}{\sqrt{1-x^2}}\D x=\arcsin x+C$&$ \displaystyle\int \cos x\D x=\sin x+C$\\
    $ \displaystyle\int \sin x\D x=-\cos x + C$ & $\displaystyle\int \frac{1}{\cos^2x}=\int \sec^2x \D x=\tan x+C$\\
    $ \displaystyle\int \frac{1}{\sin^2x}\D x = \int \csc^2 x\D x=-\cot x+C$ & $\displaystyle\int \sec x\tan x\D x=\sec x+C$\\
    $ \displaystyle\int \csc x\cot x\D x=-\csc x+C$ & $\displaystyle\int e^x\D x=e^x+C$\\
    $ \displaystyle\int a^x\D x=\frac{a^x}{\ln a}+C$& $\displaystyle\int \tan x\dif{x}=-\ln|\cos x|+C$\\
    $ \displaystyle\int \cot x\dif{x} = \ln|\sin x|+C$ & $\displaystyle\int \sec x\dif{x}=\ln|\sec x+\tan x|+C$\\
    $ \displaystyle\int \csc x\dif{x} = \ln|\csc x - \cot x| + C$ & $\displaystyle\int \frac{1}{a^2+x^2}\dif{x}=\frac{1}{a}\arctan{\frac{x}{a}}+C$\\
    $ \displaystyle\int \frac{1}{x^2-a^2}\dif{x}=\frac{1}{2a}\ln\left|\frac{x-a}{x+a}\right|+C$&$\displaystyle\int\frac{1}{\sqrt{a^2-x^2}}\dif{x}=\arctan\frac{x}{a}+C$\\
    $ \displaystyle\int \frac{1}{\sqrt{x^2+a^2}}\dif{x} = \ln(x+\sqrt{x^2+a^2})+C$&$ \displaystyle\int \frac{1}{\sqrt{x^2-a^2}}\dif{x} = \ln|x+\sqrt{x^2-a^2}|+C$\\
    $ \displaystyle\int[f(x)+g(x)]\D x=\int f(x)\D x+\int g(x)\D x$ &  $ \displaystyle\int kf(x)\D x=k\int f(x)\D x$ \\
    $ \displaystyle\int f[\varphi(x)]\varphi'(x)\D x = \left[\int f(u)\D u \right]_{u=\varphi(x)}$ & $ \displaystyle\int f(x)\dif{x} = \left[\int f[\psi(t)]\psi'(t)\dif{t}\right]_{t=\psi^{-1}(x)}$
  \end{tabular}
\end{center}

\[ \int u\dif{v}=uv-\int v\dif{u} \]

\end{document}

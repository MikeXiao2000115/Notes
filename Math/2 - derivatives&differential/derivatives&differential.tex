\documentclass[UTF8]{ctexart}

\usepackage{geometry}

\usepackage{amsmath}
\usepackage{amssymb}

\usepackage{fancyhdr}
\usepackage{graphicx}

\special{papersize={18.1cm,25.7cm}}
\geometry{left=1.5cm,right=0.5cm,top=2cm,bottom=1cm}
\pagestyle{empty}

\newcommand{\D}{\text{d}}

\begin{document}



\section*{导数 Derivative}

\bigskip

设函数$y=f(x)$在点$x_0$的某个邻域内有定义,当自变量$x$在$x_0$处取得增量$\Delta x$
(点$x_0+\Delta x$仍在该邻域内)时,相应地,因变量取得增量$\Delta y=f(x_0+\Delta x)-f(x_0)$;
如果$\Delta y$与$\Delta x$之比当$\Delta x\to0$时的极限存在,那么称函数$y=f(x)$在点$x_0$处可导,
并称这个极限为函数$y=f(x)$在点$x_0$处的导数 (Derivative),记为$f'(x_0)$,即
\[ f'(x_0)=\lim_{\Delta x\to0}\frac{\Delta y}{\Delta x}=\lim_{\Delta x\to0}\frac{f(x_0+\Delta x)-f(x_0)}{\Delta x} \]
也可记为$y'|_{x=x_0}$,$\frac{\text{d}y}{\text{d}x}\big|_{x=x_0}$或$\frac{\text{d}f(x)}{\text{d}x}\big|_{x=x_0}$

\bigskip

如果函数$f(x)$在开区间$I$内的每一点都可导,那么就称函数$f(x)$在开区间$I$内可导 (derivatived function);
这时,$\forall x\in I$都存在一个$f(x)$的导数值,这些值构成了一个新的函数,
这个函数交原来函数$y=f(x)$的导函数 (Derived function),记作$y'$,$f'(x)$,$\frac{\D y}{\D x}$或$\frac{\D f(x)}{\D x}$

\bigskip

左导数:$f'_-(x_0)=\lim_{h\to0^-}\frac{f(x_0+h)-f(x_0)}{h}$

右导数:$f'_+(x_0)=\lim_{h\to0^+}\frac{f(x_0+h)-f(x_0)}{h}$

\bigskip

切线方程:$y-y_0=f'(x_0)(x-x_0)$

法线方程:$y-y_0=-\frac{1}{f'(x_0)}(x-x_0)$

\bigskip

\textbf{可导必定连续}
\bigskip
\bigskip

\section*{函数求导法则}

\bigskip

如果函数$u=u(x)$及$v=v(x)$都在点$x$具有导数,那么它们的和、差、积和差(除分母为零的点外)
都在点$x$具有导数,且

\begin{itemize}
  \item $[u(x)\pm v(x)]'=u'(x)\pm v'(x)$
  \item $[u(x)v(x)]'=u'(x)v(x)+u(x)v(x)$
  \item $\big[\frac{u(x)}{v(x)}\big]'=\frac{u'(x)v(x)-u(x)v(x)}{v^2(x)}\quad(v(x)\ne0)$
\end{itemize}

\bigskip

如果函数$x=f(x)$在区间$I_y$内单调、可导且$f'(y)\ne0$,那么它的反函数$y=f^{-1}(x)$在区间
$I_x={x|x=f(y),\;y\in I_y}$内也可导,且
\[ \big[f^{-1}(x)\big]'=\frac{1}{f^{-1}(y)} \]

\bigskip

如果$u=g(x)$在点$x$可导,而$y=f(x)$在点$u=g(x)$可导,那么复合函数$y=f\circ g(x)$在点$x$可导,
且其函数为
\[ \frac{\D y}{\D x}=\frac{\D y}{\D u}\cdot\frac{\D u}{\D x} \]
\bigskip
\bigskip

\section*{常数和基本初等函数的求导公式}

\bigskip

\begin{center}
  \begin{tabular}{ll}
    $(C)'=0'$ & $(x^\mu)'=\mu x^{\mu-1}$\\
    $(\sin x)'=\cos x'$ & $(\cos x)'=-\sin x$\\
    $(\tan x)'=\sec^2x$ & $(\cot x)'=-\csc^2x$\\
    $(\sec x)'=\sec x\tan x$ & $(\csc x)'=-\csc x\cot x$\\
    $(a^x)'=a^x\ln a\quad(a>0,a\ne1)$ & $(e^x)'=e^x$\\
    $(\log_a x)'=\frac{1}{x\ln a}\quad(a>0,a\ne1)$ & $(\ln x)'=\frac{1}{x}$\\
    $(\arcsin x)'=\frac{1}{\sqrt{1-x^2}}$ & $(\arccos x)'=-\frac{1}{\sqrt{1-x^2}}$\\
    $(\arctan x)'=\frac{1}{1+x^2}$ & $(\text{arccot} x)'=\frac{1}{1+x^2}$\\
    $(e^x)^{(n)}=e^x$ & $[\ln(x+1)]^{(n)}=(-1)^{n-1}\frac{(n-1)!}{(x+1)^n}$\\
    $(\sin x)^{(n)}=\sin(x+n\cdot\frac{\pi}{2})$ & $(\cos x)^{(n)}=\cos(x+n\cdot\frac{\pi}{2})$\\
$(x^\mu)^{(n)} =
 \begin{cases}
     \frac{\mu!}{(\mu-n)!} x^{\mu-n}   &  \text{if }\mu>n \\
      n!=\mu !&  \text{if }\mu=n\\
      0   &  \text{if }\mu<n
 \end{cases}$ & $(uv)^{(n)}=\sum_{k=0}^{n}C_n^ku^{n-k}v^k$
  \end{tabular}
\end{center}
\bigskip
\bigskip

\section*{高阶函数}

函数$y=f(x)$的导数$y'=f'(x)$依然时$x$的函数,我们把$y'=f'(x)$的导数叫做函数$y=f(x)$的二阶导数,
记作$y''$或$\frac{\D^2y}{{\D x}^2}$
\[ \frac{\D^2y}{{\D x}^2} = \frac{\D}{\D x}\frac{\D y}{\D x} \]

\textbf{(n-1)阶导数的导数叫做n阶导数,二阶及二阶以上的导数统称高阶导数}

\[y,\,y',\,y'',\,y''',\,y^{(4)},\cdots,y^{(n)}\]
\bigskip
\bigskip

\section*{参数方程求导}

\bigskip

参数方程
$\begin{cases}
x=\varphi(t)\\
y=\psi(t)
\end{cases}$
确定$x$与$y$之间的函数关系,则称此函数关系锁表达的函数为由参数方程所确定的函数。

\[ \frac{\D y}{\D x}=\frac{\D y}{\D t}\cdot\frac{\D t}{\D x}=\frac{\D y}{\D t}\cdot\frac{1}{\frac{\D x}{\D t}}=\frac{\psi'(t)}{\varphi'(t)} \]
\bigskip
\bigskip

\section*{函数的微分}

\bigskip

设函数$y=f(x)$在某区间内有定义,$x_0$及$x_0+\Delta x$在这区间内,如果函数的增量
\[ \Delta y = f(x_0+\Delta)-f(x_0) \]
可表示为
\[ \Delta y = A\Delta x+o(\Delta x) \]
其中$A$是不依赖于$\Delta x$的常数,那么称函数$y=f(x)$在点$x_0$是可微的,而$A\Delta x$
叫做函数$y=f(x)$在点$x_0$相应于自变量增量$\Delta x$的微分,记作$\D y$,即
\[ \D y = A\Delta x \quad A=f'(x_0)\]

\[ \Delta y = \D y + o(\D y) \]
即$\D y$是$\Delta y$的主部;函数$y=f(x)$在任意点$x$的微分,称为函数微分,记作$\D y$或$\D f(x)$,即
\[ \D y = f'(x)\Delta x \]
通常把自变量$x$的增量$\Delta x$称为自变量的微分,记作$\D x$,即$\D x=\Delta x$于是函数$y=f(x)$的微分又可记作
\[ \D y = f'(x)\D x \]

\end{document}

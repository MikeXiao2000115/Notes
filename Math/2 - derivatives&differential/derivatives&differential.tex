\documentclass[UTF8]{ctexart}

\usepackage{geometry}

\usepackage{amsmath}
\usepackage{amssymb}

\usepackage{fancyhdr}
\usepackage{graphicx}

\special{papersize={18.1cm,25.7cm}}
\geometry{left=1.5cm,right=0.5cm,top=2cm,bottom=1cm}
\pagestyle{empty}

\newcommand{\D}{\text{d}}

\begin{document}



\section*{导数 Derivative}

\bigskip

设函数$y=f(x)$在点$x_0$的某个邻域内有定义,当自变量$x$在$x_0$处取得增量$\Delta x$
(点$x_0+\Delta x$仍在该邻域内)时,相应地,因变量取得增量$\Delta y=f(x_0+\Delta x)-f(x_0)$;
如果$\Delta y$与$\Delta x$之比当$\Delta x\to0$时的极限存在,那么称函数$y=f(x)$在点$x_0$处可导,
并称这个极限为函数$y=f(x)$在点$x_0$处的导数 (Derivative),记为$f'(x_0)$,即
\[ f'(x_0)=\lim_{\Delta x\to0}\frac{\Delta y}{\Delta x}=\lim_{\Delta x\to0}\frac{f(x_0+\Delta x)-f(x_0)}{\Delta x} \]
也可记为$y'|_{x=x_0}$,$\frac{\text{d}y}{\text{d}x}\big|_{x=x_0}$或$\frac{\text{d}f(x)}{\text{d}x}\big|_{x=x_0}$

\bigskip

如果函数$f(x)$在开区间$I$内的每一点都可导,那么就称函数$f(x)$在开区间$I$内可导;
这时,$\forall x\in I$都存在一个$f(x)$的导数值,这些值构成了一个新的函数,
这个函数交原来函数$y=f(x)$的导函数,记作$y'$,$f'(x)$,$\frac{\D y}{\D x}$或$\frac{\D f(x)}{\D x}$


\end{document}

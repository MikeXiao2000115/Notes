
\documentclass[UTF8]{ctexart}

\usepackage{geometry}

\usepackage{amsmath}
\usepackage{amssymb}
\usepackage{esint}
\usepackage{yhmath}
\usepackage{bm}

\usepackage{fancyhdr}
\usepackage{graphicx}

\special{papersize={18.1cm,25.7cm}}
\geometry{left=1.5cm,right=0.5cm,top=2cm,bottom=1cm}
\pagestyle{empty}

\setcounter{MaxMatrixCols}{20}

\newcommand{\D}{{\text{d}\;\!}}
\newcommand{\cross}{\times}
\newcommand{\dif}[1]{{\mathrm{d}\;\!#1}}
\newcommand{\dev}[1]{{\frac{\text{d}}{\dif{#1}}\;\!}}
\newcommand{\ve}[1]{{\bm{#1}}}
\newcommand{\mat}[1]{\ve{#1}}
\newcommand{\set}[1]{{\mathbb{#1}}}
\newcommand{\ven}[2]{{\left\langle#1,#2\right\rangle}}
\newcommand{\veN}[3]{{\left\langle#1,#2,#3\right\rangle}}
\newcommand{\ang}[2]{{(\widehat{\ve{#1},\ve{#2}})}}
\newcommand{\abs}[1]{{\left|{#1}\right|}}
\newcommand{\when}[2]{{\left.{#1}\right|_{#2}}}
\newcommand{\dist}[2]{{\left\|\ve{#1}-\ve{#2}\right\|}}
\newcommand{\norm}[1]{{\left\|#1\right\|}}
\newcommand{\emplin}{\vspace{1em}}

\begin{document}

\section*{线性方程组}
\subsection*{消元法}
对于线性方程组
\[\left\{
\begin{aligned}
a_{11}x_1+a_{12}x_2+\cdots+a_{1n}x_n&=b_1\\
a_{21}x_1+a_{22}x_2+\cdots+a_{2n}x_n&=b_2\\
\cdots\cdots\cdots\cdots\cdots\cdots\cdots\cdots\cdots&\cdots\cdots\\
a_{m1}x_1+a_{m2}x_2+\cdots+a_{mn}x_n&=b_m
\end{aligned}
\right.\]
其矩阵形式为
\[\mat{A}\mat{x}=\mat{b}\]
其中
\[\displaystyle\mat{A}=\begin{bmatrix}
a_{11}&a_{12}&\cdots&a_{1n}\\
a_{21}&a_{22}&\cdots&a_{2n}\\
\vdots&\vdots&\ddots&\vdots\\
a_{m1}&a_{m2}&\cdots&a_{mn}
\end{bmatrix},\quad
\displaystyle\mat{x}=\begin{bmatrix}
x_1\\
x_2\\
\vdots\\
x_n
\end{bmatrix},\quad
\displaystyle\mat{b}=\begin{bmatrix}
b_1\\
b_2\\
\vdots\\
b_m
\end{bmatrix}\]
称矩阵$\displaystyle\begin{bmatrix}\mat{A}&\mat{b}\end{bmatrix}$(有时记为$\widetilde{\mat{A}}$)为线性方程组的\textbf{增广矩阵}

当$b_i=0$时,线性方程组称为齐次的,否则称为非齐次的;显然,齐次线性方程组的矩阵形式为
\[\mat{A}\mat{x}=\mat{0}\]

\emplin
\emplin

设$\mat{A}=(a_{ij})_{m\times n}$,$n$元齐次线性方程组$\mat{A}\mat{x}=\mat{0}$有非零解的充要条件是系数矩阵$\mat{A}$的秩$r(\mat{A})<n$

\emplin

设$\mat{A}=(a_{ij})_{m\times n}$,$n$元非齐次线性方程组$\mat{A}\mat{x}=\mat{b}$有解的充要条件是系数矩阵$\mat{A}$的秩等于增广矩阵
$\widetilde{\mat{A}}=\begin{bmatrix}\mat{A}&\mat{b}\end{bmatrix}$的秩,即$r(\mat{A})=r(\widetilde{\mat{A}})$

\emplin

\begin{center}
  \begin{tabular}{cc|cc}
    &$\mat{A}\mat{x}=\mat{b}$&$\mat{A}\mat{x}=\mat{0}$&\\
    \hline
    $r(\mat{A})=r(\widetilde{\mat{A}})=n$&有唯一解&唯一$0$解&$r(\mat{A})=n$\\
    $r(\mat{A})=r(\widetilde{\mat{A}})<n$&欠定方程组(无穷多解)&有非$0$解&$r(\mat{A})<n$\\
    $r(\mat{A})\ne r(\widetilde{\mat{A}})$&矛盾方程组(无解)
  \end{tabular}
\end{center}

有非齐次线性方程组,将增广矩阵$\widetilde{\mat{A}}$化为行阶梯形矩阵,便可直接判断其是否有解,若有解,化为行最简形矩阵,便可直接写出其全部解;
其中要注意,当$r(\mat{A})=r(\widetilde{\mat{A}})<n$时,$\widetilde{\mat{A}}$的行阶梯形矩阵中含有$s$个非零行,
把这$s$行的第一个非零元所对应的未知量作为非自由量,其余$n-s$个作为自由未知量

\section*{向量组的线性组合}
\subsection*{$n$维向量及其线性运算}
$n$个有次序的数$a_1,a_2,\cdots,a_n$所组成的数组称为$n$维向量,这$n$个数称为该向量的$n$个分量,第$i$个数$a_i$称为第$i$个分量

分量全为实数的向量称为实向量,分量为复数的向量称为复向量

$n$为向量可写成一行,也可写成一列;分别称为行向量与列向量,也就是行矩阵与列矩阵,并规定行向量与列向量都按矩阵的运算法则进行计算;因此,
$n$维列向量$\displaystyle\mat{\alpha}=\begin{bmatrix}a_1\\a_2\\\vdots\\a_n\end{bmatrix}$与
$n$维行向量$\displaystyle\mat{\alpha}^T=\begin{bmatrix}a_1&a_2&\cdots&a_n\end{bmatrix}$总被视为是两个不同的向量

通常用黑体小写字母$\mat{\alpha}$,$\mat{\beta}$,$\mat{a}$,$\mat{b}$等表示列向量,用$\mat{\alpha}^T$,$\mat{\beta}^T$,$\mat{a}^T$,$\mat{b}^T$等
表示行向量,所讨论的向量在没有特别指明的情况下都被视为列向量

“空间”通常作为点的集合,称为点空间;$n$维向量的全体所组成的集合$\mat{R}^n=\{ \mat{x}=(x_1,x_2,\cdots,x_n)^T|x_1,x_2,\cdots,x_n\in\mathbb{R} \}$
称为$n$维向量空间

\emplin

若干个同维数的列向量(或行向量)所组成的集合称为\textbf{向量组}
一个$m\times n$矩阵$\displaystyle\mat{A}
\begin{bmatrix}
a_{11}&a_{12}&\cdots&a_{1n}\\
a_{21}&a_{22}&\cdots&a_{2n}\\
\vdots&\vdots&\ddots&\vdots\\
a_{m1}&a_{m2}&\cdots&a_{mn}
\end{bmatrix}$的每一列
\[\mat{\alpha}_j=
\begin{bmatrix}
  a_{1j}\\
  a_{2j}\\
  \vdots\\
  a_{mj}
\end{bmatrix}\]组成的向量组$\mat{\alpha}_1,\quad\mat{\alpha}_2,\quad\cdots,\quad\mat{\alpha}_n$称为矩阵$\mat{A}$的列向量组,
而由矩阵$\mat{A}$的每一行
\[\mat{\beta}_i=
\begin{bmatrix}
  a_{i1},&a_{i2},&\cdots,&a_{in}
\end{bmatrix}\]组成的向量组$\mat{\beta}_1,\quad\mat{\beta}_2,\quad\cdots,\quad\mat{\beta}_m$称为矩阵$\mat{A}$的行向量组

因而矩阵$\mat{A}$可记为
\[\mat{A}=\begin{bmatrix}
  \mat{\alpha}_1&\mat{\alpha}_2&\cdots&\mat{\alpha}_n
\end{bmatrix}=\begin{bmatrix}
  \mat{\beta}_1\\\mat{\beta}_2\\\cdots\\\mat{\beta}_m
\end{bmatrix}\]

\emplin

矩阵的列向量组和行向量组都是只含有限个向量的向量组,而线性方程组
\[\mat{A}\mat{x}=\mat{0}\]
的全部解($\mat{x}$)当$r(\mat{A})<n$时是一个含有无限多个$n$维列向量的向量组

\emplin

两个$n$维向量$\mat{\alpha}=\begin{bmatrix}a_1&a_2&\cdots&a_n\end{bmatrix}^T$与$\mat{\beta}=\begin{bmatrix}b_1&b_2&\cdots&b_n\end{bmatrix}^T$
的各对应分量之和组成的向量,称为向量$\alpha$与$\beta$的和,记为$\mat{\alpha}+\mat{\beta}$,即
\[\mat{\alpha}+\mat{\beta}=\begin{bmatrix}a_1+b_1&a_2+b_2&\cdots&a_n+b_n\end{bmatrix}^T\]

向量的减法
\[\mat{\alpha}-\mat{\beta}=\mat{\alpha}+(-\mat{\beta})=\begin{bmatrix}a_1-b_1&a_2-b_2&\cdots&a_n-b_n\end{bmatrix}^T\]

\emplin

$n$维向量$\mat{\alpha}=\begin{bmatrix}a_1&a_2&\cdots&a_n\end{bmatrix}^T$的各个分量都乘以实数$k$所组成的向量,称为数$k$与向量$\mat{\alpha}$的乘积
(又简称为数乘),记为$k\mat{\alpha}$,即\[\mat{k\alpha}=\begin{bmatrix}ka_1&ka_2&\cdots&ka_n\end{bmatrix}^T\]
向量的加法和数乘运算统称为向量的线性运算

\emplin

向量的线性运算与行(列)矩阵的运算规则相同
\begin{itemize}
  \item $\mat{\alpha}+\mat{\beta}=\mat{\beta}+\mat{\alpha}$
  \item $(\mat{\alpha}+\mat{\beta})+\mat{\gamma}=\mat{\alpha}+(\mat{\beta}+\mat{\gamma})$
  \item $\mat{\alpha}+\mat{0}=\mat{\alpha}$
  \item $\mat{\alpha}+(-\mat{\alpha})=\mat{0}$
  \item $1\mat{\alpha}=\mat{\alpha}$
  \item $k(l\mat{\alpha})=(kl)\mat{\alpha}$
  \item $k(\mat{\alpha}+\mat{\beta})=k\mat{\alpha}+k\mat{\beta}$
  \item $(k+l)\mat{\alpha}=k\mat{\alpha}+l\mat{\alpha}$
\end{itemize}

\emplin

给定向量组$\mat{A}:\mat{\alpha}_1,\mat{\alpha}_2,\cdots,\mat{\alpha}_s$,对于任一组实数$k_1,k_2,\cdots,k_s$,表达式
$k_1\mat{\alpha}_1+k_2\mat{\alpha}_2+\cdots+k_s\mat{\alpha}_s$称为向量组$\mat{A}$的一个线性组合,$k_1,k_2,\cdots,k_s$称为这个线性组合的系数,
也成为该线性组合的权重

\emplin

给定向量组$\mat{A}:\mat{\alpha}_1,\mat{\alpha}_2,\cdots,\mat{\alpha}_s$和向量$\mat{\beta}$,若存在一组数$k_1,k_2,\cdots,k_s$,使
\[\mat{\beta}=k_1\mat{\alpha}_1+k_2\mat{\alpha}_2+\cdots+k_s\mat{\alpha}_s\]
则称向量$\mat{\beta}$是向量组$\mat{A}$的线性组合,又称向量$\mat{\beta}$能由向量组$\mat{A}$线性表示

\emplin

设向量$\mat{\beta}$,$\mat{\alpha}_j$,则向量$\mat{\beta}$能由向量组$\mat{\alpha}_1,\mat{\alpha}_2,\cdots,\mat{\alpha}_s$线性表示的充分必要条件是
矩阵$\mat{A}=\begin{bmatrix}
  \mat{\alpha}_1&\mat{\alpha}_2&\cdots&\mat{\alpha}_n
\end{bmatrix}$与其增广矩阵$\widetilde{\mat{A}}=\begin{bmatrix}
  \mat{\alpha}_1&\mat{\alpha}_2&\cdots&\mat{\alpha}_n&\mat{\beta}
\end{bmatrix}$的秩相同,即
\[r(\mat{A})=r(\widetilde{\mat{A}})\]

\subsection*{向量组间的线性表示}
设有两向量组
\[\mat{A}:\mat{\alpha}_1,\mat{\alpha}_2,\cdots,\mat{\alpha}_s;\mat{B}:\mat{\beta}_1,\mat{\beta}_2,\cdots,\mat{\beta}_t\]
若向量组$\mat{B}$中的每一个向量都能由向量组$\mat{A}$线性表示,则称向量组$\mat{B}$能由向量组$\mat{A}$线性表示;若向量组$\mat{A}$
与向量组$\mat{B}$能相互线性表示,则称这两个向量组\textbf{等价}

即存在$\mat{K}_{s\times t}=(k_{ij})_{s\times t}$使得
\[(\mat{\beta}_1,\mat{\beta}_2,\cdots,\mat{\beta}_t)=(\mat{\alpha}_1,\mat{\alpha}_2,\cdots,\mat{\alpha}_s)
\begin{bmatrix}
  k_{11}&k_{12}&\cdots&k_{1t}\\
  k_{21}&k_{22}&\cdots&k_{2t}\\
  \vdots&\vdots&\ddots&\vdots\\
  k_{s1}&k_{s2}&\cdots&k_{st}
\end{bmatrix}\]
其中矩阵$\mat{K}$称为这一线性表示的\textbf{系数矩阵}

\emplin

若$\mat{C}_{s\times n}=\mat{A}_{s\times t}\mat{B}_{t\times n}$,则矩阵$\mat{C}$的\emph{列向量组}能由矩阵$\mat{A}$的\emph{列向量组}线性表示,
$\mat{B}$为这一表示的系数矩阵;而$\mat{C}$的\emph{行向量组}能由矩阵$\mat{B}$的\emph{行向量组}线性表示,$\mat{A}$为这一表示的系数矩阵

\emplin

若向量组$\mat{A}$可由向量组$\mat{B}$线性表示,向量组$\mat{B}$可由向量组$\mat{C}$线性表示,则向量组$\mat{A}$可由向量组$\mat{C}$线性表示

\section*{向量组的线性相关性}
\subsection*{概念}
给定向量组$\mat{A}:\mat{\alpha}_1,\mat{\alpha}_2,\cdots,\mat{\alpha}_s$如果存在不全为零的数$k_1,k_2,\cdots,k_s$使
\[k_1\mat{\alpha}_1+k_2\mat{\alpha}_2+\cdots+k_s\mat{\alpha}_s=\mat{0}\]
则称向量组$\mat{A}$\textbf{线性相关},否则称为\textbf{线性无关}

由上述可见
\begin{enumerate}
  \item 向量组只含有一个向量$\mat{\alpha}$时,$\mat{\alpha}$线性无关的充分必要条件是$\mat{\alpha}\ne\mat{0}$;
  因此,单个零向量是线性相关的,且任一包含零向量的向量组都是线性相关的
  \item 仅包含两个向量的向量组线性相关的充分必要条件是这两个向量的对应成分成比例
  \item 三个向量线性相关的集合意义是它们共面
\end{enumerate}

\textbf{如果当且仅当$k_1=k_2=\cdots=k_s=0$时,$k_1\mat{\alpha}_1+k_2\mat{\alpha}_2+\cdots+k_s\mat{\alpha}_s=\mat{0}$成立,
则向量组$\mat{\alpha}_1,\mat{\alpha}_2,\cdots,\mat{\alpha}_s$是线性无关的},这是论证一向量组线性无关的基本方法

\subsection*{线性相关性的判定}
向量组$\mat{\alpha}_1,\mat{\alpha}_2,\cdots,\mat{\alpha}_s\quad(s\ge2)$线性相关的充要条件是向量组中至少有一个向量可由其余$s-1$个向量线性表示

设有列向量组$\mat{\alpha}_1,\mat{\alpha}_2,\cdots,\mat{\alpha}_s$,及由该向量组构成的矩阵$\mat{A}=\begin{bmatrix}\mat{\alpha}_1&\mat{\alpha}_2&\cdots&\mat{\alpha}_s\end{bmatrix}$,
则向量组$\mat{\alpha}_1,\mat{\alpha}_2,\cdots,\mat{\alpha}_s$线性相关,就是齐次线性方程组
\[x_1\mat{\alpha}_1+x_2\mat{\alpha}_2+\cdots+x_s\mat{\alpha}_s=0\quad(\mat{A}\mat{x}=\mat{0})\]
有非零解的充要条件是系数矩阵

\emplin
\emplin
\emplin

\textbf{设有列向量组$\mat{\alpha}_j=\begin{bmatrix}a_{1j}\\a_{2j}\\\vdots\\a_{nj}\end{bmatrix}$则向量组$\mat{\alpha}_1,\mat{\alpha}_2,\cdots,\mat{\alpha}_s$
线性相关的充要条件是:矩阵$\mat{A}=\begin{bmatrix}\mat{\alpha}_1&\mat{\alpha}_2&\cdots&\mat{\alpha}_s\end{bmatrix}$的秩小于向量的个数$s$}

$s$个$n$维列向量$\mat{\alpha}_1,\mat{\alpha}_2,\cdots,\mat{\alpha}_s$线性无关(线性相关)的充要条件是:矩阵$\mat{A}=\begin{bmatrix}\mat{\alpha}_1&\mat{\alpha}_2&\cdots&\mat{\alpha}_s\end{bmatrix}$
的秩等于(小于)向量的个数$s$

$n$个$n$维列向量$\mat{\alpha}_1,\mat{\alpha}_2,\cdots,\mat{\alpha}_s$线性无关(线性相关)的充要条件是:矩阵$\mat{A}=\begin{bmatrix}\mat{\alpha}_1&\mat{\alpha}_2&\cdots&\mat{\alpha}_s\end{bmatrix}$
的行列式不等于(等于)零

\textbf{当向量组中所含向量的个数大于向量的维数是,此向量组必定线性相关}

\emplin
\emplin

\textbf{若向量组中有一部分向量(部分组)线性相关,则整个向量组线性相关}

线性无关的向量组中的任一部分组皆线性无关

\emplin
\emplin

\textbf{若向量组$\mat{\alpha}_1,\mat{\alpha}_2,\cdots,\mat{\alpha}_s,\mat{\beta}$线性相关,而向量组$\mat{\alpha}_1,\mat{\alpha}_2,\cdots,\mat{\alpha}_s$线性无关,
则向量$\mat{\beta}$可由$\mat{\alpha}_1,\mat{\alpha}_2,\cdots,\mat{\alpha}_s$线性表示,且表示法唯一}

\emplin
\emplin

\textbf{设有两向量组
\[\mat{A}:\mat{\alpha}_1,\mat{\alpha}_2,\cdots,\mat{\alpha}_s;\quad\mat{B}:\mat{\beta}_1,\mat{\beta}_2,\cdots,\mat{\beta}_t\]
向量组$\mat{B}$能由向量组$\mat{A}$线性表示,若$s<t$,则向量组$\mat{B}$线性相关}

设向量组$\mat{B}$能由向量组$\mat{A}$线性表示,若向量组$\mat{B}$线性无关,则$s\ge t$

设向量组$\mat{A}$与$\mat{B}$可以相互线性表示,若$\mat{A}$与$\mat{B}$都是线性无关的,则$s=t$

\section*{向量组的秩}
\subsection*{极大线性无关向量组}
设有向量组$\mat{A}:\mat{\alpha}_1,\mat{\alpha}_2,\cdots,\mat{\alpha}_s$,
若在向量组$\mat{A}$中能选出$r$个向量$\mat{\alpha}_{j_1},\mat{\alpha}_{j_2},\cdots,\mat{\alpha}_{j_r}$,满足
\begin{enumerate}
  \item 向量组$\mat{A}_0:\mat{\alpha}_{j_1},\mat{\alpha}_{j_2},\cdots,\mat{\alpha}_{j_r}$线性无关
  \item 向量组$\mat{A}$中任意$r+1$个向量(若有的话)均线性相关
\end{enumerate}
则称向量组$\mat{A}_0$是向量组$\mat{A}$的一个\textbf{极大线性无关向量组}

\emplin
\emplin

如果$\mat{\alpha}_{j_1},\mat{\alpha}_{j_2},\cdots,\mat{\alpha}_{j_r}$是$\mat{\alpha}_1,\mat{\alpha}_2,\cdots,\mat{\alpha}_s$的线性无关部分组,
它是极大线性无关组的充分必要条件是$\mat{\alpha}_1,\mat{\alpha}_2,\cdots,\mat{\alpha}_s$中的每个向量都可由$\mat{\alpha}_{j_1},\mat{\alpha}_{j_2},\cdots,\mat{\alpha}_{j_r}$
线性表示

\subsection*{向量组的秩}
向量组$\mat{\alpha}_1,\mat{\alpha}_2,\cdots,\mat{\alpha}_s$的极大无关组所含向量的个数称为该向量组的秩,记为
\[r(\mat{\alpha}_1,\mat{\alpha}_2,\cdots,\mat{\alpha}_s)\]
\textbf{规定由零向量组成的向量组的秩为$0$}

\subsection*{矩阵与向量组秩的关系}
\textbf{设$\mat{A}$为$m\times n$矩阵,则矩阵$\mat{A}$的秩等于它的列向量组的秩,也等于它的行向量组的秩}

矩阵$\mat{A}$的行向量组的秩与其列向量组的秩相等

若$D_s$是矩阵$\mat{A}$的一个最高阶非零子式,则$D_s$所在的$s$列就是$\mat{A}$的列向量组的一个极大无关组;$D_s$所在的$s$行即是$\mat{A}$的行向量组的一个极大无关组

\textbf{若对矩阵$\mat{A}$施以初等行变换得到矩阵$\mat{B}$,则矩阵$\mat{B}$的列向量组与$\mat{A}的列向量组由相同的线性关系$}

\emplin
\emplin

\textbf{若向量组$\mat{B}$能由向量组$\mat{A}$表示,则$r(\mat{B})\le r(\mat{A})$}

等价的向量组的秩相等

设向量组$\mat{B}$是向量组$\mat{A}$的部分组,若向量组$\mat{B}$线性无关,且向量组$\mat{A}$能由向量组$\mat{B}$线性表示,
则向量组$\mat{B}$是向量组$\mat{A}$的一个极大线性无关组

\section*{向量空间}
设$\set{V}$为$n$维向量的集合,若集合$\set{V}$非空,且集合$\set{V}$对$n$维向量的加法及数乘两种运算封闭,即
\begin{itemize}
  \item 若$\mat{\alpha}\in\set{V}$,$\mat{\beta}\in\set{V}$,则$\mat{\alpha}+\mat{\beta}\in\set{V}$
  \item 若$\mat{\alpha}\in\set{V}$,$\lambda\in\set{R}$,则$\lambda\mat{\alpha}\in\set{V}$
\end{itemize}
则称集合$\set{V}$为$\set{R}$上的向量空间

\textbf{记所有$n$为向量的集合为$\set{R}^n$,而集合$\set{R}^n$构成一向量空间,称$\set{R}^n$为$n$为向量空间}

\emplin

由向量组$\mat{\alpha}_1,\mat{\alpha}_2,\cdots,\mat{\alpha}_m$所生成的向量空间记为
\[\set{V}=\{ \mat{\xi}=\lambda_1\mat{\alpha}_1+\lambda_2\mat{\alpha}_2+\cdots+\lambda_m\mat{\alpha}_m|\lambda_1,\lambda_2,\cdots,\lambda_m\in\set{R} \}\]

\emplin

集合$\set{S}=\{ \mat{\alpha}|\mat{A}\mat{\alpha}=\mat{0} \}$是一向量空间,称其为齐次线性方程组$\mat{A}\mat{x}=\mat{0}$的\textbf{解空间}

\emplin
\emplin

设由向量空间$\set{V}_1$与$\set{V}_2$,若向量空间$\set{V}_1\subset\set{V}_2$,则称$\set{V}_1$是$\set{V}_2$的子空间

\subsection*{向量空间的基与维数}
设$\set{V}$是向量空间,若有$r$个向量$\mat{\alpha}_1,\mat{\alpha}_2,\cdots,\mat{\alpha}_r\in\set{V}$,且满足
\begin{itemize}
  \item $\mat{\alpha}_1,\mat{\alpha}_2,\cdots,\mat{\alpha}_r$线性无关
  \item $\set{V}$中任一向量都可有$\mat{\alpha}_1,\mat{\alpha}_2,\cdots,\mat{\alpha}_r$线性表示
\end{itemize}
则称向量组$\mat{\alpha}_1,\mat{\alpha}_2,\cdots,\mat{\alpha}_r$为向量空间$\set{V}$的一个基,数$r$称为向量空间$\set{V}$的维数,记为
$\text{dim}\set{V}=r$,并称$\set{V}$为$r$维向量空间

\begin{enumerate}
  \item 只含零向量的向量空间称为\textbf{$0$维向量空间},它没有基
  \item 若把向量空间$\set{V}$看作向量组,则$\set{V}$的基就是向量组的极大无关组,$\set{V}$的维数就是向量组的秩
  \item 若向量组$\mat{\alpha}_1,\mat{\alpha}_2,\cdots,\mat{\alpha}_r$为向量空间$\set{V}$的一个基,则$\set{V}$可表示为
  \[\set{V}=\{ \mat{x}|\mat{x}=\lambda_1\mat{\alpha}_1+\lambda_2\mat{\alpha}_2+\cdots+\lambda_r\mat{\alpha}_r|\lambda_1,\lambda_2,\cdots,\lambda_r\in\set{R} \}\]
  此时,$\set{V}$又称为\textbf{由基$\mat{\alpha}_1,\mat{\alpha}_2,\cdots,\mat{\alpha}_r$所生成的向量空间}
\end{enumerate}

\emplin
\emplin

\textbf{如果在向量空间$\set{V}$中取定一个基$\mat{\alpha}_1,\mat{\alpha}_2,\cdots,\mat{\alpha}_r$,那么$\set{V}$中任一向量$\mat{x}$可唯一地表示为
\[\mat{x}=\lambda_1\mat{\alpha}_1+\lambda_2\mat{\alpha}_2+\cdots+\lambda_r\mat{\alpha}_r\]
有序数组$\lambda_1,\lambda_2,\cdots,\lambda_r$称为向量$\mat{x}$在基$\mat{\alpha}_1,\mat{\alpha}_2,\cdots,\mat{\alpha}_r$下的坐标}

特别地,在$n$维向量空间$\set{R}^n$中取单位坐标向量组$\mat{\epsilon}_1,\mat{\epsilon}_2,\cdots,\mat{\epsilon}_n$为基,则以
$x_1,x_2,\cdots,x_n$为分量的向量$\mat{x}$可表示为
\[\mat{x}=x_1\mat{\epsilon}_1+x_2\mat{\epsilon}_2+\cdots+x_n\mat{\epsilon}_n\]
可见向量在基$\mat{\epsilon}_1,\mat{\epsilon}_2,\cdots,\mat{\epsilon}_n$下的坐标就是该向量的分量;
$\mat{\epsilon}_1,\mat{\epsilon}_2,\cdots,\mat{\epsilon}_n$称为$\set{R}^n$中的\textbf{自然基}

一个集合的坐标系就是这个集合中的点到$\set{R}^n$的一个一对一映射

\section*{线性方程组解的结构}
\subsection*{齐次线性方程组解的结构}
对于矩阵方程$\mat{A}\mat{x}=\mat{0}$,称向量$\mat{x}$为其\textbf{解向量}

\emplin

其具有性质
\begin{enumerate}
  \item 若$\mat{\xi}_1$,$\mat{\xi}_2$为矩阵方程的解向量,则$\mat{\xi}_1+\mat{\xi}_2$也是其解向量
  \item 若$\mat{\xi}_1$是矩阵方程的解向量,$k$为实数,则$k\mat{\xi}_1$也是其解向量
  \item 若$\mat{\xi}_1,\mat{\xi}_2,\cdots,\mat{\xi}_s$是矩阵方程的解向量,则由它们为基构成的向量空间$\set{S}$中的任一向量均为其解,
  称此向量空间为$\mat{A}\mat{x}=\mat{0}$的\textbf{解空间}
\end{enumerate}

\emplin
\emplin

若齐次线性方程组$\mat{A}\mat{x}=\mat{0}$的有限个解$\mat{\eta}_1,\mat{\eta}_2,\cdots,\mat{\eta}_t$满足
\begin{itemize}
  \item $\mat{\eta}_1,\mat{\eta}_2,\cdots,\mat{\eta}_t$线性无关
  \item $\mat{A}\mat{x}=\mat{0}$的任一解均可由$\mat{\eta}_1,\mat{\eta}_2,\cdots,\mat{\eta}_t$线性表示
\end{itemize}
则称$\mat{\eta}_1,\mat{\eta}_2,\cdots,\mat{\eta}_t$是齐次线性方程组$\mat{A}\mat{x}=\mat{0}$的一个\textbf{基础解系}

\emplin
\emplin

对于基础解系$\mat{\eta}_1,\mat{\eta}_2,\cdots,\mat{\eta}_t$,$\mat{A}\mat{x}=\mat{0}$的全部解可表示为
\[c_1\mat{\eta}_1+c_2\mat{\eta}_2+\cdots+c_t\mat{\eta}_t\quad c_1,c_2,\cdots,c_t\in\set{R}\]
该表达式称为线性方程组$\mat{A}\mat{x}=\mat{0}$的\textbf{通解}

\emplin
\emplin
\emplin

对于齐次线性方程组$\mat{A}\mat{x}=\mat{0}$,若$r(\mat{A})=r<n$,则该方程组的基础解系一定存在,且每个基础解系中所含解向量的个数均等于$n-r$,
其中$n$是方程组所含未知数的个数

因为$r(\mat{A})=r<n$,故对矩阵$\mat{A}$施以初等行变换,可化为如下形式
\[\mat{B}=\begin{bmatrix}
1&0&\cdots&0&b_{11}&b_{12}&\cdots&b_{1{n-r}}\\
0&1&\cdots&0&b_{21}&b_{22}&\cdots&b_{2{n-r}}\\
\vdots&\vdots&\ddots&\vdots&\vdots&\vdots&\ddots&\vdots\\
0&0&\cdots&1&b_{r1}&b_{r2}&\cdots&b_{r{n-r}}\\
0&0&\cdots&0&0&0&\cdots&0\\
\vdots&\vdots&\ddots&\vdots&\vdots&\vdots&\ddots&\vdots\\
0&0&\cdots&0&0&0&\cdots&0
\end{bmatrix}\]
则方程组的基础解系为
\[\mat{\eta}_1=\begin{bmatrix}
-b_{11}\\
\vdots\\
-b_{r1}\\
1\\
0\\
\vdots\\
0
\end{bmatrix},\quad
\mat{\eta}_2=\begin{bmatrix}
-b_{12}\\
\vdots\\
-b_{r2}\\
0\\
1\\
\vdots\\
0
\end{bmatrix},
\cdots,\quad
\mat{\eta}_{n-r}=\begin{bmatrix}
-b_{1{n-r}}\\
\vdots\\
-b_{r{n-r}}\\
0\\
0\\
\vdots\\
1
\end{bmatrix},\quad
\]

\emplin
\emplin
\emplin

设$\mat{A}$为$m\times n$矩阵,则$n$元齐次线性方程组$\mat{A}\mat{x}=\mat{0}$的全体解构成的集合$\set{S}$是一个向量空间,称其为该方程组的解空间;
当系数矩阵的秩$r(\mat{A})=r$时,解空间$\set{S}$的维数为$n-r$,当$r(\mat{A})=n$时,方程组$\mat{A}\mat{x}=\mat{0}$只有零解,此时解空间$\set{S}$
只含有一个零向量,解空间$\set{S}$的维数为$0$,当$r(\mat{A})=r<n$时,方程组的任一解$\mat{x}$可以表示为
\[\mat{x}=c_1\mat{\eta}_1+c_2\mat{\eta}_2+\cdots+c_{n-r}\mat{\eta}_{n-r}\quad,c_1,c_2,\cdots,c_{n-r}\in\set{R}\]
而解空间$\set{S}$可以表示为
\[\set{V}=\{ \mat{x}|\mat{x}=c_1\mat{\eta}_1+c_2\mat{\eta}_2+\cdots+c_{n-r}\mat{\eta}_{n-r}\quad,c_1,c_2,\cdots,c_{n-r}\in\set{R} \}\]

\subsection*{非齐次线性方程组解的结构}
非齐次线性方程组可写作向量方程$\mat{A}\mat{x}=\mat{b}$,称$\mat{A}\mat{x}=\mat{0}$为$\mat{A}\mat{x}=\mat{b}$对应的齐次线性方程组(也称\textbf{导出组})

\emplin
\emplin

若$\mat{\eta}_1$,$\mat{\eta}_2$是非齐次线性方程组$\mat{A}\mat{x}=\mat{b}$的解,则$\mat{\eta}_1-\mat{\eta}_2$是对应的齐次线性方程组$\mat{A}\mat{x}=\mat{0}$的解

\emplin

设$\mat{\eta}$是非齐次线性方程组$\mat{A}\mat{x}=\mat{b}$的解,$\mat{\xi}$为对应的齐次线性方程组$\mat{A}\mat{x}=\mat{0}$的解,
则$\mat{\xi}+\mat{\eta}$为非齐次线性方程组$\mat{A}\mat{x}=\mat{b}$的解

\emplin

\textbf{设$\mat{\eta}^*$是非齐次线性方程组$\mat{A}\mat{x}=\mat{b}$的一个解,$\mat{\xi}$是对应齐次线性方程组$\mat{A}\mat{x}=\mat{0}$的通解,
则$\mat{x}=\mat{\xi}+\mat{\eta}^*$是非齐次线性方程组$\mat{A}\mat{x}=\mat{b}$的通解}

设$\mat{\eta}_1,\mat{\eta}_2,\cdots,\mat{\eta}_{n-r}$是$\mat{A}\mat{x}=\mat{0}$的基础解系,$\mat{\eta}^*$是$\mat{A}\mat{x}=\mat{b}$
的一个解,则非齐次线性方程组$\mat{A}\mat{x}=\mat{b}$的通解可表示为
\[\mat{x}=c_1\mat{\eta}_1+c_2\mat{\eta}_2+\cdots+c_{n-r}\mat{\eta}_{n-r}+\mat{\eta}^*\]
其中$c_1,c_2,\cdots,c_{n-r}\in\set{R}$

\emplin
\emplin

综上所述,设非齐次线性方程组$\mat{A}\mat{x}=\mat{b}$,而$\mat{\alpha}_1,\mat{\alpha}_2,\cdots,\mat{\alpha}_n$是系数矩阵$\mat{A}$的列向量组,
则一下四个命题等价
\begin{itemize}
  \item 非齐次线性方程组$\mat{A}\mat{x}=\mat{b}$等价
  \item 向量$\mat{b}$能由向量组$\mat{\alpha}_1,\mat{\alpha}_2,\cdots,\mat{\alpha}_n$线性表示
  \item 向量组$\mat{\alpha}_1,\mat{\alpha}_2,\cdots,\mat{\alpha}_n$与向量组$\mat{\alpha}_1,\mat{\alpha}_2,\cdots,\mat{\alpha}_n,\mat{b}$等价
  \item $r(\mat{A})=r(\mat{A}\quad b)$
\end{itemize}

\section*{线性方程组的应用}
\subsection*{网络流模型}
一个\textbf{网络}由一个点集以及连接部分或全部点的直线或弧线构成,网络中的点称作\textbf{联结点}(或\textbf{节点}),网络中的连接线称为\textbf{分支};
每个分支中的流量方向已经指定,并且流量(或流速)已知或以标为变量;\textbf{网络流的基本假设是网络中流入与流出的总量相等,并且每个节点的流入与流出总量也相等},
因而将节点的流入流出量冠以不同符号,每个节点可对应为以方程,进而组成一齐次线性方程组
\subsection*{递归方程}
某些量常按离散时间间隔来测量,这样就产生了与时间间隔相应的向量序列$\mat{x}_0,\mat{x}_1,\mat{x}_2,\cdots,\mat{x}_n,\cdots$其中$\mat{x}_n$表示第$n$次测量时
系统的相关信息,而$\mat{x}_0$常被称为\textbf{初始向量}

如果存在矩阵$\mat{A}$,并给定初始向量$\mat{x}_0$,使得$\mat{x}_1=\mat{A}\mat{x}_0,\quad\mat{x}_2=\mat{A}\mat{x}_1,\cdots$,即
\[\mat{x}_{n+1}=\mat{A}\mat{x}_n\]
则称其为一个\textbf{线性差分方程}或者\textbf{递归方程},有时称$\mat{A}$为\textbf{迁移矩阵}
\subsection*{电网模型}
\[\mat{u}=\mat{R}\mat{i}\]

\end{document}

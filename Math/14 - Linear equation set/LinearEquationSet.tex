\documentclass[UTF8]{ctexart}

\usepackage{geometry}

\usepackage{amsmath}
\usepackage{amssymb}
\usepackage{esint}
\usepackage{yhmath}
\usepackage{bm}

\usepackage{fancyhdr}
\usepackage{graphicx}

\special{papersize={18.1cm,25.7cm}}
\geometry{left=1.5cm,right=0.5cm,top=2cm,bottom=1cm}
\pagestyle{empty}

\setcounter{MaxMatrixCols}{20}

\newcommand{\D}{{\text{d}\;\!}}
\newcommand{\cross}{\times}
\newcommand{\dif}[1]{{\mathrm{d}\;\!#1}}
\newcommand{\dev}[1]{{\frac{\text{d}}{\dif{#1}}\;\!}}
\newcommand{\ve}[1]{{\bm{#1}}}
\newcommand{\mat}[1]{\ve{#1}}
\newcommand{\ven}[2]{{\left\langle#1,#2\right\rangle}}
\newcommand{\veN}[3]{{\left\langle#1,#2,#3\right\rangle}}
\newcommand{\ang}[2]{{(\widehat{\ve{#1},\ve{#2}})}}
\newcommand{\abs}[1]{{\left|{#1}\right|}}
\newcommand{\when}[2]{{\left.{#1}\right|_{#2}}}
\newcommand{\dist}[2]{{\left\|\ve{#1}-\ve{#2}\right\|}}
\newcommand{\norm}[1]{{\left\|#1\right\|}}
\newcommand{\emplin}{\vspace{1em}}

\begin{document}

\section*{线性方程组}
\subsection*{消元法}
对于线性方程组
\[\left\{
\begin{aligned}
a_{11}x_1+a_{12}x_2+\cdots+a_{1n}x_n&=b_1\\
a_{21}x_1+a_{22}x_2+\cdots+a_{2n}x_n&=b_2\\
\cdots\cdots\cdots\cdots\cdots\cdots\cdots\cdots\cdots&\cdots\cdots\\
a_{m1}x_1+a_{m2}x_2+\cdots+a_{mn}x_n&=b_m
\end{aligned}
\right.\]
其矩阵形式为
\[\mat{A}\mat{x}=\mat{b}\]
其中
\[\displaystyle\mat{A}=\begin{bmatrix}
a_{11}&a_{12}&\cdots&a_{1n}\\
a_{21}&a_{22}&\cdots&a_{2n}\\
\vdots&\vdots&\ddots&\vdots\\
a_{m1}&a_{m2}&\cdots&a_{mn}
\end{bmatrix},\quad
\displaystyle\mat{x}=\begin{bmatrix}
x_1\\
x_2\\
\vdots\\
x_n
\end{bmatrix},\quad
\displaystyle\mat{b}=\begin{bmatrix}
b_1\\
b_2\\
\vdots\\
b_m
\end{bmatrix}\]
称矩阵$\displaystyle\begin{bmatrix}\mat{A}&\mat{b}\end{bmatrix}$(有时记为$\widetilde{\mat{A}}$)为线性方程组的\textbf{增广矩阵}

当$b_i=0$时,线性方程组称为齐次的,否则称为非齐次的;显然,齐次线性方程组的矩阵形式为
\[\mat{A}\mat{x}=\mat{0}\]

\emplin
\emplin

设$\mat{A}=(a_{ij})_{m\times n}$,$n$元齐次线性方程组$\mat{A}\mat{x}=\mat{0}$有非零解的充要条件是系数矩阵$\mat{A}$的秩$r(\mat{A})<n$

\emplin

设$\mat{A}=(a_{ij})_{m\times n}$,$n$元非齐次线性方程组$\mat{A}\mat{x}=\mat{b}$有解的充要条件是系数矩阵$\mat{A}$的秩等于增广矩阵
$\widetilde{\mat{A}}=\begin{bmatrix}\mat{A}&\mat{b}\end{bmatrix}$的秩,即$r(\mat{A})=r(\widetilde{\mat{A}})$

\emplin

\begin{center}
  \begin{tabular}{cc|cc}
    &$\mat{A}\mat{x}=\mat{b}$&$\mat{A}\mat{x}=\mat{0}$&\\
    \hline
    $r(\mat{A})=r(\widetilde{\mat{A}})=n$&有唯一解&唯一$0$解&$r(\mat{A})=n$\\
    $r(\mat{A})=r(\widetilde{\mat{A}})<n$&欠定方程组(无穷多解)&有非$0$解&$r(\mat{A})<n$\\
    $r(\mat{A})\ne r(\widetilde{\mat{A}})$&矛盾方程组(无解)
  \end{tabular}
\end{center}

有非齐次线性方程组,将增广矩阵$\widetilde{\mat{A}}$化为行阶梯形矩阵,便可直接判断其是否有解,若有解,化为行最简形矩阵,便可直接写出其全部解;
其中要注意,当$r(\mat{A})=r(\widetilde{\mat{A}})<n$时,$\widetilde{\mat{A}}$的行阶梯形矩阵中含有$s$个非零行,
把这$s$行的第一个非零元所对应的未知量作为非自由量,其余$n-s$个作为自由未知量

\section*{向量组的线性组合}
\subsection*{$n$维向量及其线性运算}
$n$个有次序的数$a_1,a_2,\cdots,a_n$所组成的数组称为$n$维向量,这$n$个数称为该向量的$n$个分量,第$i$个数$a_i$称为第$i$个分量

分量全为实数的向量称为实向量,分量为复数的向量称为复向量

$n$为向量可写成一行,也可写成一列;分别称为行向量与列向量,也就是行矩阵与列矩阵,并规定行向量与列向量都按矩阵的运算法则进行计算;因此,
$n$维列向量$\displaystyle\mat{\alpha}=\begin{bmatrix}a_1\\a_2\\\vdots\\a_n\end{bmatrix}$与
$n$维行向量$\displaystyle\mat{\alpha}^T=\begin{bmatrix}a_1&a_2&\cdots&a_n\end{bmatrix}$总被视为是两个不同的向量

通常用黑体小写字母$\mat{\alpha}$,$\mat{\beta}$,$\mat{a}$,$\mat{b}$等表示列向量,用$\mat{\alpha}^T$,$\mat{\beta}^T$,$\mat{a}^T$,$\mat{b}^T$等
表示行向量,所讨论的向量在没有特别指明的情况下都被视为列向量

“空间”通常作为点的集合,称为点空间;$n$维向量的全体所组成的集合$\mat{R}^n=\{ \mat{x}=(x_1,x_2,\cdots,x_n)^T|x_1,x_2,\cdots,x_n\in\mathbb{R} \}$
称为$n$维向量空间

\emplin

若干个同维数的列向量(或行向量)所组成的集合称为\textbf{向量组}
一个$m\times n$矩阵$\displaystyle\mat{A}
\begin{bmatrix}
a_{11}&a_{12}&\cdots&a_{1n}\\
a_{21}&a_{22}&\cdots&a_{2n}\\
\vdots&\vdots&\ddots&\vdots\\
a_{m1}&a_{m2}&\cdots&a_{mn}
\end{bmatrix}$的每一列
\[\mat{\alpha}_j=
\begin{bmatrix}
  a_{1j}\\
  a_{2j}\\
  \vdots\\
  a_{mj}
\end{bmatrix}\]组成的向量组$\mat{\alpha}_1,\quad\mat{\alpha}_2,\quad\cdots,\quad\mat{\alpha}_n$称为矩阵$\mat{A}$的列向量组,
而由矩阵$\mat{A}$的每一行
\[\mat{\beta}_i=
\begin{bmatrix}
  a_{i1},&a_{i2},&\cdots,&a_{in}
\end{bmatrix}\]组成的向量组$\mat{\beta}_1,\quad\mat{\beta}_2,\quad\cdots,\quad\mat{\beta}_m$称为矩阵$\mat{A}$的行向量组

因而矩阵$\mat{A}$可记为
\[\mat{A}=\begin{bmatrix}
  \mat{\alpha}_1&\mat{\alpha}_2&\cdots&\mat{\alpha}_n
\end{bmatrix}=\begin{bmatrix}
  \mat{\beta}_1\\\mat{\beta}_2\\\cdots\\\mat{\beta}_m
\end{bmatrix}\]

\emplin

矩阵的列向量组和行向量组都是只含有限个向量的向量组,而线性方程组
\[\mat{A}\mat{x}=\mat{0}\]
的全部解($\mat{x}$)当$r(\mat{A})<n$时是一个含有无限多个$n$维列向量的向量组

\emplin

两个$n$维向量$\mat{\alpha}=\begin{bmatrix}a_1&a_2&\cdots&a_n\end{bmatrix}^T$与$\mat{\beta}=\begin{bmatrix}b_1&b_2&\cdots&b_n\end{bmatrix}^T$
的各对应分量之和组成的向量,称为向量$\alpha$与$\beta$的和,记为$\mat{\alpha}+\mat{\beta}$,即
\[\mat{\alpha}+\mat{\beta}=\begin{bmatrix}a_1+b_1&a_2+b_2&\cdots&a_n+b_n\end{bmatrix}^T\]

向量的减法
\[\mat{\alpha}-\mat{\beta}=\mat{\alpha}+(-\mat{\beta})=\begin{bmatrix}a_1-b_1&a_2-b_2&\cdots&a_n-b_n\end{bmatrix}^T\]

\emplin

$n$维向量$\mat{\alpha}=\begin{bmatrix}a_1&a_2&\cdots&a_n\end{bmatrix}^T$的各个分量都乘以实数$k$所组成的向量,称为数$k$与向量$\mat{\alpha}$的乘积
(又简称为数乘),记为$k\mat{\alpha}$,即\[\mat{k\alpha}=\begin{bmatrix}ka_1&ka_2&\cdots&ka_n\end{bmatrix}^T\]
向量的加法和数乘运算统称为向量的线性运算

\emplin

向量的线性运算与行(列)矩阵的运算规则相同
\begin{itemize}
  \item $\mat{\alpha}+\mat{\beta}=\mat{\beta}+\mat{\alpha}$
  \item $(\mat{\alpha}+\mat{\beta})+\mat{\gamma}=\mat{\alpha}+(\mat{\beta}+\mat{\gamma})$
  \item $\mat{\alpha}+\mat{0}=\mat{\alpha}$
  \item $\mat{\alpha}+(-\mat{\alpha})=\mat{0}$
  \item $1\mat{\alpha}=\mat{\alpha}$
  \item $k(l\mat{\alpha})=(kl)\mat{\alpha}$
  \item $k(\mat{\alpha}+\mat{\beta})=k\mat{\alpha}+k\mat{\beta}$
  \item $(k+l)\mat{\alpha}=k\mat{\alpha}+l\mat{\alpha}$
\end{itemize}

\emplin

给定向量组$\mat{A}:\mat{\alpha}_1,\mat{\alpha}_2,\cdots,\mat{\alpha}_s$,对于任一组实数$k_1,k_2,\cdots,k_s$,表达式
$k_1\mat{\alpha}_1+k_2\mat{\alpha}_2+\cdots+k_s\mat{\alpha}_s$称为向量组$\mat{A}$的一个线性组合,$k_1,k_2,\cdots,k_s$称为这个线性组合的系数,
也成为该线性组合的权重




\end{document}

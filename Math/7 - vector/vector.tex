\documentclass[UTF8]{ctexart}

\usepackage{geometry}

\usepackage{amsmath}
\usepackage{amssymb}

\usepackage{bm}

\usepackage{fancyhdr}
\usepackage{graphicx}

\special{papersize={18.1cm,25.7cm}}
\geometry{left=1.5cm,right=0.5cm,top=2cm,bottom=1cm}
\pagestyle{empty}

\newcommand{\D}{{\text{d}\;\!}}
\newcommand{\cross}{\times}
\newcommand{\dif}[1]{{\text{d}\;\!#1}}
\newcommand{\dev}[1]{{\frac{\text{d}}{\dif{#1}}\;\!}}
\newcommand{\ve}[1]{{\bm{#1}}}
\newcommand{\ven}[2]{{\left\langle#1,#2\right\rangle}}
\newcommand{\veN}[3]{{\left\langle#1,#2,#3\right\rangle}}
\newcommand{\ang}[2]{{(\widehat{\ve{#1},\ve{#2}})}}
\newcommand{\abs}[1]{{\left|{#1}\right|}}
\newcommand{\when}[2]{{\left.{#1}\right|_{#2}}}

\begin{document}

\section*{向量 vector}

既有大小,又有方向的量称为向量(矢量),以$\bf{a}$、$\bf{b}$、$\bf{r}$、$\bf{F}$
(或$\vec{a}$、$\vec{b}$、$\vec{r}$、$\vec{F}$)表示

如果向量$\ve{a}$与$\ve{b}$在经过平移后可以完全重合(方向,大小均相同),则称它们相等,
记作$\ve{a}=\ve{b}$

向量的大小叫做向量的模;向量$\ve{a}$的模记为$|\ve{a}|$;
模等于$1$的向量叫做单位向量,一般用$\ve{e}$表示;
模等一$0$的向量叫做零向量,记作$\ve{0}$或$\vec{0}$

向量$\ve{a}$与$\ve{b}$之间的夹角记为$(\widehat{\ve{a},\ve{b}})$或$(\widehat{\ve{b},\ve{a}})$;
如果$\ang{a}{b}=0$,那么就称$\ve{a}$与$\ve{b}$平行,记作$\ve{a}\parallel\ve{b}$;
如果$\ang{a}{b}=\frac{\pi}{2}$,那么就称$\ve{a}$与$\ve{b}$垂直,记作$\ve{a}\perp\ve{b}$;
$\ve{0}$与任意向量平行,也与任意向量垂直

如果$k$个向量的起点为同一点时,它们的终点与起点(共$k+1$点)共线,则称它们为共线;
如果$k\;(k\ge3)$个向量的起点为同一点时,它们的终点与起点共面,则称它们共面
\bigskip
\bigskip

\section*{向量的运算}
\subsection*{加法}

设有两向量$\ve{a}$与$\ve{b}$,它们的和$\ve{c}=\ve{a}+\ve{b}$为与$\ve{a}$共起点,与$\ve{b}$共终点的一向量,
而$\ve{a}$的终点与$\ve{b}$的起点共点

向量加法满足{\bf 交换律}($\ve{a}+\ve{b}=\ve{b}+\ve{a}$)和{\bf 结合律}($(\ve{a}+\ve{b})+\ve{c}=\ve{a}+(\ve{b}+\ve{c})$)

\subsection*{负 \& 减法}


设$\ve{a}$为一向量,与$\ve{a}$大小相同,但方向相反的向量叫做$\ve{a}$的负向量,记作$-\ve{a}$

我们规定向量$\ve{b}$与向量$\ve{a}$的差$\ve{b}-\ve{a}=\ve{b}+(-\ve{a})$

\subsection*{与标量的积}


向量$\ve{a}$与实数$\lambda$的乘积记作$\lambda\ve{a}$,它是一个向量,
如果$\lambda$为正,则与$\ve{a}$同向,否则与$-\ve{a}$同向;
其模$|\lambda\ve{a}|=|\lambda||\ve{a}|$

向量与标量的乘法满足{\bf 结合律}($\lambda(\mu\ve{a})=(\lambda\mu)\ve{a}$)和{\bf 分配律}($(\lambda+\mu)\ve{a}=\lambda\ve{a}+\mu\ve{a}$且$\lambda(\ve{a}+\ve{b})=\lambda\ve{a}+\lambda\ve{b}$)

\bigskip

设向量$\ve{a}\ne0$,则向量$\ve{b}$平行于$\ve{a}$的充分必要条件是:存在唯一的实数$\lambda$,使$\ve{b}=\lambda\ve{a}$

\subsection*{点乘(数量积)}

定义两向量的数量积为
\[\ve{a}\cdot\ve{b}=|\ve{a}||\ve{b}|\cos\ang{a}{b}=|\ve{a}|\text{Prj}_{\ve{a}} \ve{b}=|\ve{b}|\text{Prj}_{\ve{b}}\ve{a}\]

\[\ve{a}\cdot\ve{a}=|a|^2\]
\[\ve{a}\cdot\ve{b}=0\quad \text{if }\ve{a}\perp\ve{b}\]
\[\cos\ang{a}{b}=\frac{\ve{a}\cdot\ve{b}}{|\ve{a}||\ve{b}|}\]

数量积满足{\bf 交换律}($\ve{a}\cdot\ve{b}=\ve{b}\cdot\ve{a}$)和{\bf 分配律}($(\ve{a}+\ve{b})\cdot\ve{c}=\ve{a}\cdot\ve{c}+\ve{b}\cdot\ve{c}$)
以及{\bf 结合律}($(\lambda\ve{a})\cdot\ve{b}=\lambda(\ve{a}\cdot\ve{b})$)

\subsection*{叉乘}
\[\ve{a}\times\ve{b}=\begin{vmatrix} \ve{i}&\ve{j}&\ve{k}\\a_x&a_y&a_z\\b_x&b_y&b_z\end{vmatrix}\]

\[\ve{a}\cross\ve{a}=\vec{0}\]
\[\ve{a}\cdot\ve{b}=0\quad \text{if }\ve{a}\parallel\ve{b}\]
\[\ve{a}\cross\ve{b}=-\ve{b}\cross\ve{a}\]
\[(\ve{a}+\ve{b})\cross\ve{c}=\ve{a}\cross\ve{c}+\ve{b}\cross\ve{c}\]
\[(\lambda\ve{a})\cross\ve{b}=\lambda(\ve{a}\cross\ve{b})=\ve{a}\cross(\lambda\ve{b})\]

\subsection*{混合积}
已知三个变量$\ve{a}$、$\ve{b}$与$\ve{c}$,定义它们的混合积$[\ve{a}\ve{b}\ve{c}]$为$[\ve{a}\ve{b}\ve{c}]=(\ve{a}\cross\ve{b})\cdot\ve{c}$
\[[\ve{a}\ve{b}\ve{c}]=\begin{vmatrix} c_x&c_y&c_z\\a_x&a_y&a_z\\b_x&b_y&b_z\end{vmatrix}=\begin{vmatrix} a_x&a_y&a_z\\b_x&b_y&b_z\\c_x&c_y&c_z\end{vmatrix}\]

\bigskip
\bigskip

\section*{平面及其方程}

如果曲面$S$与三元方程
\[F(x,y,z)=0\]
满足
\begin{itemize}
  \item 在曲面$S$上的任一点的坐标均满足$F(x,y,z)=0$
  \item 不在曲面$S$上的任一点的坐标均满足$F(x,y,z)\ne0$
\end{itemize}
则称$F(x,y,z)=0$为曲面$S$的方程,而$S$为$F(x,y,z)=0$的图形

曲线则可以看作两个曲面的交线,及同时满足$F(x,y,z)=0$与$G(x,y,z)=0$的点集

\subsection*{法线 \& 平面的法线方程}
法线向量是垂直于所在点切面的一非零向量

任意过点$M_0(x_0,y_0,z_0)$且有法向量$\ve{n}=\veN{A}{B}{C}$的平面可表示为
\[F(x,y,z)=\ve{n}\cdot\veN{x-x_0}{y-y_0}{z-z_0}=0\]

\subsection*{平面的一般方程}
平面的一般方程是
\[F(x,y,z)=Ax+By+Cz+D=0\]
平面的截距式
\[F(x,y,z)=\frac{x}{a}+\frac{y}{b}+\frac{z}{c}-1=0\]
其中$a$,$b$与$c$依次为平面在$x$,$y$与$z$轴上的截距

\subsection*{平面的夹角}
两平面的法线向量的夹角称为这两个平面的夹角$\theta=\ang{n_1}{n_2}$
\bigskip
\bigskip
\section*{空间直线}

\bigskip

\subsection*{空间直线的一般方程}
空间中的一条直线可表示为满足方程组
\[\begin{cases}
A_1x+B_1y+C_1z+D_1=0\\
A_2x+B_2y+C_2z+D_2=0
\end{cases}\]
的点集

\subsection*{对称方程}
如果一个非零向量平行于一条直线,那么这个向量为这条直线的方向向量

如果点$M(x,y,z)$在直线$L$上,而$L$过点$M_0(x_0,y_0,z_0)$且有方向向量$\ve{s}=\veN{m}{n}{p}$,则有
\[\veN{x-x_0}{y-y_0}{z-z_0}=\lambda\ve{s}\quad\lambda\in\mathbb{R}\]
或
\[\frac{x-x_0}{m}=\frac{y-y_0}{n}=\frac{z-z_0}{p}\]

直线的任一方向向量的坐标称为该直线的方向数,其对应的方向余弦称为该直线的方向余弦
\subsection*{参数方程}
\[\begin{cases}
x=x_0+mt\\
y=y_0+mt\\
z=z_0+mt
\end{cases}\]

\subsection*{夹角}
直线与直线的夹角为它们方向向量的夹角

直线与平面的夹角为方向向量与法向量的夹角

\section*{曲面及其方程}

\subsection*{旋转面}
设被旋转的曲线(母线)在$yOz$平面上,其方程为
\[f(y,z)=0\]
绕$z$轴旋转(以$z$轴为轴),得旋转曲面$S$为
\[ f(\pm\sqrt{x^2+y^2},z)=0 \]

\subsection*{柱面}
直线$L$沿定曲线$C$平行移动形成的轨迹,叫做柱面,定曲线$C$叫做柱面的准线,动直线$L$叫做主线的母线

\subsection*{二次曲面}
圆锥曲面 (Cone cylinder):
\[\frac{x^2}{a^2}+\frac{y^2}{b^2}=z^2\]

平面$z=c$与曲面$F(x,y,z)=0$的交线称为截痕 (contour map)

椭球面 (Ellipsoid cylinder):
\[\frac{x^2}{a^2}+\frac{y^2}{b^2}+\frac{z^2}{c^2}=1\]

单叶双曲面 (one sheet Hyperboloid):
\[\frac{x^2}{a^2}+\frac{y^2}{b^2}-\frac{z^2}{c^2}=1\]

双叶双曲面 (two sheet Hyperboloid):
\[\frac{x^2}{a^2}-\frac{y^2}{b^2}-\frac{z^2}{c^2}=1\]

椭圆抛物面 (Paraboloid elliptic):
\[\frac{x^2}{a^2}+\frac{y^2}{b^2}=z\]

双曲抛物面(马鞍面) (Paraboloid hyperbolic):
\[\frac{x^2}{a^2}-\frac{y^2}{b^2}=z\]

\subsection*{曲面的参数方程}
曲面可看作点集
\[G(u,v) = (x(u,v),y(u,v),z(u,v))\]
其中$u$与$v$为参数

其法向量为
\[\vec{N} =  \frac{\partial G}{\partial u}\times\frac{\partial G}{\partial v}\]

\section*{空间曲线}
\subsection*{一般方程}
\[\begin{cases}
F(x,y,z)=0\\
G(x,y,z)=0
\end{cases}\]
\subsection*{参数方程}
\[\begin{cases}
x=x(t)\\
y=y(t)\\
z=z(t)
\end{cases}\]
\subsection*{在坐标平面上的投影}
如果将空间曲线$C$可消去变量$z$以后,所得方程$H(x,y)=0$是一个母线平行于$z$轴的柱面,称为$C$在$xOy$面上的投影柱面,
柱面与$xOy$的交线叫做空间曲线$C$在$xOy$上的投影曲线


\end{document}
